\clearpage
\chapter*{ABSTRAK}
\addcontentsline{toc}{chapter}{Abstrak}

Dalam aplikasi dengan arsitektur Microservice, ketika terjadi perubahan yang mengakibatkan penurunan kinerja atau regresi, sulit untuk melakukan analisis atau pemeriksaan bagian mana dari keseluruhan Microservice yang menjadi penyebab utama regresi tersebut karena sifat terdistribusi dari Microservice. Dengan bantuan dari \textit{distributed tracing} dapat dilakukan analisis untuk menentukan apakah terjadi regresi dan penyebab regresi dengan memanfaatkan data \textit{latency} dari operasi-operasi yang ada pada aplikasi Microservice.

Tugas Akhir ini mengimplementasikan sistem Analisis Regresi Kinerja atau \textit{Performance Regression Analysis} dengan menggunakan kakas \textit{distributed tracing} Zipkin. Pendeteksian regresi dilakukan dengan menggunakan analisis statistik Kolmogorov-Smirnov yang akan membandingkan sampel data \textit{latency} yang terjadi secara periodikal dan sampel data \textit{latency} \textit{baseline} yang merepresentasikan kinerja aplikasi dalam keadaan normal. Analisis akan membandingkan \textit{Cummulative Distribution Function} (CDF) dari kedua sampel tersebut dan melihat apakah kedua CDF berasal dari distribusi yang berbeda. Jika didapat bahwa kedua CDF berasal dari distribusi yang berbeda, maka dapat dicurigai telah terjadi regresi kinerja sebab distribusi dari data periodikal telah menyimpang dari distribusi data \textit{baseline}.

Jika regresi terdeteksi, maka selanjutnya sistem akan melakukan analisis \textit{critical path} yaitu melihat operasi mana saja dari setiap \textit{service} yang berkemungkinan besar berkontribusi pada regresi tersebut. Analisis akan dilakukan dengan mencari selisih \textit{latency} dari data operasi \textit{real-time} dan \textit{baseline} dan akan terlihat mana operasi yang selisih \textit{latency} nya melebihi batas yang telah ditentukan sebelumnya.

Sistem \textit{Performance Regression Analysis} telah diujikan pada aplikasi Microservice yang dijalankan di Kubernetes. Hasilnya adalah dari 10 dari 11 kasus uji, regresi berhasil dideteksi dan pada 6 dari 10 kasus tersebut berhasil ditentukan operasi penyebab regresi. Pengimplementasian sistem menambah \textit{overhead} rata-rata penggunaan CPU sebesar 0,78\% dan penggunaan Memory sebesar 0,67\%.



\vspace{10mm}
Kata kunci: Regresi Kinerja, \textit{Distributed Tracing}, Microservice, Kubernetes
\clearpage