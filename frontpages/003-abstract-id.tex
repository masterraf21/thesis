\clearpage
\chapter*{ABSTRAK}
\addcontentsline{toc}{chapter}{Abstrak}

Dengan banyaknya adopsi penggunaan arsitektur Microservice yang berbasis sistem terdistribusi saat ini, semakin banyak juga tantangan yang muncul terkait dengan adopsi gaya arsitektur tersebut salah satunya terkait dengan \textit{observability}. Arsitektur Microservice yang membagi-bagi aplikasi menjadi \textit{service} yang kecil dapat membuat tugas \textit{monitoring} seperti \textit{debugging} dan \textit{logging} menjadi sulit karena sifat nya yang terdistribusi. Jika terjadi penurunan kinerja atau regresi pada aplikasi dengan arsitektur Microservice, apabila metode \textit{debugging} tradisional digunakan maka akan menjadi tantangan sendiri untuk mencari penyebab dari regresi tersebut pada \textit{service}-\textit{service} yang tersebar terlebih jika banyaknya \textit{service} sudah mencapai ratusan ataupun ribuan. 

Adanya kakas \textit{distributed tracing} yang dapat mengambil data  \textit{trace} dari setiap \textit{request} yang dibuat oleh \textit{service} dan hasilnya secara terpusat dapat menjadi metode baru untuk menganalisis penyebab dari regresi yang terjadi pada aplikasi dengan arsitektur Microservice.




\vspace{15mm}
Kata kunci: \textit{Distributed Tracing}, Microservice, regresi kinerja, Kubernetes
\clearpage