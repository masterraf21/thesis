\clearpage
\chapter*{ABSTRAK}
\addcontentsline{toc}{chapter}{Abstrak}

Analisis regresi kinerja atau \textit{Performance Regression Analysis} adalah sebuah teknik \textit{debugging} yang bertujuan untuk mendeteksi dan mencari sumber penyebab regresi atau penurunan kinerja pada aplikasi dengan arsitektur Microservice. Pendekatan baru dalam melakukan \textit{debugging} diperlukan pada aplikasi dengan arsitektur Microservice karena sifat dari Microservice itu sendiri yang terdistribusi sehingga jika menggunakan pendekatan \textit{debugging} yang tradisional dengan mencari penyebab regresi masing-masing pada setiap \textit{service} akan menjadi tidak efisien. Oleh karena itu dibutuhkan pendekatan yang dapat menganalisis regresi kinerja  

%Dengan banyaknya adopsi penggunaan arsitektur Microservice yang berbasis sistem terdistribusi saat ini, semakin banyak juga tantangan yang muncul terkait dengan adopsi gaya arsitektur tersebut salah satunya terkait dengan \textit{observability}. Arsitektur Microservice yang membagi-bagi aplikasi menjadi \textit{service} yang kecil dapat membuat tugas \textit{monitoring} seperti \textit{debugging} dan \textit{logging} menjadi sulit karena sifat nya yang terdistribusi. Jika terjadi penurunan kinerja atau regresi pada aplikasi dengan arsitektur Microservice, apabila metode \textit{debugging} tradisional digunakan maka akan menjadi tantangan sendiri untuk mencari penyebab dari regresi tersebut pada \textit{service}-\textit{service} yang tersebar terlebih jika banyaknya \textit{service} sudah mencapai ratusan ataupun ribuan. 
%
%Adanya kakas \textit{distributed tracing} yang dapat mengambil data  \textit{trace} dari setiap \textit{request} yang dibuat oleh \textit{service} dan menyimpan hasilnya secara terpusat dapat menjadi metode baru untuk menganalisis penyebab dari regresi yang terjadi pada aplikasi dengan arsitektur Microservice.
%
%Analisis regresi akan dilakukan dengan membandingkan sampel data \textit{latency} dari kasus terjadinya regresi dengan sampel data \textit{baseline} yang merepresentasikan keadaan normal dari aplikasi, sehingga jika hasil perbandingan menunjukkan bahwa kedua sampel data tersebut berasal dari distribusi yang berbeda, dapat menjadi indikasi bahwa telah terjadi regresi kinerja.





\vspace{15mm}
Kata kunci: \textit{Distributed Tracing}, Microservice, regresi kinerja, Kubernetes
\clearpage