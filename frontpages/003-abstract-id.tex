\clearpage
\chapter*{ABSTRAK}
\addcontentsline{toc}{chapter}{Abstrak}
% \begin{center}
% \MakeTextUppercase{\textbf{\large{\thetitle}}}

% Oleh

% \MakeTextUppercase{\theauthor}
% \end{center}
% \medskip
% \begin{spacing}{1.0}
Pengumpulan data merupakan proses kerja yang sangat penting yang sering ditemui pada kehidupan sehari-hari. Pengumpulan data biasanya dilakukan menggunakan aplikasi \textit{spreadsheet}. Hal ini disebabkan oleh mudahnya penggunaan \textit{spreadsheet} sehingga banyak orang awam yang memilih menggunakan \textit{spreadsheet} dibandingkan basis data. Pengumpulan data menggunakan aplikasi \textit{spreadsheet} memiliki beberapa kelemahan seperti lemahnya validasi, terisolasinya data yang dikumpulkan, serta terdapat kemungkinan sulitnya berkolaborasi dalam pengumpulan data. Dari permasalahan tersebut, dibuat kakas pengumpulan data berbasis \textit{spreadsheet} yang diharapkan dapat memudahkan pengguna dalam melakukan pengumpulan data dan mengatasi permasalahan-permasalahan yang sering terjadi.

Kakas pengumpulan data ini diimplementasi sebagai fitur tambahan pada aplikasi EtherCalc sehingga permasalahan kolaborasi akan ditangani oleh aplikasi EtherCalc tersebut. Selanjutnya, pengguna dapat mendefinisikan \textit{metadata table} secara manual atau juga dilakukan secara otomatis oleh kakas menggunakan algoritma framefinder yang telah dibuat oleh penelitian lain. Dari \textit{metadata table} tersebut, pengguna dapat melakukan perubahan aturan-aturan validasi yang dibagi menjadi tiga tipe validasi yakni, tipe data, domain data, dan relasi antar data. Kakas akan menerjemahkan \textit{metadata table} yang dibuat serta mencocokkannya dengan data yang ada pada \textit{spreadsheet}, lalu memasukkan data tersebut ke dalam basis data relasional.

Pengujian dilakukan pada fitur-fitur yang diimplementasikan pada kakas. Pengujian dilakukan dengan menggunakan beberapa kasus yang dibuat dan diujikan kebenaran hasil data masukan menjadi data pada basis data.

\vspace{15mm}
Kata kunci: \textit{spreadsheet}, pengumpulan data, \textit{data quality}, \textit{data management}.
\clearpage