\chapter*{Kata Pengantar}
\addcontentsline{toc}{chapter}{Kata Pengantar}

Puji syukur saya panjatkan kehadirat Tuhan Yang Maha Esa, karena degan kelimpahan rahmat dan kemurahan hati-Nya penulis dapat menyelesaikan Tugas Akhir yang berjudul "\thetitle".

Dalam penyusunan Tugas Akhir ini penulis banyak mendapatkan masukan, kritik, dorongan, bantuan, bimbingan, serta dukungan baik secara fisik maupun moral dari berbagai pihak yang merupakan pengalaman yang berharga yang tidak dapat diukur secara materi dan dapat menjadi pembelajaran yang berharga dikemudian hari. Oleh karena itu dengan segala hormat dan kerendahan hati perkenankanlah penulis mengucapkan terima kasih kepada:

\begin{enumerate}
	\item Bapak Achmad Imam Kistijantoro, S.T, M.Sc., Ph.D., selaku pembimbing yang senantiasa memberikan arahan dan masukan selama pengerjaan Tugas Akhir.
	\item Bapak Riza Satria Perdana, S.T, M.T. selaku dosen penguji yang atas saran dan masukkannya membuat Tugas Akhir ini menjadi lebih baik.
	\item Kedua orang tua penulis. Terima kasih atas dukungan baik secara moral dan material sehingga penulis dapat melaksanakan Tugas Akhir ini hingga selesai dengan baik.
	\item Asiya Mufida Yumna yang selalu menyemangati, menjadi inspirasi, dan mendukung penulis selama pengerjaan Tugas Akhir.	
	\item Xavier Prasetyo sebagai teman satu bimbingan yang t
	\item Teman-teman penulis lainnya yang tidak bisa penulis sebutkan satu-persatu yang telah membantu penulis selama pengerjaan Tugas Akhir ini.
\end{enumerate}

Penulis menyadari bahwa Tugas Akhir ini masih jauh dari sempurna serta memiliki banyak kekurangan. Oleh karena itu, penulis sangat terbuka dalam menerima kritik dan saran yang membangun untuk Tugas Akhir ini. Semoga Tugas Akhir ini dapat bermanfaat bagi pembaca.

%\vspace{15mm}
%\begin{tabularx}{\textwidth}{l@{\hskip 0.5\textwidth}l}
%	 & Bandung, \thedate{} \yearsidang{} \\
%	 & Penulis
%\end{tabularx}
% \begin{flushright}
% Bandung, Agustus 2017 \\
% \vspace{25mm}
% Penulis
% \end{flushright}