\chapter{Lampiran B. Detail Pengujian}

\section{Skenario Pengujian}
Skenario pengujian dilakukan berdasarkan kasus uji yang sudah dijabarkan pada Bab 4. Untuk setiap kasus uji, dapat mempunyai satu atau lebih skenario. Tiap skenario diberi kode TC-XX-YY dengan TC-XX adalah ID kasus uji dan YY adalah nomor skenario terkait ID kasus uji. Skenario pengujian dapat dilihat pada Tabel \ref{SkenarioUji}.

\begin{small}
\begin{longtable}{ | p{2cm} | p{4cm} | p{7cm} | }
    \caption{Skenario Pengujian}
    \label{SkenarioUji}\\ \hline
    \centering\bfseries{ID Skenario} & \centering\bfseries{Deskripsi} & \centering\bfseries{Prosedur} \tabularnewline \hline
    \endfirsthead
    \hline
    \centering\bfseries{ID Skenario} & \centering\bfseries{Deskripsi} & \centering\bfseries{Prosedur} \tabularnewline \hline
    \endhead
    TC-01-01 & Validasi koneksi kepada pengaturan basis data yang diberikan dapat dilakukan saat konfigurasi benar & 1. Masukkan konfigurasi basis data pada kolom yang tersedia. Pada pengujian, basis data yang dicoba berada di Docker dengan konfigurasi\newline Host: 172.17.0.3 atau 172.17.0.2\newline Port: 3306\newline Username: root\newline Password: root\newline Database: TA\newline2. Tekan tombol `Connect`\\ \hline

    TC-01-02 & Memberikan pesan kesalahan saat konfigurasi basis data tidak benar & 1. Masukkan konfigurasi kosong pada kolom yang disediakan\newline 2. Tekan tombol `Connect`"\\ \hline

    TC-01-03 & Menyimpan dan mengambil data \textit{metadata table} yang telah dibuat ke basis data & 1. Menambahkan konfigurasi tabel baru secara manual dengan masukkan;\newline \{\newline  "header":[4],\newline  "data":"5:13",\newline  "range":[1,2,3,4,5,6,7]\newline \}\newline 2. Tambahkan manual\newline 3. Tutup browser\newline 4. Koneksikan kakas ke basis data yang sebelumnya\newline 5. Buka kembali browser ke spreadsheet tadi\\ \hline 

	TC-02-01 & Menampilkan hasil pendeteksian label dan data secara otomatis & 1. Menekan tombol `Detect Spreadsheet Table` pada antarmuka\\ \hline
	TC-02-02 & Pendeteksian header dengan hirarki dua atau lebih & 1. Buat tabel yang dapat dideteksi dan memiliki header dengan hirarki dua atau lebih\newline 2. Menekan tombol `Detect Spreadsheet Table` pada antarmuka\\ \hline 

	TC-03-01 & Membuat \textit{metadata table} baru sesuai masukan manual pengguna dimana masukan sesuai dengan aturan & 1. Menambahkan konfigurasi tabel baru secara manual dengan masukkan;\newline \{\newline  "header":[4],\newline  "data":"5:13",\newline  "range":[1,2,3,4,5,6,7]\newline \}\newline 2. Tekan tombol `Add New Table`\\ \hline 
	TC-03-02 & Membuat \textit{metadata table} baru sesuai masukan manual pengguna dimana masukan tidak sesuai dengan aturan & 1. Menambahkan konfigurasi tabel baru secara manual dengan masukkan;\newline \{\newline  "header":[5,6],\newline  "range":"A9:F11"\newline \}\newline 2. Tekan tombol `Add New Table`\\ \hline 

	TC-04-01 & Mengubah atribut nama label pada baris menggunakan \textit{metadata table} & 1. Tambahkan \textit{metadata table} baik secara manual atau otomatis\newline 2. Ubah kolom label pada \textit{metadata table}\newline 3. Tekan tombol `Save` pada \textit{metadata table}\newline 4. Tutup browser\newline 5. Buka kembali browser ke spreadsheet tadi\\ \hline 
	TC-04-02 & Mengubah atribut 'kolom data' pada \textit{metadata table} & 1. Tambahkan \textit{metadata table} baik secara manual atau otomatis\newline 2. Ubah kolom data pada \textit{metadata table}\newline 3. Tekan tombol `Save` pada \textit{metadata table}\newline 4. Tutup browser\newline 5. Buka kembali browser ke spreadsheet tadi\\ \hline 
	TC-04-03 & Mengubah tipe data pada suatu kolom menggunakan \textit{metadata table} & 1. Tambahkan \textit{metadata table} baik secara manual atau otomatis\newline 2. Ubah kolom tipe data pada \textit{metadata table}\newline 3. Tekan tombol `Save` pada \textit{metadata table}\newline 4. Tutup browser\newline 5. Buka kembali browser ke spreadsheet tadi\\ \hline 
	TC-04-04 & Mengubah nilai yang diizinkan pada suatu kolom menggunakan \textit{metadata table} & 1. Tambahkan \textit{metadata table} baik secara manual atau otomatis\newline 2. Ubah kolom permitted values pada \textit{metadata table}\newline 3. Tekan tombol `Save` pada \textit{metadata table}\newline 4. Tutup browser\newline 5. Buka kembali browser ke spreadsheet tadi\\ \hline 
	TC-04-05 & Mengubah relasi yang diinginkan pada suatu kolom menggunakan \textit{metadata table} & 1. Tambahkan \textit{metadata table} baik secara manual atau otomatis\newline 2. Ubah kolom relation pada \textit{metadata table} menjadi range cell atau suatu tabel dan kolom unik pada basis data\newline 3. Tekan tombol `Save` pada \textit{metadata table}\newline 4. Tutup browser\newline 5. Buka kembali browser ke spreadsheet tadi\\ \hline 
	TC-04-06 & Menghapus \textit{metadata table} & 1. Tambahkan \textit{metadata table} baik secara manual atau otomatis\newline 2. Hapus \textit{metadata table} yang sudah ditambahkan\newline 3. Tekan tombol `Save` pada \textit{metadata table}\newline 4. Tutup browser\newline 5. Buka kembali browser ke spreadsheet tadi\\ \hline 
	TC-04-07 & Menentukan kolom key & 1. Tambahkan \textit{metadata table} baik secara manual atau otomatis\newline 2. Centang pilihan tipe key pada suatu kolom\newline 3. Tekan tombol `Save` pada \textit{metadata table}\newline 4. Tekan tombol `Save to Database`\\ \hline 

	TC-05-01 & Data dimasukkan ke basis data pertama kali dimana belum ada data sebelumnya & 1. Tambahkan \textit{metadata table} baik secara manual atau otomatis\newline 2. Tekan tombol `Save to Database`\\ \hline 
	TC-05-02 & Nama pada dua \textit{metadata table} pada spreadsheet yang sama, saling ditukar satu dengan yang lain. Contohnya, tabel 1 bernama A dan tabel 2 bernama B menjadi tabel 1 bernama B dan tabel 2 bernama A. & 1. Tambahkan \textit{metadata table} baik secara manual atau otomatis sehingga minimal terdapat dua tabel\newline 2. Tekan tombol `Save to Database`\newline 3. Tukar nama kedua tabel tersebut\newline 4. Tekan tombol `Save to Database`\\ \hline 
	TC-05-03 & Spreadsheet lain menggunakan nama tabel yang sama dengan konfigurasi kolom yang sama, tanpa kolom key & 1. Tambahkan \textit{metadata table} baik secara manual atau otomatis, tanpa kolom key\newline 2. Tekan tombol `Save to Database`\newline 3. Pindah ke spreadsheet lain dan tambahkan \textit{metadata table} baru dengan nama dan jumlah kolom yang sama\newline 4. Tekan tombol `Save to Database`\newline \\ \hline 
	TC-05-04 & Spreadsheet lain menggunakan nama tabel yang sama dengan konfigurasi kolom yang sama, dengan nama kolom key yang sama & 1. Tambahkan \textit{metadata table} baik secara manual atau otomatis, dengan satu kolom key\newline 2. Tekan tombol `Save to Database`\newline 3. Pindah ke spreadsheet lain dan tambahkan \textit{metadata table} baru dengan nama dan jumlah kolom yang sama dan sebuah kolom key\newline 4. Tekan tombol `Save to Database`\newline \\ \hline 
	TC-05-05 & Spreadsheet lain menggunakan nama tabel yang sama dengan konfigurasi kolom yang sama, dengan nama kolom key yang berbeda & 1. Tambahkan \textit{metadata table} baik secara manual atau otomatis, dengan kolom key\newline 2. Tekan tombol `Save to Database`\newline 3. Pindah ke spreadsheet lain dan tambahkan \textit{metadata table} baru dengan nama dan jumlah kolom yang sama dan kolom unik yang berbeda\newline 4. Tekan tombol `Save to Database`\newline \\ \hline 
	TC-05-06 & Spreadsheet lain menggunakan nama tabel yang sama dengan konfigurasi kolom yang berbeda dimana sudah ada spreadsheet lain yang menggunakan, tanpa kolom key & 1. Tambahkan \textit{metadata table} baik secara manual atau otomatis, tanpa kolom key\newline 2. Tekan tombol `Save to Database`\newline 3. Pindah ke spreadsheet lain dan tambahkan \textit{metadata table} baru dengan nama dan jumlah kolom yang berbeda seluruhnya tanpa kolom key\newline 4. Tekan tombol `Save to Database`\newline \\ \hline 
	TC-05-07 & Spreadsheet lain menggunakan nama tabel yang sama dengan konfigurasi kolom yang berbeda dimana sudah ada spreadsheet lain yang menggunakan, dengan nama kolom key yang sama & 1. Tambahkan \textit{metadata table} baik secara manual atau otomatis, dengan sebuah kolom key\newline 2. Tekan tombol `Save to Database`\newline 3. Pindah ke spreadsheet lain dan tambahkan \textit{metadata table} baru dengan nama dan jumlah kolom yang berbeda, dengan satu kolom key yang sama\newline 4. Tekan tombol `Save to Database`\newline \\ \hline 
	TC-05-08 & Spreadsheet lain menggunakan nama tabel yang sama dengan konfigurasi kolom yang berbeda dimana sudah ada spreadsheet lain yang menggunakan, dengan nama kolom unik yang berbeda & 1. Tambahkan \textit{metadata table} baik secara manual atau otomatis, dengan suatu kolom unik\newline 2. Tekan tombol `Save to Database`\newline 3. Pindah ke spreadsheet lain dan tambahkan \textit{metadata table} baru dengan nama dan jumlah kolom yang berbeda, dan dengan kolom unik yang berbeda juga\newline 4. Tekan tombol `Save to Database`\newline \\ \hline 
	TC-05-09 & Menggunakan nama tabel yang sama dengan konfigurasi kolom yang berbeda dimana belum ada spreadsheet lain yang menggunakan & 1. Tambahkan \textit{metadata table} baik secara manual atau otomatis\newline 2. Tekan tombol `Save to Database`\newline 3. Ganti nama \textit{metadata table} sebelumnya menjadi nama lain lalu gunakan \textit{metadata table} lain dengan kolom yang berbeda menjadi nama tabel lama.\newline 4. Tekan tombol `Save to Database`\newline \\ \hline 
	TC-05-10 & Penghapusan suatu konfigurasi tabel & 1. Tambahkan \textit{metadata table} baik secara manual atau otomatis\newline 2. Tekan tombol `Save to Database`\newline 3. Hapus \textit{metadata table}\newline 4. Tekan tombol `Save to Database`\newline \\ \hline 
	TC-05-11 & Penggantian nama tabel pada \textit{metadata table} & 1. Tambahkan \textit{metadata table} baik secara manual atau otomatis\newline 2. Tekan tombol `Save to Database`\newline 3. Ganti nama \textit{metadata table} sebelumnya menjadi nama lain.\newline 4. Tekan tombol `Save to Database`\newline \\ \hline 
	TC-05-12 & Memasukkan data dengan fragmen gabungan pada dua atau lebih spreadsheet secara tidak beraturan, namun data yang ada tidak saling menimpa & Terdapat data yang akan dimasukkan berupa NIM, Nilai 1, Nilai 2, dan Nilai 3. Terdapat tiga kelompok NIM. \newline 1. Dibuat spreadsheet A yang akan mengisi data pada NIM kelompok 1 untuk Nilai 1 dan 3, dan NIM kelompok 3 untuk Nilai 1.\newline 2. Dibuat spreadsheet B yang akan mengisi data pada NIM kelompok 1 untuk Nilai 2, dan NIM kelompok 2 untuk Nilai 3.\newline 3. Dibuat spreadsheet C yang akan mengisi data pada NIM kelompok 2 untuk Nilai 1 dan 2, dan NIM kelompok 3 untuk Nilai 2.\newline 4. Setiap spreadsheet dimasukkan datanya ke dalam basis data dengan urutan yang berbeda-beda\\ \hline 

	TC-06-01 & Memvalidasi tipe data Integer & 1. Tambahkan \textit{metadata table} baik secara manual atau otomatis\newline 2. Pada \textit{metadata table} yang telah ada, ubah tipe data kolom menjadi Integer\newline 3. Tekan tombol `Save to Database`\\ \hline 
	TC-06-02 & Memvalidasi tipe data Double & 1. Tambahkan \textit{metadata table} baik secara manual atau otomatis\newline 2. Pada \textit{metadata table} yang telah ada, ubah tipe data kolom menjadi Double\newline 3. Tekan tombol `Save to Database`\\ \hline 
	TC-06-03 & Memvalidasi tipe data String & 1. Tambahkan \textit{metadata table} baik secara manual atau otomatis\newline 2. Pada \textit{metadata table} yang telah ada, ubah tipe data kolom menjadi String\newline 3. Tekan tombol `Save to Database`\\ \hline 
	TC-06-04 & Memvalidasi tipe data Boolean & 1. Tambahkan \textit{metadata table} baik secara manual atau otomatis \newline 2. Pada \textit{metadata table} yang telah ada, ubah tipe data kolom menjadi Boolean\newline 3. Tekan tombol `Save to Database`\\ \hline 
	TC-06-05 & Memvalidasi masukan yang diizinkan berupa nilai diskrit & 1. Tambahkan \textit{metadata table} baik secara manual atau otomatis \newline 2. Pada \textit{metadata table} yang telah ada, ubah kolom permitted value menjadi ["Alabama", "Seattle"]\newline 3. Tekan tombol `Save to Database`\\ \hline 
	TC-06-06 & Memvalidasi masukan yang diizinkan berupa nilai range & 1. Tambahkan \textit{metadata table} baik secara manual atau otomatis \newline 2. Pada \textit{metadata table} yang telah ada, ubah kolom permitted value menjadi 10-1000\newline 3. Tekan tombol `Save to Database`\\ \hline 
	TC-06-07 & Memvalidasi masukan yang diizinkan dengan syarat kurang dari & 1. Tambahkan \textit{metadata table} baik secara manual atau otomatis \newline 2. Pada \textit{metadata table} yang telah ada, ubah kolom permitted value menjadi \( < 1000 \)\newline 3. Tekan tombol `Save to Database`\\ \hline 
	TC-06-08 & Memvalidasi masukan yang diizinkan dengan syarat kurang dari sama dengan & 1. Tambahkan \textit{metadata table} baik secara manual atau otomatis \newline 2. Pada \textit{metadata table} yang telah ada, ubah kolom permitted value menjadi \( \leq 1000 \)\newline 3. Tekan tombol `Save to Database`\\ \hline 
	TC-06-09 & Memvalidasi masukan yang diizinkan dengan syarat lebih dari & 1. Tambahkan \textit{metadata table} baik secara manual atau otomatis \newline 2. Pada \textit{metadata table} yang telah ada, ubah kolom permitted value menjadi \( > 1000 \)\newline 3. Tekan tombol `Save to Database`\\ \hline 
	TC-06-10 & Memvalidasi masukan yang diizinkan dengan syarat lebih dari sama dengan & 1. Tambahkan \textit{metadata table} baik secara manual atau otomatis \newline 2. Pada \textit{metadata table} yang telah ada, ubah kolom permitted value menjadi \( \geq 1000 \)\newline 3. Tekan tombol `Save to Database`\\ \hline 
	TC-06-11 & Memvalidasi relasi antar sel & 1. Tambahkan \textit{metadata table} baik secara manual atau otomatis \newline 2. Pada \textit{metadata table} yang telah ada, ubah kolom relasi menjadi suatu range sel. Contohnya A8:B9\newline 3. Tekan tombol `Save to Database`\\ \hline 
	TC-06-12 & Memvalidasi relasi dengan tabel & 1. Tambahkan \textit{metadata table} baik secara manual atau otomatis \newline 2. Pada \textit{metadata table} yang telah ada, ubah kolom relasi menjadi target tabel dan kolom. Contohnya ["nama\_tabel", "nama\_kolom"] \newline 3. Tekan tombol `Save to Database` \\ \hline 
	TC-06-13 & Memvalidasi konflik yang terjadi jika terjadi penyimpanan & 1. Buat suatu spreadsheet dan masukkan data menggunakan \textit{metadata table} dan memiliki kolom unik\newline 2. Buat spreadsheet lain dengan nama kolom yang sama dan kolom unik yang juga sama.\newline 3. Tekan tombol `Check Conflict` untuk memastikan adanya konflik\newline 4. Ubah nama kolom yang bukan merupakan kolom unik pada spreadsheet kedua\newline 5. Tekan kembali tombol `Check Conflict`\\ \hline 

	% TC-07-01 & Pencatatan waktu eksekusi yang dibutuhkan untuk terkoneksi dan mengambil data konfigurasi dari basis data & 1. Masukkan konfigurasi basis data pada kolom yang tersedia. Pada pengujian, basis data yang dicoba berada di Docker dengan konfigurasi;\newline * Host: 172.17.0.3 atau 172.17.0.2\newline * Port: 3306\newline * Username: root\newline * Password: root\newline * Database: TA\newline 2. Tekan tombol `Connect`\\ \hline 
	% TC-07-02 & Pencatatan waktu eksekusi yang dibutuhkan untuk melakukan pencarian label dan data secara otomasi & 1. Menekan tombol `Detect Spreadsheet Table` pada antarmuka\\ \hline 
	% TC-07-03 & Pencatatan waktu eksekusi yang dibutuhkan untuk menambahkan \textit{metadata table} secara manual & 1. Menambahkan konfigurasi tabel baru secara manual \newline 2. Tekan tombol `Add New Table`\\ \hline 
	% TC-07-04 & Pencatatan waktu eksekusi yang dibutuhkan untuk menyimpan data ke basis data & 1. Tambahkan \textit{metadata table} baik secara manual atau otomatis\newline 2. Pada \textit{metadata table} yang telah ada, ubah tipe data kolom menjadi Integer\newline 3. Tekan tombol `Save to Database`\newline 4. Lakukan untuk berbagai jenis ukuran dan jumlah tabel\\ \hline 
\end{longtable}
\end{small}

\section{Hasil Pengujian}
Hasil pengujian berkaitan dengan ID skenario pengujian yang telah dijabarkan sebelumnya. Tiap hasil pengujian mempunyai ID skenario terkait, ekspektasi, hasil uji yang dilakukan, dan diterima atau tidaknya hasil pengujian. Hasil pengujian dapat dilihat pada Tabel \ref{HasilUji}.

\begin{small}
\begin{longtable}{ | p{2cm} | p{4cm} | p{4cm} | p{2cm} | }
    \caption{Hasil Pengujian}
    \label{HasilUji}\\ \hline
    \centering\bfseries{ID Skenario} & \centering\bfseries{Ekspektasi} & \centering\bfseries{Hasil} & \centering\bfseries{Diterima} \tabularnewline \hline
    \endfirsthead
    \hline
    \centering\bfseries{ID Skenario} & \centering\bfseries{Ekspektasi} & \centering\bfseries{Hasil} & \centering\bfseries{Diterima} \tabularnewline \hline
    \endhead
    TC-01-01 & Kakas dapat terkoneksi ke basis data dan dapat melakukan perintah-perintah SQL kepada basis data yang dipilih & Berhasil melakukan koneksi ke basis data dan dapat melakukan perintah-perintah SQL yang dibutuhkan, serta dapat menginisiasi tabel kosong dengan konfigurasi yang dibutuhkan. & Ya
	\\ \hline TC-01-02 & Muncul pesan kesalahan yang menjelaskan bahwa koneksi belum dapat dilakukan & Muncul pesan ‘Cannot Established Connection to Database’ pada layar dan informasi untuk mencoba memperbaiki pengaturan & Ya
	\\ \hline TC-01-03 & \textit{metadata table} yang pernah dibuat, muncul kembali walaupun aplikasi ditutup & Konfigurasi terakhir, muncul kembali setelah terkoneksi ke basis data yang sama. & Ya

	\\ \hline TC-02-01 & Muncul \textit{metadata table} sesuai dengan label dan data yang terdeteksi & Pada kasus yang dapat ditangani oleh model, muncul tabel-\textit{metadata table} sesuai dengan jumlah tabel, letak header, dan data yang di deteksi. Jika tidak, muncul pesan ‘No Table Detected’ & Ya
	\\ \hline TC-02-02 & Header yang berhirarki akan bernama header1\_header2 dan seterusnya & Header berhirarki berhasil dideteksi, sesuai dengan tingkat hirarkinya. Tingkat hirarki yang dicoba adalah satu, dua, dan tiga tingkat & Ya

	\\ \hline TC-03-01 & \textit{metadata table} baru ditambahkan ke layar dan sesuai dengan masukkan pengguna & \textit{metadata table} berhasil ditambahkan dan ditampilkan kepada pengguna & Ya
	\\ \hline TC-03-02 & Muncul pesan kesalahan dan tidak muncul tambahan \textit{metadata table} & Muncul pesan kesalahan berupa kesalahan yang terjadi karena tidak adanya bagian data dan \textit{metadata table} tidak ditambahkan & Ya

	\\ \hline TC-04-01 & Label pada \textit{metadata table} berubah dan tersimpan pada basis data & Label yang disimpan, muncul kembali walaupun browser telah ditutup dan konfigurasinya tersimpan dibasis data. Pada saat penyimpanan, tabel dibentuk dengan nama kolom yang sama dengan nama label & Ya
	\\ \hline TC-04-02 & Kolom data pada \textit{metadata table} berubah dan tersimpan pada basis data & Perubahan kolom data yang disimpan, muncul kembali walaupun browser telah ditutup dan konfigurasinya tersimpan dibasis data. Pada saat penyimpanan, data yang masuk sesuai dengan kolom data yang dipilih & Ya
	\\ \hline TC-04-03 & Tipe data pada \textit{metadata table} berubah dan tersimpan pada basis data & Perubahan tipe data yang disimpan, muncul kembali walaupun browser telah ditutup dan konfigurasinya tersimpan dibasis data & Ya
	\\ \hline TC-04-04 & Nilai yang diizinkan pada \textit{metadata table} berubah dan tersimpan pada basis data & Perubahan nilai yang diizinkan (permitted value) berhasil disimpan dan muncul kembali walaupun browser telah ditutup & Ya
	\\ \hline TC-04-05 & Relasi pada \textit{metadata table} berubah dan tersimpan pada basis data & Perubahan rujukan relasi berhasil disimpan dan muncul kembali walaupun browser telah ditutup & Ya
	\\ \hline TC-04-06 & \textit{metadata table} yang dihapus tidak muncul lagi dan terhapus dari basis data. Data yang terhapus dari basis data hanya data yang berhubungan dengan \textit{metadata table}. & \textit{metadata table} berhasil dihapus dari basis data dan tidak muncul kembali pada saat dibuka kembali walaupun browser telah ditutup. & Ya
	\\ \hline TC-04-07 & Jika kolom tidak unik, maka akan menghasilkan error. Jika sudah unik, akan dimasukkan ke basis data. & Key hanya bisa dapat dipilih untuk maksimal satu kolom. Pada saat kolonm tersebut memiliki nilai yang tidak unik, akan muncul pesan kesalahan. Jika unik, maka kolom dapat dijadikan key. Pengaturan key juga akan muncul kembali walaupun broswer telah ditutup & Ya

	\\ \hline TC-05-01 & Tabel baru dibuat dan data masuk ke dalam basis data & Tabel berhasil dibuat dan data berhasil masuk ke dalam tabel yang sesuai. & Ya
	\\ \hline TC-05-02 & Data yang ada pada tabel tersebut berubah menjadi data baru & Data pada kedua tabel berhasil ditukar dengan baik jika belum ada spreadsheet lain yang memasukkan datanya pada tabel yang ingin ditukar. Jika sudah, maka tabel gagal ditukar. & Ya
	\\ \hline TC-05-03 & Terdapat data dari dua spreadsheet pada satu tabel. Digabungkan secara horizontal & Data dari kedua spreadsheet atau lebih berhasil digabungkan secara horizontal pada suatu tabel. & Ya
	\\ \hline TC-05-04 & Terjadi penimpaan data. & Data disimpan duluan akan tertimpa oleh data yang disimpan setelahnya sesuai dengan key data tersebut. & Ya
	\\ \hline TC-05-05 & Tidak dapat digabungkan & Tabel tidak dapat digabungkan dan muncul pesan kesalahan. & Ya
	\\ \hline TC-05-06 & Tidak dapat digabungkan & Tabel tidak dapat digabungkan dan muncul pesan kesalahan. & Ya
	\\ \hline TC-05-07 & Terdapat data dari dua spreadsheet pada satu tabel. Digabungkan secara horizontal. Jika kolom key sama, maka digabungkan vertikal. & Data dari kedua spreadsheet atau lebih berhasil digabungkan secara vertikal dimana data yang berbeda, digabungkan menurut key yang dipilih. Jika tidak ada key yang bersesuaian, maka digabungkan secara horizontal & Ya
	\\ \hline TC-05-08 & Tidak dapat digabungkan & Tabel tidak dapat digabungkan dan muncul pesan kesalahan. & Ya
	\\ \hline TC-05-09 & Data lama dihapus, lalu jumlah dan nama kolom disesuaikan dengan konfigurasi baru. & Tabel berhasil dihapus, dan datanya diganti dengan data baru dengan kolom yang baru. & Ya
	\\ \hline TC-05-10 & Data yang tadinya berhubungan dengan \textit{metadata table} yang dihapus, ikut terhapus & Data yang berhubungan dengan \textit{metadata table} tersebut berhasil dihapus dari baris pada tabel yang bersangkutan dan jika tabel kosong, maka akan dilakuakan penghapusan tabel pada basis data. & Ya
	\\ \hline TC-05-11 & Data yang berhubungan dengan \textit{metadata table} pada tabel dengan nama awal, dipindahkan ke tabel dengan nama baru. Data yang berhubungan dengan \textit{metadata table} pada tabel dengan nama awal dihapus. & Data yang berhubungan dengan \textit{metadata table} awal berhasil dihapus, serta dipindahkan ke tabel dengan nama yang baru. & Ya
	\\ \hline TC-05-12 & Data dapat masuk seluruhnya tanpa harus berurutan dan data yang masuk bersesuaian dengan kolom yang diinginkan & Data berhasil dimasukkan tanpa harus memperhitungkan urutan masukan dengan syarat data telah \textit{disjoint} satu dengan yang lainnya. Jika tidak, maka urutan masukkan data harus diperhatikan. & Ya

	\\ \hline TC-06-01 & Jika data Integer, maka data akan masuk. Jika tidak, maka akan muncul pesan kesalahan. & Data berhasil divalidasi. Jika terdapat kesalahan, muncul pesan error yang menunjukkan sel yang salah. Jika sudah benar, maka data akan masuk ke basis data. & Ya
	\\ \hline TC-06-02 & Jika data Double, maka data akan masuk. Jika tidak, maka akan muncul pesan kesalahan. & Data berhasil divalidasi. Jika terdapat kesalahan, muncul pesan error yang menunjukkan sel yang salah. Jika sudah benar, maka data akan masuk ke basis data. & Ya
	\\ \hline TC-06-03 & Jika data String, maka data akan masuk. Jika tidak, maka akan muncul pesan kesalahan. & Data berhasil divalidasi. Jika terdapat kesalahan, muncul pesan error yang menunjukkan sel yang salah. Jika sudah benar, maka data akan masuk ke basis data. & Ya
	\\ \hline TC-06-04 & Jika data Boolean, maka data akan masuk. Jika tidak, maka akan muncul pesan kesalahan. & Data berhasil divalidasi. Jika terdapat kesalahan, muncul pesan error yang menunjukkan sel yang salah. Jika sudah benar, maka data akan masuk ke basis data. & Ya
	\\ \hline TC-06-05 & Jika data berada di daftar nilai diskrit, maka data akan masuk ke basis data. Jika tidak, maka akan muncul pesan kesalahan. & Data berhasil divalidasi. Jika terdapat kesalahan, muncul pesan error yang menunjukkan sel yang salah. Jika sudah benar, maka data akan masuk ke basis data. & Ya
	\\ \hline TC-06-06 & Jika data berada pada rentang yang ditentukan, maka data akan masuk ke basis data. Jika tidak, maka akan muncul pesan kesalahan. & Data berhasil divalidasi. Jika terdapat kesalahan, muncul pesan error yang menunjukkan sel yang salah. Jika sudah benar, maka data akan masuk ke basis data. & Ya
	\\ \hline TC-06-07 & Jika data berada pada rentang yang ditentukan, maka data akan masuk ke basis data. Jika tidak, maka akan muncul pesan kesalahan. & Data berhasil divalidasi. Jika terdapat kesalahan, muncul pesan error yang menunjukkan sel yang salah. Jika sudah benar, maka data akan masuk ke basis data. & Ya
	\\ \hline TC-06-08 & Jika data berada pada rentang yang ditentukan, maka data akan masuk ke basis data. Jika tidak, maka akan muncul pesan kesalahan. & Data berhasil divalidasi. Jika terdapat kesalahan, muncul pesan error yang menunjukkan sel yang salah. Jika sudah benar, maka data akan masuk ke basis data. & Ya
	\\ \hline TC-06-09 & Jika data berada pada rentang yang ditentukan, maka data akan masuk ke basis data. Jika tidak, maka akan muncul pesan kesalahan. & Data berhasil divalidasi. Jika terdapat kesalahan, muncul pesan error yang menunjukkan sel yang salah. Jika sudah benar, maka data akan masuk ke basis data. & Ya
	\\ \hline TC-06-10 & Jika data berada pada rentang yang ditentukan, maka data akan masuk ke basis data. Jika tidak, maka akan muncul pesan kesalahan. & Data berhasil divalidasi. Jika terdapat kesalahan, muncul pesan error yang menunjukkan sel yang salah. Jika sudah benar, maka data akan masuk ke basis data. & Ya
	\\ \hline TC-06-11 & Jika data ada yang tidak berada pada rentang sel relasi yang ditentukan, maka akan muncul pesan kesalahan dan seluruh data tidak akan masuk ke basis data. Selain itu, data akan masuk ke basis data. & Data berhasil divalidasi. Jika terdapat kesalahan, muncul pesan error yang menunjukkan sel yang salah. Jika sudah benar, maka data akan masuk ke basis data. & Ya
	\\ \hline TC-06-12 & Jika data ada yang tidak berada pada tabel di kolom yang ditentukan, maka akan muncul pesan kesalahan dan seluruh data tidak akan masuk ke basis data. Selain itu, data akan masuk ke basis data. & Data berhasil divalidasi. Jika terdapat kesalahan, muncul pesan error yang menunjukkan sel yang salah. Jika sudah benar, maka data akan masuk ke basis data. & Ya
	\\ \hline TC-06-13 & Pada saat ada konflik data, maka akan keluar error. Pada saat tidak ada konflik, akan diberikan info bahwa tidak ada konflik. & Jika terdapat kemungkinan bahwa pada saat penyimpanan terjadi penimpaan terhadap data yang ada, maka akan muncul pesan error. Jika tidak, maka  akan muncul pesan bahwa tidak terjadi konflik. & Ya

	% \\ \hline TC-07-01 & - & 22 ms hingga 26 ms tidak bergantung ukuran spreadsheet & Ya
	% \\ \hline TC-07-02 & - & 1054 ms pada spreadsheet dengan range terisi A1:K16 (16 baris) \newline 5736 ms pada spreadsheet dengan range terisi A1:F112 (112 baris) & Ya
	% \\ \hline TC-07-03 & - & 30ms pada spreadsheet dengan 16 baris data \newline 2000ms pada spreadsheet dengan 112 baris data  --- 580 barus -> 20000ms & Ya
	% \\ \hline TC-07-04 & - & 150ms pada spreadsheet dengan 16 baris data \newline 200ms pada spreadsheet dengan 112 baris data & Ya
	\\ \hline
\end{longtable}
\end{small}