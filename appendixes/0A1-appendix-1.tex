\chapter{Lampiran A. Detail Implementasi}

\begin{small}
\begin{longtable}{ | p{2cm} | p{10cm} | }
    \caption{Fungsi pada Kelas \texttt{player}}
    \label{FungsiModulPlayer}\\ \hline
    \centering\bfseries{Fungsi} & \centering\bfseries{Deskripsi} \tabularnewline \hline
    \endfirsthead
    \hline
    \centering\bfseries{Fungsi} & \centering\bfseries{Deskripsi} \tabularnewline \hline
    \endhead
    refreshView & Melakukan pembaharuan tampilan sehingga menunjukkan tabel terbaru.\\ \hline
    getDBSetting & Mengambil pengaturan basis data yang telah diberikan pengguna.\\ \hline
    saveState & Menyimpan pengaturan yang telah dilakukan ke basis data.\\ \hline
    loadState & Mengambil pengaturan yang pernah disimpan pada basis data.\\ \hline
    saveConfig & Melakukan penyimpanan konfigurasi tabel yang dilakukan oleh pengguna. \\ \hline
    deleteTable & Menghapus \textit{metadata table} yang dipilih.\\ \hline
    addManual & Menambahkan \textit{metadata table} baru sesuai dengan masukan pengguna.\\ \hline
    connect & Melakukan koneksi ke basis data yang dipilih. \\ \hline
    save & Melakukan pemanggilan terhadap modul \texttt{table} dan melakukan penyimpanan ke basis data. \\ \hline
    scan & Melakukan identifikasi tabel melalui pemanggilan modul \texttt{framefinder} yang selanjutnya akan menampilkan hasil identifikasi dan kolom perubahan konfigurasi yang dapat diisi pengguna. \\ \hline
\end{longtable}
\end{small}

\begin{small}
\begin{longtable}{ | p{2cm} | p{10cm} | }
    \caption{Fungsi pada Kelas \texttt{mysql}}
    \label{FungsiModulDB}\\ \hline
    \centering\bfseries{Fungsi} & \centering\bfseries{Deskripsi} \tabularnewline \hline
    \endfirsthead
    \hline
    \centering\bfseries{Fungsi} & \centering\bfseries{Deskripsi} \tabularnewline \hline
    \endhead
    createTable & Fungsi yang digunakan untuk membuat Table tempat pengisian data.\\ \hline
    isTableExists & Melakukan pengecekan ada atau tidaknya tabel tersebut pada basis data.\\ \hline
    getColumns & Mendapatkan kolom-kolom yang ada pada suatu tabel.\\ \hline
    selectData & Mendapatkan data sesuai dengan syarat yang diminta.\\ \hline
    insertData & Memasukkan data ke dalam tabel yang dipilih.\\ \hline
    deleteData & Menghapus data dari tabel.\\ \hline
    updateData & Melakukan pembaharuan data dari tabel.\\ \hline
    dropTable & Menghapus tabel yang dipilih.\\ \hline
\end{longtable}
\end{small}

\begin{small}
\begin{longtable}{ | p{3cm} | p{10cm} | }
    \caption{Kelas pada Modul \texttt{FrameFinder}}
    \label{KelasModulFF}\\ \hline
    \centering\bfseries{Nama Kelas} & \centering\bfseries{Deskripsi} \tabularnewline \hline
    \endfirsthead
    \hline
    \centering\bfseries{Nama Kelas} & \centering\bfseries{Deskripsi} \tabularnewline \hline
    \endhead
    LoadSheet & Kelas ini berfungsi sebagai kelas yang melakukan pengambilan data dan konversi sel dan \textit{sheet} pada \textit{spreadsheet} ke dalam bentuk kelas-kelas yang ada pada modul ini.\\ \hline
    MySheet & Merupakan kelas bentukan yang merepresentasikan \textit{sheet} pada \textit{spreadsheet} yang dipilih.\\ \hline
    MyCell & Merupakan kelas untuk merepresentasikan \textit{properties} yang ada pada sel pada \textit{sheet} yang dipilih.\\ \hline
    FeatureSheetRow & Melakukan ekstraksi fitur-fitur yang terdapat pada suatu \textit{sheet} pada \textit{spreadsheet}.\\ \hline
    PredictSheetRows & Kelas ini digunakan untuk menghasilkan file dalam format yang dapat dibaca oleh algoritma Conditional Random Field (CRF) dari fitur-fitur yang telah diekstraksi pada \textit{sheet}.\\ \hline
\end{longtable}
\end{small}

\begin{small}
\begin{longtable}{ | p{4cm} | p{9cm} | }
    \caption{Fungsi pada Kelas LoadSheet}
    \label{FungsiLoadSheet}\\ \hline
    \centering\bfseries{Nama Fungsi} & \centering\bfseries{Deskripsi} \tabularnewline \hline
    \endfirsthead
    \hline
    \centering\bfseries{Nama Fungsi} & \centering\bfseries{Deskripsi} \tabularnewline \hline
    \endhead
    loadSheetDict & Fungsi ini merupakan fungsi utama yang bertugas untuk membuat representasi \textit{spreadsheet} yang diterima ke dalam kelas MySheet.\\ \hline
    getValueType & Untuk mendapatkan tipe representasi data yang diberikan oleh \textit{spreadsheet}. Contoh: tanggal, nominal uang, desimal, dan lain-lain.\\ \hline
    getDataType & Untuk mendapatkan tipe data primitif pada suatu sel.\\ \hline
    featureIndentation & Digunakan untuk mengecek keberadaan \textit{property} \textit{indentation} pada sel.\\ \hline
    featureAlignStyle & Digunakan untuk mengecek keberadaan \textit{property} \textit{align} pada sel\\ \hline
    featureFontBold & Digunakan untuk mengecek keberadaan \textit{property} \textit{bold} pada sel\\ \hline
    featureFontHeight & Digunakan untuk mengecek keberadaan \textit{property} \textit{height} pada sel\\ \hline
    featureFontUnderline & Digunakan untuk mengecek keberadaan \textit{property} \textit{underline} pada sel\\ \hline
    featureFontItalic & Digunakan untuk mengecek keberadaan \textit{property} \textit{italic} pada sel\\ \hline
    featureFontBgcolor & Digunakan untuk mengecek keberadaan \textit{property} \textit{background color} pada sel\\ \hline
    featureBorderStyle & Digunakan untuk mengecek keberadaan \textit{property} \textit{border} pada sel.\\ \hline
\end{longtable}
\end{small}

\begin{small}
\begin{longtable}{ | p{3cm} | p{8cm} | }
    \caption{Atribut pada Kelas MySheet}
    \label{AtributMySheet}\\ \hline
    \centering\bfseries{Nama Atribut} & \centering\bfseries{Deskripsi} \tabularnewline \hline
    \endfirsthead
    \hline
    \centering\bfseries{Nama Atribut} & \centering\bfseries{Deskripsi} \tabularnewline \hline
    \endhead
    sheetdict & Merupakan representasi kumpulan sel-sel pada suatu \textit{sheet}. Tiap sel direpresentasikan dalam bentuk kelas MyCell.\\ \hline
    mergerowdict & Merupakan kumpulan sel-sel yang digabungkan.\\ \hline
    maxcolnum & Nilai kolom terbesar pada sel.\\ \hline
    maxrownum & Nilai baris terbesar pada sel.\\ \hline
\end{longtable}
\end{small}

\begin{small}
\begin{longtable}{ | p{4cm} | p{9cm} | }
    \caption{Fungsi pada Kelas MySheet}
    \label{FungsiMySheet}\\ \hline
    \centering\bfseries{Nama Fungsi} & \centering\bfseries{Deskripsi} \tabularnewline \hline
    \endfirsthead
    \hline
    \centering\bfseries{Nama Fungsi} & \centering\bfseries{Deskripsi} \tabularnewline \hline
    \endhead
    getCellsArray & Digunakan untuk mendapatkan seluruh refresentasi sel pada kelas ini dalam bentuk \textit{array}.\\ \hline
    addMergeCell & Digunakan pada saat terdapat sel yang digabungkan. Sel tersebut akan dimasukkan ke dalam daftar \textit{merged cells}.\\ \hline
    insertCell & Menambahkan sel ke dalam kelas ini. Sel yang ditambahkan akan direpresentasikan dalam bentuk kelas MyCell.\\ \hline
\end{longtable}
\end{small}

\begin{small}
\begin{longtable}{ | p{3cm} | p{8cm} | }
    \caption{Atribut pada Kelas MyCell}
    \label{AtributMyCell}\\ \hline
    \centering\bfseries{Nama Atribut} & \centering\bfseries{Deskripsi} \tabularnewline \hline
    \endfirsthead
    \hline
    \centering\bfseries{Nama Atribut} & \centering\bfseries{Deskripsi} \tabularnewline \hline
    \endhead
    x & Merupakan letak sel pada koordinat X.\\ \hline
    y & Merupakan letak sel pada koordinat Y. \\ \hline
    w & Nilai lebar sel.\\ \hline
    h & Nilai tinggi sel.\\ \hline
    cstr & Isi sel dalam bentuk \textit{string}.\\ \hline
    mtype & Tipe konten yang ada di dalam sel.\\ \hline
    indents & Nilai indentasi jika terdapat indentasi pada konten.\\ \hline
    centeralign & Bernilai \textit{true} atau \textit{false} bergantung pada \textit{align} sel merupakan rata tengah atau tidak.\\ \hline
    leftalign & Bernilai \textit{true} atau \textit{false} bergantung pada \textit{align} sel merupakan rata kiri atau tidak.\\ \hline
    rightalign & Bernilai \textit{true} atau \textit{false} bergantung pada \textit{align} sel merupakan rata kanan atau tidak.\\ \hline
    bottomborder & Bernilai \textit{true} atau \textit{false} bergantung pada \textit{property} \textit{bottom border} ada atau tidak.\\ \hline
    upperborder & Bernilai \textit{true} atau \textit{false} bergantung pada \textit{property} \textit{upper border} ada atau tidak.\\ \hline
    leftborder & Bernilai \textit{true} atau \textit{false} bergantung pada \textit{property} \textit{left border} ada atau tidak.\\ \hline
    rightborder & Bernilai \textit{true} atau \textit{false} bergantung pada \textit{property} \textit{right border} ada atau tidak.\\ \hline
    bold & Bernilai \textit{true} atau \textit{false} bergantung pada \textit{property} \textit{bold} ada atau tidak.\\ \hline
    italic & Bernilai \textit{true} atau \textit{false} bergantung pada \textit{property} \textit{italic} ada atau tidak.\\ \hline
    underline & Bernilai \textit{true} atau \textit{false} bergantung pada \textit{property} \textit{underline} ada atau tidak.\\ \hline
\end{longtable}
\end{small}

\begin{small}
\begin{longtable}{ | p{10cm} | }
    \caption{Fitur yang Diambil dari Sel}
    \label{FiturEkstraksi}\\ \hline
    \centering\bfseries{Fitur} \tabularnewline \hline
    \endfirsthead
    \hline
    \centering\bfseries{Fitur} \tabularnewline \hline
    \endhead
    Baris memiliki sel yang digabung \\ \hline
    Sel pada baris mencapai kolom paling kanan \\ \hline
    Sel pada baris mencapai kolom paling kiri \\ \hline
    Baris hanya memiliki 1 kolom \\ \hline
    Baris memiliki sel rata tengah \\ \hline
    Baris memiliki sel rata kiri \\ \hline
    Baris memiliki sel yang ditebalkan (\textit{bold}) \\ \hline
    Baris memiliki sel berindentasi \\ \hline
    Baris memiliki sel berisi kata `Table` \\ \hline
    Baris memiliki sel berisi kata berawalan tanda baca \\ \hline
    Baris memiliki sel dengan presentase angka tinggi \\ \hline
    Baris memiliki sel berisi huruf besar seluruhnya \\ \hline
    Baris memiliki sel berisi kata dengan awal huruf besar \\ \hline
    Baris memiliki sel berisi kata dengan awal huruf kecil \\ \hline
    Baris memiliki persentase sel memiliki isi tinggi \\ \hline
    Baris memiliki persentase sel memiliki isi berupa kata tinggi \\ \hline
    Baris memiliki sel berisi karakter spesial \\ \hline
    Baris memiliki sel berisi karakter titik koma \\ \hline
    Baris memiliki jumlah sel berisi tahun tinggi \\ \hline
    Baris memiliki persentase sel berisi tahun tinggi \\ \hline
    Baris memiliki jumlah sel berisi kata dengan huruf yang banyak tinggi \\ \hline
\end{longtable}
\end{small}

\begin{small}
\begin{longtable}{ | p{3cm} | p{8cm} | }
    \caption{Atribut pada Kelas Table}
    \label{AtributTable}\\ \hline
    \centering\bfseries{Nama Atribut} & \centering\bfseries{Deskripsi} \tabularnewline \hline
    \endfirsthead
    \hline
    \centering\bfseries{Nama Atribut} & \centering\bfseries{Deskripsi} \tabularnewline \hline
    \endhead
    sheet & Berisikan data untuk masing-masing sel pada suatu \textit{sheet}.\\ \hline
    rows & Representasi tabel yang berisikan; nama \textit{header}, kolom data, dan aturan-aturan validasi.\\ \hline
    range & \textit{Range} sel data pada tabel.\\ \hline
    title & Kumpulan baris yang diidentifikasi sebagai \textit{title} oleh pengenalan otomatis.\\ \hline
    footnote & Kumpulan baris yang diidentifikasi sebagai \textit{footnote} oleh pengenalan otomatis.\\ \hline
    header & Kumpulan baris yang diidentifikasi sebagai \textit{header} oleh pengenalan otomatis.\\ \hline
    data & Kumpulan baris yang diidentifikasi sebagai \textit{data} oleh pengenalan otomatis.\\ \hline
    name & Nama tabel yang akan dijadikan nama tabel pada basis data.\\ \hline
\end{longtable}
\end{small}

\begin{small}
\begin{longtable}{ | p{4cm} | p{9cm} | }
    \caption{Fungsi pada Kelas LoadSheet}
    \label{FungsiLoadSheet}\\ \hline
    \centering\bfseries{Nama Fungsi} & \centering\bfseries{Deskripsi} \tabularnewline \hline
    \endfirsthead
    \hline
    \centering\bfseries{Nama Fungsi} & \centering\bfseries{Deskripsi} \tabularnewline \hline
    \endhead
    ParseData & Melakukan \textit{parse} terhadap data dari \textit{spreadsheet} menjadi bentuk objek tabel. \\ \hline
	TupleSerializeWithChecker & Digunakan untuk mengubah struktur tabel menjadi tabel relasional dan melakukan validasi data masukan sesuai dengan aturan yang diminta pengguna.\\ \hline
	Serialize & Mengubah objek tabel menjadi JSON.\\ \hline
	Deserialize & Mengubah JSON menjadi objek tabel.\\ \hline
	GetHTMLForm & Menghasilkan bentuk HTML dari objek tabel agar dapat ditampilkan pada antarmuka.\\ \hline
\end{longtable}
\end{small}