%--------------------------------------------------------------------%
% REV01: Fri 23 Jul 2021 20:08:29 WIB (RMS)
% Berkas utama templat LaTeX.
% author Petra Barus, Peb Ruswono Aryan
%--------------------------------------------------------------------%
% Berkas ini berisi struktur utama dokumen LaTeX yang akan dibuat.
%--------------------------------------------------------------------%
\documentclass[12pt, a4paper, onecolumn, oneside, final]{report}

%-------------------------------------------------------------------%
%
% Konfigurasi dokumen LaTeX untuk laporan tesis IF ITB
%
% @author Petra Novandi
%
%-------------------------------------------------------------------%
%
% Berkas asli berasal dari Steven Lolong
%
%-------------------------------------------------------------------%

% Ukuran kertas
\special{papersize=210mm,297mm}

% Setting margin
\usepackage[top=3cm,bottom=2.5cm,left=4cm,right=2.5cm]{geometry}

\usepackage{mathptmx}

% Judul bahasa Indonesia
\usepackage[bahasa]{babel}

% Format citation
%\usepackage[backend=bibtex,citestyle=authoryear]{biblatex}
\usepackage{natbib}

\usepackage[utf8]{inputenc}
\usepackage{microtype}
\usepackage{makecell}
\usepackage{graphicx}
\usepackage{csquotes}
\usepackage{tabularx}
\usepackage{listings}
\usepackage{tabto}
\usepackage{comment}
\usepackage{amsmath}
\usepackage[labelfont=bf]{caption}
\usepackage{enumitem}
\usepackage{tocbibind}
\usepackage{tocloft}
\usepackage{float}
\usepackage{indentfirst}
\usepackage[auto]{chappg}
\usepackage{titling}
\usepackage{blindtext}
\usepackage{sectsty}
\usepackage{chngcntr}
\usepackage{etoolbox}
\usepackage{hyperref}  
\usepackage{titlesec}
\usepackage{parskip}
\usepackage[htt]{hyphenat}
\usepackage{longtable}
\usepackage{pgfgantt}
\usepackage{booktabs}
\usepackage{pifont}
\usepackage{siunitx}
\usepackage[section]{placeins}
\usepackage[ruled]{algorithm2e}
%\RestyleAlgo{ruled}
\usepackage{setspace}
 
% \usepackage{ragged2e}
% \usepackage{libertine}
% \usepackage{array}
% \usepackage{textcase}
% \usepackage{setspace}
% \usepackage{afterpage}

% Line satu setengah spasi
\renewcommand{\baselinestretch}{1.5}
% Line untuk pseudocode
\AtBeginEnvironment{algorithm}{\setstretch{1.35}}

% Setting judul
\titlespacing*{\chapter}{0pt}{-50pt}{10pt}
\chapterfont{\centering \Large}
\titleformat{\chapter}[display]
  {\Large\centering\bfseries}
  {\chaptertitlename\ \thechapter}{0pt}
    {\Large\bfseries\uppercase}

% Setting nomor pada subbsubsubbab
\setcounter{secnumdepth}{3}
\setcounter{tocdepth}{4}

\makeatletter
\setlength{\@fptop}{0pt}
\setlength{\@fpbot}{0pt plus 1fil}
\makeatother

% Counter untuk figure dan table.
\counterwithin{figure}{chapter}
\counterwithin{table}{chapter}

% Counter untuk penomoran halaman lanjut
\newcounter{savepage}

% Kode lebih kecil
% \let\OldTexttt\texttt
% \renewcommand{\texttt}[1]{\OldTexttt{\footnotesize{#1}}}

% Pengaturan Caption
\captionsetup{labelsep=space}

% Pengaturan spasi untuk justify
\pretolerance=10000
\tolerance=2000 
\emergencystretch=10pt
%\tolerance=1
%\emergencystretch=10pt
%\hyphenpenalty=10000
%\exhyphenpenalty=1000

% Pengaturan untuk Daftar Rumus
\newcommand{\listequationsname}{Daftar Rumus}
\newlistof{myequations}{equ}{\listequationsname}
\newcommand{\myequations}[1]{%
\addcontentsline{equ}{myequations}{\protect\numberline{\theequation}#1}\par}
\setlength{\cftmyequationsnumwidth}{2.5em}

%\newcommand{\listalgorithmname}{Daftar Algoritme}
%\newlistof{algorithm}{alg}{\listalgorithmname}
%\newcommand{\algorithm}[1]{%
%\refstepcounter{algorithm}
%\par\noindent\centering\normalsize\textbf{Algoritme {\thechapter.\thealgorithm\ }\normalfont #1}
%\addcontentsline{alg}{algorithm}{\protect\numberline{\thechapter.\thealgorithm}#1}\par}
%\setlength{\cftalgorithmnumwidth}{2.5em}

% Pengaturan untuk title Daftar Isi, Tabel, dan Gambar
\renewcommand{\cfttoctitlefont}{\hspace*{\fill}\Large\bfseries\MakeUppercase}
\renewcommand{\cftaftertoctitle}{\hspace*{\fill}}
\renewcommand{\cftlottitlefont}{\hspace*{\fill}\Large\bfseries\MakeUppercase}
\renewcommand{\cftafterlottitle}{\hspace*{\fill}}
\renewcommand{\cftloftitlefont}{\hspace*{\fill}\Large\bfseries\MakeUppercase}
\renewcommand{\cftafterloftitle}{\hspace*{\fill}} 
\renewcommand{\cftequtitlefont}{\hspace*{\fill}\Large\bfseries\MakeUppercase}
\renewcommand{\cftafterequtitle}{\hspace*{\fill}} 
%\renewcommand{\cftalgtitlefont}{\hspace*{\fill}\Large\bfseries\MakeUppercase}
%\renewcommand{\cftafteralgtitle}{\hspace*{\fill}} 

\renewcommand{\cftchappresnum}{Bab }
\renewcommand{\cftchapaftersnum}{}
\renewcommand{\cftchapnumwidth}{3.7em}

\setlength{\cftbeforetoctitleskip}{-4em}
\setlength{\cftbeforeloftitleskip}{-4em}
\setlength{\cftbeforelottitleskip}{-4em}
\setlength{\cftbeforeequtitleskip}{-4em}
%\setlength{\cftbeforealgtitleskip}{-4em}

\newcolumntype{Y}{>{\centering\arraybackslash}X}

% Listing Code thingy
\lstset{frame=single,
  columns=fullflexible,
  basicstyle={\small\ttfamily},
  breaklines=true,
  breakatwhitespace=false,
  postbreak=\mbox{$\hookrightarrow$\space},
  tabsize=3
}
\renewcommand{\lstlistingname}{Kode}
\renewcommand{\lstlistlistingname}{Daftar \lstlistingname}
% \newcommand{\listingsfont}{\ttfamily}
% https://en.wikibooks.org/wiki/LaTeX/Source_Code_Listings
\renewcommand{\cftchapleader}{\cftdotfill{\cftdotsep}}

\cftsetpnumwidth{2em}


% \DefineBibliographyStrings{english}{%
%     urlseen = {Waktu akses},
%     url = {URL:},
%     and = {dan},
%     andothers = {dkk\adddot},
%     in = {dalam}
% }

% Add comma between Author and Year
%\renewcommand*{\nameyeardelim}{\addcomma\space}

% \titleformat{\paragraph}
% {\normalfont\normalsize\bfseries}{\theparagraph}{1em}{}
% \titlespacing*{\paragraph}
% {0pt}{3.25ex plus 1ex minus .2ex}{1.5ex plus .2ex}

% \titlespacing*{\section} {0pt}{2.5ex plus 1ex minus .2ex}{1.3ex plus .2ex}
% \titlespacing*{\subsection} {0pt}{2.25ex plus 1ex minus .2ex}{0.5ex plus .2ex}

% For Gantt Chart
% \newcounter{myWeekNum}
% \stepcounter{myWeekNum}
% 
% \newcommand{\myWeek}{\themyWeekNum
%     \stepcounter{myWeekNum}
%     \ifnum\themyWeekNum=53
%          \setcounter{myWeekNum}{1}
%     \else\fi
% }
% 
% \newcolumntype{L}[1]{>{\raggedright\let\newline\\\arraybackslash\hspace{0pt}}m{#1}}
% \newcolumntype{C}[1]{>{\centering\let\newline\\\arraybackslash\hspace{0pt}}m{#1}}
% \newcolumntype{R}[1]{>{\raggedleft\let\newline\\\arraybackslash\hspace{0pt}}m{#1}}
%
% \makeatletter
% \def\@chapter[#1]#2{\ifnum \c@secnumdepth >\m@ne
%                          \refstepcounter{chapter}%
%                          \typeout{\@chapapp\space\thechapter.}%
%                          \addcontentsline{toc}{chapter}%
%                                    {\protect{BAB \numberline{\thechapter}\texorpdfstring{\uppercase{#1}}{#1}}}%
%                     \else
%                       \addcontentsline{toc}{chapter}{\texorpdfstring{\uppercase{#1}}{#1}}%
%                     \fi
%                     \chaptermark{#1}%
%                     \addtocontents{lof}{\protect\addvspace{10\p@}}%
%                     \addtocontents{lot}{\protect\addvspace{10\p@}}%
%                     \if@twocolumn
%                       \@topnewpage[\@makechapterhead{#2}]%
%                     \else
%                       \@makechapterhead{#2}%
%                       \@afterheading
%                     \fi}
% \makeatother


\makeatletter

\makeatother
%\bibliography{references}
\begin{document}

%Basic configuration
\title{Pengembangan kakas untuk melakukan \textit{Performance Regression Analysis} menggunakan \textit{Distributed Tracing} pada Aplikasi berbasis Microservice}
\date{25 April}
\newcommand{\yearsidang}{2022}
\newcommand{\jenislaporan}{Draf Bab 5 Tugas Akhir}
\author{Rafi Abbel Mohammad}
\newcommand{\nim}{18218027}

\pagenumbering{roman}
\setcounter{page}{0}

\input{frontpages/000-cover}
\clearpage
\pagestyle{empty}

\begin{center}
    \smallskip

    {\Large \bfseries Lembar Pengesahan}

    \MakeUppercase{\normalsize \bfseries \thetitle}
    \vfill

    \normalsize \jenislaporan \\
    Program Studi: Sarjana Sistem dan Teknologi Informasi \\
    Sekolah Teknik Elektro dan Informatika \\
    Institut Teknologi Bandung \\
    \vfill

    \normalsize oleh :

    \normalsize \theauthor \\
    \normalsize NIM: \nim

    \vfill
    \normalsize \normalfont
    Telah disetujui dan disahkan sebagai Laporan \jenislaporan \\
    di Bandung, pada tanggal \thedate{} \yearsidang{}.

    \vfill
    \setlength{\tabcolsep}{12pt}
    \begin{tabularx}{\textwidth}{c@{\hskip 0.2\textwidth}cc@{\hskip 0.3\textwidth}}
         & {\bfseries Pembimbing}                                  & \\
         &                                                         & \\
         &                                                         & \\
         &                                                         & \\
         &                                                         & \\
         & \underline{\pembimbing} & \\
         & \nipPembimbing                            &
    \end{tabularx}

\end{center}
\clearpage


\pagestyle{plain}


\titleformat*{\section}{\centering\bfseries\Large\MakeUpperCase}

\clearpage
\tableofcontents

\clearpage
{%
    \let\oldnumberline\numberline%
    \renewcommand{\numberline}{\figurename~\oldnumberline}%
    \listoffigures%
}

\clearpage
{%
    \let\oldnumberline\numberline%
    \renewcommand{\numberline}{\tablename~\oldnumberline}%
    \listoftables%
}

\clearpage
{%
    \let\oldnumberline\numberline%
    \renewcommand{\numberline}{\lstlistingname~\oldnumberline}%
    \lstlistoflistings%
}

\newpage
\setcounter{savepage}{\arabic{page}}

\titleformat*{\section}{\bfseries\large}

%----------------------------------------------------------------%
% Konfigurasi Bab
%----------------------------------------------------------------%
\renewcommand{\chaptername}{BAB}
\renewcommand{\thechapter}{\Roman{chapter}}
\pagenumbering{bychapter}
\setlength{\parindent}{1cm}
%----------------------------------------------------------------%

%----------------------------------------------------------------%
% Dafter Bab
% Untuk menambahkan daftar bab, buat berkas bab misalnya `chapter-6` di direktori `chapters`, dan masukkan ke sini.
%----------------------------------------------------------------%
\input{chapters/01-chapter-1-empty}
\input{chapters/02-chapter-2-empty}
\input{chapters/03-chapter-3-empty}
\input{chapters/04-chapter-4-empty}
\chapter{Kesimpulan dan Saran}
Bab ini berisi hal-hal yang dapat disimpulkan dari pelaksanaan Tugas Akhir ini. Bab ini juga mencakup saran untuk pengembangan Tugas Akhir ini di masa mendatang.

\section{Kesimpulan}
Berdasarkan hasil pengembangan dan pengujian sistem \textit{Performance Regression Analysis} (PRA) yang telah dilakukan. Berikut ini adalah kesimpulan yang diperoleh.
\begin{enumerate}
	\item Telah berhasil dilakukan pendeteksian regresi pada kinerja aplikasi berbasis Microservice.
	\item Telah berhasil dilakukan analisis untuk menentukan akar penyebab regresi pada aplikasi berbasis Microservice.
	\item Hasil pengujian menunjukkan sistem PRA telah berhasil menguji 10 dari 11 kasus regresi atau sekitar 90\% kasus. Satu-satunya kasus regresi gagal terdeteksi adalah kasus dengan penambahan \textit{latency} sebesar 100ms. Pada kasus ini regresi tidak terdeteksi karena algoritma tidak menganggap penambahan \textit{latency} sebesar 100ms menjadikan sampel data menjadi berbeda distribusinya dengan sampel data \textit{baseline} sehingga sistem tidak dapat mendeteksi terjadinya regresi.
	\item Pendeteksian regresi dapat dilakukan menggunakan tes statistik Kolmogorov-Smirnov dengan membandingkan data \textit{latency} kasus regresi dengan data \textit{latency} \textit{baseline} yang merepresentasikan kinerja aplikasi Microservice pada keadaan normal.
	\item Analisis penyebab regresi belum dapat ditentukan dari hasil pendeteksian regresi oleh algoritma tes Kolmogorov-Smirnov namun dapat ditentukan dari hasil analisis Critical Path yang membandingkan selisih data \textit{latency} operasi pada kasus regresi dan \textit{baseline}.
	\item Menurut pengujian, komponen PRA Engine mengakibatkan \textit{overhead} CPU sebanyak 0,78 \% dan Memory sebanyak 0,67 \% kepada \textit{cluster} Kubernetes. \textit{Overhead} yang diakibatkan oleh kakas termasuk rendah sehingga penggunaan kakas PRA Engine tidak akan berdampak pada kinerja \textit{cluster} Kubernetes secara keseluruhan.\textbf{}
\end{enumerate}

\section{Saran}
Saran yang dapat diberikan untuk pengembangan di masa mendatang adalah sebagai berikut:
\begin{enumerate}
	\item Pada pengembangan kakas selanjutnya dapat ditambahkan penanganan untuk pengujian pada lingkungan aplikasi selain Kubernetes
%	\item Menggunakan solusi \textit{distributed tracing} selain Zipkin untuk melakukan analisis serupa sebab operasi \textit{query} pada API Zipkin menjadi \textit{bottleneck} yang menjadikan pemanggilan API \textit{engine} lambat
	\item Pengembangan kakas menggunakan bahasa pemrograman yang lebih cepat dibandingkan Python seperti Go atau Rust
\end{enumerate}
%----------------------------------------------------------------%

% Daftar pustaka
% Bibliography to Daftar Pustaka
\renewcommand{\bibname}{Daftar Pustaka}
\clearpage
\phantomsection
\pagenumbering{roman}
\setcounter{page}{\thesavepage}
\bibliography{references}
\bibliographystyle{apalike}


\end{document}
