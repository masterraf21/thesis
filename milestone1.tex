%--------------------------------------------------------------------%
% REV01: Fri 23 Jul 2021 20:08:29 WIB (RMS)
% Berkas utama templat LaTeX.
% author Petra Barus, Peb Ruswono Aryan
%--------------------------------------------------------------------%
% Berkas ini berisi struktur utama dokumen LaTeX yang akan dibuat.
%--------------------------------------------------------------------%
\documentclass[12pt, a4paper, onecolumn, oneside, final]{book}

%-------------------------------------------------------------------%
%
% Konfigurasi dokumen LaTeX untuk laporan tesis IF ITB
%
% @author Petra Novandi
%
%-------------------------------------------------------------------%
%
% Berkas asli berasal dari Steven Lolong
%
%-------------------------------------------------------------------%

% Ukuran kertas
\special{papersize=210mm,297mm}

% Setting margin
\usepackage[top=3cm,bottom=2.5cm,left=4cm,right=2.5cm]{geometry}

\usepackage{mathptmx}

% Judul bahasa Indonesia
\usepackage[bahasa]{babel}

% Format citation
%\usepackage[backend=bibtex,citestyle=authoryear]{biblatex}
\usepackage{natbib}

\usepackage[utf8]{inputenc}
\usepackage{microtype}
\usepackage{makecell}
\usepackage{graphicx}
\usepackage{csquotes}
\usepackage{tabularx}
\usepackage{listings}
\usepackage{tabto}
\usepackage{comment}
\usepackage{amsmath}
\usepackage[labelfont=bf]{caption}
\usepackage{enumitem}
\usepackage{tocbibind}
\usepackage{tocloft}
\usepackage{float}
\usepackage{indentfirst}
\usepackage[auto]{chappg}
\usepackage{titling}
\usepackage{blindtext}
\usepackage{sectsty}
\usepackage{chngcntr}
\usepackage{etoolbox}
\usepackage{hyperref}  
\usepackage{titlesec}
\usepackage{parskip}
\usepackage[htt]{hyphenat}
\usepackage{longtable}
\usepackage{pgfgantt}
\usepackage{booktabs}
\usepackage{pifont}
\usepackage{siunitx}
\usepackage[section]{placeins}
\usepackage[ruled]{algorithm2e}
%\RestyleAlgo{ruled}
\usepackage{setspace}
 
% \usepackage{ragged2e}
% \usepackage{libertine}
% \usepackage{array}
% \usepackage{textcase}
% \usepackage{setspace}
% \usepackage{afterpage}

% Line satu setengah spasi
\renewcommand{\baselinestretch}{1.5}
% Line untuk pseudocode
\AtBeginEnvironment{algorithm}{\setstretch{1.35}}

% Setting judul
\titlespacing*{\chapter}{0pt}{-50pt}{10pt}
\chapterfont{\centering \Large}
\titleformat{\chapter}[display]
  {\Large\centering\bfseries}
  {\chaptertitlename\ \thechapter}{0pt}
    {\Large\bfseries\uppercase}

% Setting nomor pada subbsubsubbab
\setcounter{secnumdepth}{3}
\setcounter{tocdepth}{4}

\makeatletter
\setlength{\@fptop}{0pt}
\setlength{\@fpbot}{0pt plus 1fil}
\makeatother

% Counter untuk figure dan table.
\counterwithin{figure}{chapter}
\counterwithin{table}{chapter}

% Counter untuk penomoran halaman lanjut
\newcounter{savepage}

% Kode lebih kecil
% \let\OldTexttt\texttt
% \renewcommand{\texttt}[1]{\OldTexttt{\footnotesize{#1}}}

% Pengaturan Caption
\captionsetup{labelsep=space}

% Pengaturan spasi untuk justify
\pretolerance=10000
\tolerance=2000 
\emergencystretch=10pt
%\tolerance=1
%\emergencystretch=10pt
%\hyphenpenalty=10000
%\exhyphenpenalty=1000

% Pengaturan untuk Daftar Rumus
\newcommand{\listequationsname}{Daftar Rumus}
\newlistof{myequations}{equ}{\listequationsname}
\newcommand{\myequations}[1]{%
\addcontentsline{equ}{myequations}{\protect\numberline{\theequation}#1}\par}
\setlength{\cftmyequationsnumwidth}{2.5em}

%\newcommand{\listalgorithmname}{Daftar Algoritme}
%\newlistof{algorithm}{alg}{\listalgorithmname}
%\newcommand{\algorithm}[1]{%
%\refstepcounter{algorithm}
%\par\noindent\centering\normalsize\textbf{Algoritme {\thechapter.\thealgorithm\ }\normalfont #1}
%\addcontentsline{alg}{algorithm}{\protect\numberline{\thechapter.\thealgorithm}#1}\par}
%\setlength{\cftalgorithmnumwidth}{2.5em}

% Pengaturan untuk title Daftar Isi, Tabel, dan Gambar
\renewcommand{\cfttoctitlefont}{\hspace*{\fill}\Large\bfseries\MakeUppercase}
\renewcommand{\cftaftertoctitle}{\hspace*{\fill}}
\renewcommand{\cftlottitlefont}{\hspace*{\fill}\Large\bfseries\MakeUppercase}
\renewcommand{\cftafterlottitle}{\hspace*{\fill}}
\renewcommand{\cftloftitlefont}{\hspace*{\fill}\Large\bfseries\MakeUppercase}
\renewcommand{\cftafterloftitle}{\hspace*{\fill}} 
\renewcommand{\cftequtitlefont}{\hspace*{\fill}\Large\bfseries\MakeUppercase}
\renewcommand{\cftafterequtitle}{\hspace*{\fill}} 
%\renewcommand{\cftalgtitlefont}{\hspace*{\fill}\Large\bfseries\MakeUppercase}
%\renewcommand{\cftafteralgtitle}{\hspace*{\fill}} 

\renewcommand{\cftchappresnum}{Bab }
\renewcommand{\cftchapaftersnum}{}
\renewcommand{\cftchapnumwidth}{3.7em}

\setlength{\cftbeforetoctitleskip}{-4em}
\setlength{\cftbeforeloftitleskip}{-4em}
\setlength{\cftbeforelottitleskip}{-4em}
\setlength{\cftbeforeequtitleskip}{-4em}
%\setlength{\cftbeforealgtitleskip}{-4em}

\newcolumntype{Y}{>{\centering\arraybackslash}X}

% Listing Code thingy
\lstset{frame=single,
  columns=fullflexible,
  basicstyle={\small\ttfamily},
  breaklines=true,
  breakatwhitespace=false,
  postbreak=\mbox{$\hookrightarrow$\space},
  tabsize=3
}
\renewcommand{\lstlistingname}{Kode}
\renewcommand{\lstlistlistingname}{Daftar \lstlistingname}
% \newcommand{\listingsfont}{\ttfamily}
% https://en.wikibooks.org/wiki/LaTeX/Source_Code_Listings
\renewcommand{\cftchapleader}{\cftdotfill{\cftdotsep}}

\cftsetpnumwidth{2em}


% \DefineBibliographyStrings{english}{%
%     urlseen = {Waktu akses},
%     url = {URL:},
%     and = {dan},
%     andothers = {dkk\adddot},
%     in = {dalam}
% }

% Add comma between Author and Year
%\renewcommand*{\nameyeardelim}{\addcomma\space}

% \titleformat{\paragraph}
% {\normalfont\normalsize\bfseries}{\theparagraph}{1em}{}
% \titlespacing*{\paragraph}
% {0pt}{3.25ex plus 1ex minus .2ex}{1.5ex plus .2ex}

% \titlespacing*{\section} {0pt}{2.5ex plus 1ex minus .2ex}{1.3ex plus .2ex}
% \titlespacing*{\subsection} {0pt}{2.25ex plus 1ex minus .2ex}{0.5ex plus .2ex}

% For Gantt Chart
% \newcounter{myWeekNum}
% \stepcounter{myWeekNum}
% 
% \newcommand{\myWeek}{\themyWeekNum
%     \stepcounter{myWeekNum}
%     \ifnum\themyWeekNum=53
%          \setcounter{myWeekNum}{1}
%     \else\fi
% }
% 
% \newcolumntype{L}[1]{>{\raggedright\let\newline\\\arraybackslash\hspace{0pt}}m{#1}}
% \newcolumntype{C}[1]{>{\centering\let\newline\\\arraybackslash\hspace{0pt}}m{#1}}
% \newcolumntype{R}[1]{>{\raggedleft\let\newline\\\arraybackslash\hspace{0pt}}m{#1}}
%
% \makeatletter
% \def\@chapter[#1]#2{\ifnum \c@secnumdepth >\m@ne
%                          \refstepcounter{chapter}%
%                          \typeout{\@chapapp\space\thechapter.}%
%                          \addcontentsline{toc}{chapter}%
%                                    {\protect{BAB \numberline{\thechapter}\texorpdfstring{\uppercase{#1}}{#1}}}%
%                     \else
%                       \addcontentsline{toc}{chapter}{\texorpdfstring{\uppercase{#1}}{#1}}%
%                     \fi
%                     \chaptermark{#1}%
%                     \addtocontents{lof}{\protect\addvspace{10\p@}}%
%                     \addtocontents{lot}{\protect\addvspace{10\p@}}%
%                     \if@twocolumn
%                       \@topnewpage[\@makechapterhead{#2}]%
%                     \else
%                       \@makechapterhead{#2}%
%                       \@afterheading
%                     \fi}
% \makeatother


\makeatletter

\makeatother
%\bibliography{references}
\begin{document}

%Basic configuration
%\title{Pengembangan Aplikasi Pengumpulan Data Menggunakan \textit{Spreadsheet}}
\title{Pengembangan Kakas Pengumpulan Data dalam Format \textit{Spreadsheet}}
\date{}
\author{
    Feryandi Nurdiantoro \\
    NIM : 13513042
}

\frontmatter
\input{frontpages/000-cover}
\clearpage
\pagestyle{empty}

\begin{center}
    \smallskip

    {\Large \bfseries Lembar Pengesahan}

    \MakeUppercase{\normalsize \bfseries \thetitle}
    \vfill

    \normalsize \jenislaporan \\
    Program Studi: Sarjana Sistem dan Teknologi Informasi \\
    Sekolah Teknik Elektro dan Informatika \\
    Institut Teknologi Bandung \\
    \vfill

    \normalsize oleh :

    \normalsize \theauthor \\
    \normalsize NIM: \nim

    \vfill
    \normalsize \normalfont
    Telah disetujui dan disahkan sebagai Laporan \jenislaporan \\
    di Bandung, pada tanggal \thedate{} \yearsidang{}.

    \vfill
    \setlength{\tabcolsep}{12pt}
    \begin{tabularx}{\textwidth}{c@{\hskip 0.2\textwidth}cc@{\hskip 0.3\textwidth}}
         & {\bfseries Pembimbing}                                  & \\
         &                                                         & \\
         &                                                         & \\
         &                                                         & \\
         &                                                         & \\
         & \underline{\pembimbing} & \\
         & \nipPembimbing                            &
    \end{tabularx}

\end{center}
\clearpage

\chapter*{Lembar Pernyataan}

Dengan ini saya menyatakan bahwa:

\begin{enumerate}

    \item Pengerjaan dan penulisan Laporan Tugas Akhir ini dilakukan tanpa menggunakan bantuan yang tidak dibenarkan.
    \item Segala bentuk kutipan dan acuan terhadap tulisan orang lain yang digunakan di dalam penyusunan laporan tugas akhir ini telah dituliskan dengan baik dan benar.
    \item Laporan Tugas Akhir ini belum pernah diajukan pada program pendidikan di perguruan tinggi mana pun.

\end{enumerate}

Jika terbukti melanggar hal-hal di atas, saya bersedia dikenakan sanksi sesuai dengan Peraturan Akademik dan Kemahasiswaan Institut Teknologi Bandung bagian Penegakan Norma Akademik dan Kemahasiswaan khususnya Pasal 2.1 dan Pasal 2.2.
\vspace{15mm}

Bandung, 4 Agustus 2017 \\
\vspace{20mm} \\
Feryandi Nurdiantoro \\
NIM 13513042 \\
\clearpage

\pagestyle{plain}

\clearpage
\chapter*{ABSTRAK}
\addcontentsline{toc}{chapter}{Abstrak}

Dalam aplikasi dengan arsitektur Microservice, ketika terjadi perubahan yang mengakibatkan penurunan kinerja atau regresi, sulit untuk melakukan analisis atau pemeriksaan bagian mana dari keseluruhan Microservice yang menjadi penyebab utama regresi tersebut karena sifat terdistribusi dari Microservice. Dengan bantuan dari \textit{distributed tracing} dapat dilakukan analisis untuk menentukan apakah terjadi regresi dan penyebab regresi dengan memanfaatkan data \textit{latency} dari operasi-operasi yang ada pada Microservice.
%Analisis regresi kinerja atau \textit{Performance Regression Analysis} adalah sebuah teknik \textit{debugging} yang bertujuan untuk mendeteksi dan mencari sumber penyebab regresi atau penurunan kinerja pada aplikasi dengan arsitektur Microservice. Pendekatan baru dalam melakukan \textit{debugging} diperlukan pada aplikasi dengan arsitektur Microservice karena sifat dari Microservice itu sendiri yang terdistribusi sehingga jika menggunakan pendekatan \textit{debugging} tradisional dengan mencari penyebab regresi masing-masing pada setiap \textit{service} akan menjadi tidak efisien terlebih jika banyaknya \textit{service} sudah mencapai ratusan ataupun ribuan. Oleh karena itu dibutuhkan pendekatan yang dapat menganalisis regresi kinerja pada seluruh \textit{service} yang ada secara terpusat. 
%
%Pendekatan yang digunakan untuk melakukan analisis regresi kinerja pada Tugas Akhir ini adalah dengan menggunakan \textit{distributed tracing}. \textit{Distributed tracing} adalah sebuah tipe logging terkorelasi yang dapat memberikan pengguna mendapatkan visibilitas pada operasi dari perangkat lunak terdistribusi untuk kasus seperti debugging di environment \textit{production}, dan analisis penggalian akar masalah pada kegagalan atau insiden lainnya. Dengan menggunakan \textit{distributed tracing}, kinerja dari setiap \textit{service} yang diukur melalui metrik \textit{latency} dapat dianalisis secara terpusat.

Tugas Akhir ini mengimplementasikan sistem \textit{Performance Regression Analysis} dengan menggunakan kakas \textit{distributed tracing} Zipkin. Pendeteksian regresi dilakukan dengan menggunakan analisis statistik Kolmogorov-Smirnov yang akan membandingkan sampel data \textit{latency} yang terjadi secara \textit{real-time} dan sampel data \textit{latency} \textit{baseline} yang merepresentasikan kinerja aplikasi dalam keadaan normal. Analisis akan membandingkan \textit{Cummulative Distribution Function} (CDF) dari kedua sampel tersebut dan melihat apakah kedua CDF berasal dari distribusi yang berbeda. Jika didapat bahwa kedua CDF berasal dari distribusi yang berbeda, maka dapat dicurigai telah terjadi regresi kinerja sebab distribusi dari data \textit{real-time} telah menyimpang dari distribusi data \textit{baseline}.

Jika regresi terdeteksi, maka selanjutnya sistem akan melakukan analisis \textit{critical path} yaitu melihat operasi mana saja dari setiap \textit{service} yang berkemungkinan besar berkontribusi pada regresi tersebut. Analisis akan dilakukan dengan mencari selisih \textit{latency} dari data operasi \textit{real-time} dan \textit{baseline} dan akan terlihat mana operasi yang selisih \textit{latency} nya melebihi batas yang telah ditentukan sebelumnya.

Sistem \textit{Performance Regression Analysis} telah diujikan pada aplikasi Microservice yang dijalankan di Kubernetes. Hasilnya adalah dari 10 dari 11 kasus uji, regresi berhasil dideteksi dan pada 6 dari 10 kasus tersebut berhasil ditentukan operasi penyebab regresi. Pengimplementasian sistem menambah \textit{overhead} rata-rata penggunaan CPU sebesar 0,78\% dan penggunaan Memory sebesar 0,67\%.



\vspace{15mm}
Kata kunci: Regresi Kinerja, \textit{Distributed Tracing}, Microservice, Kubernetes
\clearpage
%\clearpage
\chapter*{Abstract}
\addcontentsline{toc}{chapter}{Abstract}

The usage of spreadsheet in daily basis, usually used as a data collector. Spreadsheet is one of the easy to use software, so that many people prefer it as data management software than using a proper database management system. This could causing so many problems because the spreadsheet itself is not design to be a data collector. Some problem presisted as people use is as data collector, such as no data validation, isolation of data, and hard to collaborate. In order to solve those problem, this final year project want to create a spreadsheet software that could integrate with existing database and synchronize it with the data in the spreadsheet, could verify the data, and easy to collaborate.

This report will cover the analysis behind the software requirement in order to get the integrated software with the database. This will include data transformation with 4 steps, clustering, row identification, cell identification, and header-data assignment. This software is using two machine learning algorithm in order to do a proper transformation, the Hierarchical Clustering and Conditional Random Field. The training data set is from Statistical Abstract of the United States (SAUS) 2010 and some manual data gathering. The next step is validation, this software will cover 3 type of validation, such as data type, data domain, and data relation. Those algorithm build on top of an open source spreadsheet software called EtherCalc which already could do online collaboration.

Keyword: spreadsheet, data collector, data quality, data management.
\clearpage
\chapter*{Kata Pengantar}
\addcontentsline{toc}{chapter}{KATA PENGANTAR}

Puji syukur saya panjatkan kehadirat Tuhan Yang Maha Esa, karena degan kelimpahan rahmat dan kemurahan hati-Nya penulis dapat menyelesaikan Tugas Akhir yang berjudul "\thetitle". 

Dalam penyusunan Tugas Akhir ini penulis banyak mendapatkan masukan, kritik, dorongan, bantuan, bimbingan, serta dukungan baik secara fisik maupun moral dari berbagai pihak yang merupakan pengalaman yang berharga yang tidak dapat diukur secara materi dan dapat menjadi pembelajaran yang berharga dikemudian hari. Oleh karena itu dengan segala hormat dan kerendahan hati perkenankanlah penulis mengucapkan terima kasih kepada:

\begin{enumerate}
	\item Bapak Yudistira Dwi Wardhana Asnar, Ph.D dan Ibu Tricya Esterina Widagdo, ST., M.Sc. selaku pembimbing yang senantiasa memberikan arahan dan masukan selama pengerjaan Tugas Akhir.
	\item Bapak Adi Mulyanto, ST., MT. selaku dosen penguji yang atas saran dan masukkannya membuat Tugas Akhir ini menjadi lebih baik.
	\item Kedua orang tua penulis. Terima kasih atas dukungan baik secara moral dan material sehingga penulis dapat melaksanakan Tugas Akhir ini hingga selesai dengan baik.
	\item Ibu Dr. Fazat Nur Azizah ST, M.Sc. selaku dosen Tugas Akhir yang telah mendukung saya dan memberikan arahan dalam penyelesaian dan pengerjaan Tugas Akhir ini.
	\item Ibu Dr. Eng. Ayu Purwarianti, ST., MT. selaku dosen mentor Imagine Cup yang ikut membantu dan memberikan dukungan agar penulis dapat menyelesaikan pengerjaan Tugas Akhir.
	\item Bapak Dr.techn. Saiful Akbar ST, MT. dan Bapak Achmad Imam Kistijantoro, ST, M.Sc, Ph.D selaku Ketua Program Studi dari Teknik Informatika dan Sistem dan Teknologi Informasi yang ikut mendukung penulis dalam menyelesaikan Tugas Akhir ini.
	\item Seluruh dosen, karyawan dan civitas program studi Teknik Informatika, Institut Teknologi Bandung.
	\item Rekan-rekan yang telah penulis minta bantuan dalam penyelesaian administrasi Tugas Akhir ini disaat penulis berhalangan terutama Muhamad Visat Sutarno dan Fiqie Ulya Sidiastahta
	\item Rekan-rekan laboratorium basis data yang saling mendukung dalam menyelesaikan Tugas Akhir ini pada khususnya Vanya Deasy Safrina, Albert Tri Adrian, Marco Orlando, Fiqie Ulya Sidiastahta, dan Wilhelmus Andrian.
	\item Rekan-rekan seperjuangan Teknik Informatika yang menamakan dirinya Happy Anti Wacana yang selalu mendukung penulis untuk mengerjakan Tugas Akhir.
	\item Rekan-rekan dari Binary 2013 dan HMIF yang telah memberikan dukungan dan bantuan dalam pengerjaan Tugas Akhir.
	\item Pihak-pihak lain yang tidak dapat disebutkan satu-persatu.
\end{enumerate}

Penulis menyadari bahwa Tugas Akhir ini masih jauh dari sempurna serta memiliki banyak kekurangan. Oleh karena itu, penulis sangat terbuka dalam menerima kritik dan saran yang membangun untuk Tugas Akhir ini. Semoga Tugas Akhir ini dapat bermanfaat bagi pembaca.

\begin{flushright}
Bandung, Agustus 2017 \\
\vspace{25mm}
Penulis
\end{flushright}


\titleformat*{\section}{\centering\bfseries\Large\MakeUpperCase}

\clearpage
\tableofcontents

\addcontentsline{toc}{chapter}{DAFTAR ISI}
% \afterpage{\null\newpage}
% \addcontentsline{toc}{chapter}{DAFTAR LAMPIRAN}
{%
    \let\oldnumberline\numberline%
    \renewcommand{\numberline}{\figurename~\oldnumberline}%
    \listoffigures%
}
\addcontentsline{toc}{chapter}{DAFTAR GAMBAR}
{%
    \let\oldnumberline\numberline%
    \renewcommand{\numberline}{\tablename~\oldnumberline}%
    \listoftables%
}
\addcontentsline{toc}{chapter}{DAFTAR TABEL}

%----------------------------------------------------------------%
% Konfigurasi Bab
%----------------------------------------------------------------%
\renewcommand{\chaptername}{BAB}
\renewcommand{\thechapter}{\Roman{chapter}}
%----------------------------------------------------------------%

\titleformat*{\section}{\bfseries\large}
\mainmatter
%----------------------------------------------------------------%
% Dafter Bab
% Untuk menambahkan daftar bab, buat berkas bab misalnya `chapter-6` di direktori `chapters`, dan masukkan ke sini.
%----------------------------------------------------------------%
\chapter{Pendahuluan}

Pada bab ini akan dibahas mengenai gambaran dasar dari pelaksanaan Tugas Akhir dalam bentuk penjelasan latar belakang yang mendasari pemilihan topik. Dari latar belakang tersebut, akan diurai kembali menjadi rumusan masalah, tujuan, batasan masalah, metodologi, serta sistematika pembahasan tugas akhir.                                                                        

\section{Latar Belakang}
\label{ch1-latbel}


Dengan banyaknya adopsi penggunaan arsitektur Microservice yang berbasis sistem terdistribusi saat ini, semakin banyak juga tantangan yang muncul terkait dengan adopsi gaya arsitektur tersebut. Arsitektur Microservice menawarkan beberapa keuntungan antara lain heterogenitas teknologi, resiliensi, kemudahan \textit{scaling}, kemudahan \textit{deployment}, kemudahan dalam melakukan penyelarasan secara tim teknologi, fleksibilitas dalam menentukan komposisi aplikasi, dan sistem yang teroptimasi untuk melakukan penggantian kompoen satu sama lain \citep{building-microservices}. Keuntungan-keuntungan tersebut menjadikan adopsi aristektur Microservice cukup marak terutama pada aplikasi-aplikasi yang membutuhkan \textit{scalability} untuk melayani kebutuhan pelanggan yang semakin banyak.  

%Salah satu perusahaan yang mengadopsi arsitektur Microservice secara masif adalah Netflix \citep{netflix-nginx, netflix-infoq}. Dengan menggunakan arsitektur Microservice, para \textit{developer} di Netflix dapat melakukan dekomposisi secara \textit{decoupling} yang dapat memungkinkan \textit{developer} untuk menggunakan berbagai \textit{framework} atau bahasa pemrgoraman yang berbeda untuk setiap service nya. Hal tersebut menjadi keuntungan besar sebab para \textit{developer} dapat mendesain sebuah service dengan kemampuan dan tujuan tertentu dengan sebuah bahasa pemrograman yang sesuai dengan tujuan dari service tersebut. Kemampuan untuk melakukan hal tersebut didukung oleh teknologi \textit{container} yang dipopulerkan dengan kehadiran dari Docker dan juga teknologi \textit{container orchestration} oleh Kubernetes. Dengan adanya Kubernetes, sistem dapat melakukan \textit{scale up} dan \textit{scale down} sebuah service ketika dirasa ada \textit{workload} lebih besar yang dibutuhkan untuk mengakomodasi kebutuhan \textit{request}.

Dari banyaknya keuntungan yang ditawarkan oleh arsitektur Microservice, timbul suatu permasalahan yang secara alami muncul ketika jumlah service bertambah yaitu meningkatnya kompleksitas dari sistem \citep{fowler-complexity}. Salah satu dari banyaknya kompleksitas muncul adalah dalam proses \textit{debugging}. Pada umumnya, dalam aplikasi dengan aristektur Monolith, proses \textit{debugging} hanya dilakukan pada satu sumber, sehingga jika terdapat masalah maka akan dapat dengan mudah ditemukan. Namun, dalam model aplkasi dengan arsitektur Microservice yang bisa memiliki ratusan bahkan ribuan service sekaligus, hal tersebut tidak akan menjadi mudah, sebab suatu \textit{bug} tertentu akan sulit dilacak jika masih menggunakan metode biasa. Hal tersebut wajar terjadi mengingat sifat terdistribusi dari service yang ada. Dari permasalahan \textit{debugging} yang terdistribusi itulah muncul suatu inisiatif untuk membuat kakas yang dapat melakukan pelacakan dari service-service pada suatu sistem Microservice agar para \textit{developer} dapat mendapatkan petunjuk atau \textit{observability} mengenai kondisi internal yang terjadi pada sebuah service.

Ada tiga pilar untuk mencapai \textit{observability} pada sebuah sistem terdistribusi, yaitu melalui \textit{log}, \textit{metric}, dan \textit{trace} \citep{sridharan2018distributed}. \textit{Log} atau \textit{event log} adalah suatu catatan yang bersifat \textit{immutable} dari \textit{event} yang terjadi sepanjang waktu. Sebuah \textit{event log} pada umumnya terdiri dari informasi mengenai \textit{timestamp} dan \textit{payload} yang berisikan konteks. \textit{Metric} dan \textit{trace} merupakan abstraksi yang dibuat di atas \textit{log} yang melakukan pra-pemrosesan dan melakukan dekode informasi berdasarkan dua sumbu, yang satu bersifat \textit{request} sentris yaitu \textit{trace}, dan yang lainnya bersifat sistem sentris yaitu \textit{metric}.

Pada akhirnya \textit{observability} hanya memiliki dua tujuan utama \citep{parker2020distributed}, yaitu :
\begin{enumerate}
	\item Meningkatkan kinerja  \textit{baseline} pada sistem
	\item Mengembalikan kinerja \textit{baseline} setelah terjadi regresi pada sistem
\end{enumerate}

Dengan meningkatkan kinerja \textit{baseline} pada sistem, para \textit{developer} berharap dapat meningkatkan kepuasan pelanggan, menurunkan biaya infrastruktur, ataupun keduanya. Untuk aplikasi yang langsung melayani pelanggan, kinerja seringkali berarti \textit{latency} dari \textit{request}. Proses optimisasi seperti ini biasanya adalah proses bertahap yang membutuhkan waktu.

Kakas yang digunakan untuk membantu mencapai \textit{observability} penting untuk meningkatkan kinerja \textit{baseline} dengan pertama kali mengukur kinerja awal yang menjadi \textit{baseline} pengukuran dan digunakan untuk mengarahkan para \textit{developer} agar dapat mencari bagian mana dari aplikasi yang dapat ditingkatkan kinerjanya. Dengan aplikasi yang menggunakan arsitektur Monolith, para \textit{developer} dapat dengan mudah melakukan \textit{profiling} proses mana saja yang dapat ditingkatkan penggunaan \textit{resource}-nya seperti CPU ataupun Memory. Namun dengan penggunaan arsitektur Microservice yang berbasis sistem terdistribusi, terkadang sulit untuk mengetahui \textit{service} manakah yang tepatnya perlu ditingkatkan penggunaan \textit{resource}-nya, sehingga penggunaan \textit{distributed tracing} akan sangat membantu untuk meningkatkan kinerja \textit{baseline} dari sebuah sistem terdistribusi.

Berbeda dengan tujuan untuk meningkatkan kinerja \textit{baseline}, tujuan lainnya yaitu untuk mengembalikan kinerja \textit{baseline} bukanlah suatu hal yang dapat begitu saja direncanakan. Regresi dalam kinerja dapat muncul tiba-tiba seperti terjadinya \textit{outtage}, terjadi \textit{error} pada salah satu service, pembaharuan pada sistem, atau pemberhentian tiba-tiba dari sebuah \textit{cluster} sistem terdistribusi. Melihat sifat dari sistem terdistribusi, mencari penyebab utama dari sebuah \textit{outtage} bukanlah sebuah hal yang mudah terlebih jika ada ratusan bahkan ribuan \textit{node} yang terdapat pada \textit{cluster} dan masing-masing \textit{node} saling terhubung dengan yang lainnya. Jika hal semacam tersebut terjadi dalam lingkungan aplikasi \textit{production} maka dampaknya akan terasa langsung oleh pelanggan dan dalam jangka panjang dapat menimbulkan kerugian material. Oleh karena itu, penting untuk segera mengetahui sumber atau akar dari suatu kejadian yang menyebabkan regresi pada kinerja sistem.

Salah satu hal yang dapat dilakukan untuk mengembalikan kinerja sistem setelah terjadi regresi adalah melakukan \textit{Root Cause Analysis} (RCA) menggunakan hasil \textit{trace} dari \textit{distributed tracing}.  Dengan bantuan dari \textit{trace}, \textit{developer} bisa mendapatkan suatu gambaran dari masing-masing \textit{request} yang terjadi pada sebuah \textit{resource} atau komponen yang berinteraksi dengan komponen lainnya dalam sebuah Sistem Terdistribusi seperti \textit{node}, \textit{service}, \textit{network}, ataupun \textit{mutex}. 





%Ide dasar dari \textit{tracing}  adalah dengan mengidentifikasi sebuah titik spesifik, dapat jadi sebuah pemanggilan \textit{remote procedure call} (RPC) dalam sebuah aplikasi, \textit{library}, ataupun \textit{middleware} dalam jalur sebuah \textit{request} yang merepresentasikan kedua hal berikut:
%\begin{enumerate}
%	\item \textit{Fork} pada eksekusi di level Sistem Operasi
%	\item Sebuah lompatan atau akses horizontal ke luar melalui jaringan
%\end{enumerate}


%Sudah terdapat beberapa solusi \textit{tracing} yang ada sepanjang beberapa tahun ke belakang, dimulai dengan dipublikasikannya \textit{paper} dari Google yang mengajukan solusi \textit{tracing} untuk sistem terdistribusi bernama Dapper \citep{dapper-paper} hingga yang terbaru solusi \textit{tracing} berbasis transparan yang bernama Inkle \citep{tracing-abram}. Solusi \textit{tracing} yang diajukan Inkle berfokus untuk membuat kakas instrumentasi \textit{tracing} yang transparan, artinya proses instrumentasi tidak mengubah sama sekali kode dari sebuah service, melainkan melakukan instrumentasi data dari service dengan melakukan \textit{intercept} \textit{traffic} jaringan dari service tersebut. Metode \textit{tracing} seperti itu disebut juga dengan \textit{passive tracing}. Hasil dari metode \textit{tracing} seperti yang dilakukan Inkle adalah \textit{tracing} dapat tetap dilakukan tanpa adanya intervensi sama sekali pada level kode di dalam service, namun kekurangannya adalah akurasi hasil \textit{tracing} yang rendah dan juga hasil \textit{tracing} antar service tidak dapat menampilkan kausalitas atau hubungan sebab akibat dari \textit{request} yang dilakukan.
%
%Metode lain untuk melakukan instrumentasi pada service adalah dengan menggunakan \textit{library} pada level kode untuk memberikan konteks dan informasi kepada agen \textit{tracing}. Sudah ada beberapa \textit{library} ataupun \textit{framework} \textit{open source} yang menyediakan solusi untuk melakukan instrumentasi pada level kode seperti contohnya adalah OpenTracing, Zipkin, dan Jaeger \citep{opentracing, zipkin, jaeger}. Metode untuk melakukan tracing ini disebut juga dengan \textit{active tracing}. Dengan metode ini, data yang diperoleh dari instrumentasi akan lebih kaya, salah satunya adalah dapat memberikan konteks pemanggilan ke service lain yang nantinya dapat diolah lebih lanjut untuk mendapatkan gambaran yang lebih besar yaitu \textit{request causality} antar service.
%
%Namun dari solusi \textit{open source} yang ada, penulis menemukan adanya kekurangan dalam kakas visualisasi hasil \textit{tracing}. Penulis menemukan hanya Zipkin yang memiliki kakas visualisasi yang dapat langsung digunakan. Namun fitur visualisasi Zipkin masih berfokus pada penyediaan data \textit{tracing} dalam sebuah service dan untuk penyediaan data secara keseluruhan antar service masih belum dapat menampilkan data yang perlu untuk mencapai \textit{observability} bagi keseluruhan sistem Microservice. Dari kondisi itulah penulis menilai perlu adanya kakas \textit{open source} yang dapat menyediakan \textit{developer} informasi yang diperlukan untuk mencapai \textit{observability} dalam sebuah sistem Microservice secara utuh. 


\section{Rumusan Masalah}\label{RumusanMasalah}

Berdasarkan latar belakang yang telah dijelaskan pada Subbab \ref{ch1-latbel} terdapat beberapa permasalahan terkait \textit{Root Cause Analysis} pada Microservice. Rumusan masalah yang akan diselesaikan pada Tugas Akhir ini adalah sebagai berikut:
\begin{enumerate}
	\item 
	\item Bagaimana pengaruh kakas \textit{distributed tracing} terhadap kinerja aplikasi maupun infrastruktur?
\end{enumerate}

\section{Tujuan}

Tujuan yang ingin dicapai dalam Tugas Akhir ini adalah:
\begin{enumerate}
	\item Mengembangkan kakas untuk melakukan \textit{Root Cause Analysis} menggunakan \textit{distributed tracing}  yang dapat menampilkan visualisasi dari hasil \textit{tracing} pada sistem Microservice yang menggunakan protokol gRPC dan berjalan di atas Kubernetes sehingga
	\item Mengukur \textit{overhead} yang diakibatkan oleh kakas \textit{distributed tracing} baik pada level aplikasi maupun level infrastruktur
\end{enumerate}


\section{Batasan Masalah}

Dalam pengerjaan Tugas Akhir ini, terdapat beberapa batasan-batasan yang perlu diperhatikan. Batasan tersebut ditujukan untuk memperjelas dan memfokuskan objek penelitian dan pengembangan tugas akhir. Batasan-batasan masalah pengerjaan tugas akhir adalah sebagai berikut,

\begin{enumerate}
	\item \textit{Library} instrumentasi yang digunakan pada level kode Microservice tidak akan dibuat dari awal namun akan mengimplementasikan \textit{library} instrumentasi yang bersifat \textit{open source}
	\item Protokol komunikasi Microservice yang akan diuji terbatas pada protokol gRPC
	\item Kasus yang diuji terbatas pada komunikasi antar aplikasi Microservice yang berjalan di atas Kubernetes
	\item Kasus yang diuji tidak termasuk komunikasi antar aplikasi Microservice yang menggunakan protokol HTTPS
\end{enumerate}

\section{Metodologi}

Metodologi yang digunakan dalam pengerjaan Tugas Akhir ini yakni:
\begin{enumerate}
    \item Studi Literatur \\
          Pengerjaan tugas akhir diawali dengan mencari dan mempelajari referensi berupa buku, jurnal ilmiah dan solusi \textit{Open Source}  yang telah ada sebelumnya yang dapat membantu pengembangan kakas yang dibuat pada tugas akhir ini. Literatur yang dicari dan dipelajari berkaitan dengan topik tugas akhir yaitu mengenai \textit{Distributed Tracing},  protokol komunikasi Microservice, Kubernetes, serta hal-hal lain yang masih berkaitan dengan topik tugas akhir ini.

    \item Analisis Masalah \\
          Pada tahap ini dilakukan analisis permasalahan yang berkaitan dengan topik yang diangkat pada tugas akhir ini. Diantaranya adalah menganalisis kebutuhan instrumentasi pada protokol Microservice, kebutuhan visualisasi dari hasil \textit{tracing}, dan kebutuhan infrastruktur \textit{deployment} pada Kubernetes.

    \item Perancangan Solusi \\
          Pada tahap ini dilakukan perancangan solusi yang dapat menyelesaikan masalah-masalah yang telah dijelaskan pada bagian analisis masalah. Bagian perancangan ini juga menjelaskan arsitektur yang digunakan untuk membangun perangkat lunak berdasarkan spesifikasi dan metode yang digunakan.

    \item Implementasi \\
          Pada tahap ini dilakukan pembangunan kakas sesuai dengan kebutuhan dan spesifikasi dari hasil analisis masalah serta rancangan solusi yang diajukan.

    \item Pengujian dan Analisis Hasil \\
          Pada tahap ini dilakukan pengujian dengan menggunakan data set uji yang sesuai dengan batasan masalah ke dalam kakas yang diimplementasikan. Selanjutnya dilakukan analisis hasil pengujian dan penarikan kesimpulan.

\end{enumerate}

\section{Sistematika Pembahasan}

Penulisan tugas akhir ini terdiri dari 5 bab, yaitu: BAB I Pendahuluan, BAB II Studi Literatur, BAB III Analisis dan Perancangan, BAB IV Rancangan, Implementasi, dan Pengujian, dan BAB V Penutup.

Bab satu membahas mengenai latar belakang permasalahan, rumusan masalah, tujuan, batasan masalah, metodologi serta sistematika pembahasan yang digunakan. Bab ini juga menjelaskan secara umum isi dari tugas besar serta gambaran dasar dari pelaksanaan tugas akhir.

Bab dua menjelaskan mengenai dasar teori yang digunakan di dalam menyelesaikan permasalahan yang diangkat. Teori yang digunakan berasal dari literatur dan referensi yang berhubungan dengan permasalahan yang diangkat seperti hal-hal yang berkaitan dengan \textit{Distributed Tracing}, protokol komunikasi Microservice, dan Kubernetes. Dasar teori ini menjadi dasar analisis dan rancangan solusi pada bab selanjutnya.

Bab tiga memaparkan analisis permasalahan yang terkait dengan visualisasi hasil Distributed Tracing pada Microservice beserta rancangan solusi yang akan digunakan untuk mengatasi permasalahan tersebut. Selanjutnya solusi umum tersebut akan dibuat rancangan dan arsitekturnya agar dapat diimplementasikan.

Bab empat memperlihatkan rancangan perangkat lunak yang dibuat serta hasil implementasinya. Pada akhir bab akan ditunjukkan hasil pengujian yang dilakukan kepada kakas yang dibuat dan pembahasan dari pengujian tersebut. Pengujian dilakukan untuk mengetahui keberhasilan kakas yang dibuat untuk menyelesaikan permasalahan yang di definisikan pada rumusan masalah.

Bab lima berisikan kesimpulan terhadap hasil implementasi dan solusi yang dipaparkan untuk menyelesaikan permasalahan. Di samping itu, terdapat bagian saran yang memaparkan saran pengembangan dan perbaikan yang dapat dilakukan untuk memperkaya fitur dan menyelesaikan permasalahan yang lebih luas.

% \section{Jadwal Pelaksanaan}
% \newsavebox\mybox
% \begin{lrbox}{\mybox}
%     \begin{ganttchart}[
%     vgrid={*{6}{draw=none}, dotted},
%     x unit=.05cm,
%     y unit title=.6cm,
%     y unit chart=.6cm,
%     time slot format=isodate,
%     time slot format/start date=2016-09-01]{2021-09-01}{2022-04-30}
%     \ganttset{bar height=.6}
%     \gantttitlecalendar{year, month} \\
%     \ganttbar[bar/.append style={fill=blue}]{Studi Literatur}{2021-09-01}{2021-11-30}\\
%     \ganttbar[bar/.append style={fill=blue}]{Analisis Masalah}{2021-10-01}{2021-11-15}\\
%     \ganttbar[bar/.append style={fill=blue}]{Perancangan Solusi}{2021-11-01}{2021-12-15}\\
%     \ganttbar[bar/.append style={fill=blue}]{Implementasi}{2021-12-15}{2022-03-01}\\
%     \ganttbar[bar/.append style={fill=blue}]{Pengujian dan Analisis Hasil}{2022-02-01}{2022-04-30}
%     \end{ganttchart}
% \end{lrbox}
%
% Pengerjaan tugas akhir ini direncanakan mulai pada September 2021 sampai April 2022. Pelaksanaan tugas akhir ini dibagi menjadi 5 tahap yang dapat dipetakan kepada metodologi pengerjaan sebagai berikut,
% \begin{enumerate}
%     \item Tahap 1: Studi Literatur
%     \item Tahap 2: Analisis Masalah
%     \item Tahap 3: Perancangan Solusi
%     \item Tahap 4: Implementasi
%     \item Tahap 5: Pengujian dan Analisis Hasil
% \end{enumerate}
% Jadwal pelaksanaan tugas akhir berdasarkan metodologi pengerjaan tugas akhir dapat dilihat pada Tabel \ref{Gantt-Chart} dibawah ini.
% \begin{table}[htb]
% \centering
% \caption{Gantt Chart jadwal pelaksanaan tugas akhir}
% \label{Gantt-Chart}
% \tikz{
%   \node[inner sep=0pt,outer sep=0pt] (gantt)
%   {\begin{tabular}{c}
%     \toprule
%     \resizebox{\textwidth}{!}{\usebox\mybox} \\
%     \bottomrule
%    \end{tabular}%
%    };
% }   
% \end{table}
\chapter{Studi Literatur}

Pada bab ini akan dideskripsikan kajian literatur yang terkait dengan persoalan Tugas Akhir. Studi literatur ini akan dijadikan dasar di dalam melakukan penyelesaian persoalan yang telah didefinisikan.

\section{\textit{Spreadsheet}}
Bagian ini akan membahas mengenai definisi umum \textit{spreadsheet} serta teknologi yang sering digunakan di dalam membuat \textit{spreadsheet}. Dengan adanya definisi umum ini diharapkan dapat menyamakan persepsi mengenai \textit{spreadsheet} yang dimaksud pada Tugas Akhir ini. Teknologi yang dijelaskan pada bagian ini merupakan teknologi yang memungkinkan untuk dijadikan dasar pengembangan perangkat lunak pada Tugas Akhir ini.

\subsection{Definisi Umum}
Secara harafiah, \textit{spreadsheet} adalah suatu perangkat lunak yang dapat melakukan kalkulasi terhadap angka serta mengorganisir informasi yang ada di dalamnya berdasarkan kolom dan baris \citep{meriamwebster-spreadsheet}. Konsep dasar pada aplikasi \textit{spreadsheet} modern adalah sebuah aplikasi yang berupa sekumpulan sel terdiri dari baris dan kolom yang disebut \textit{sheet} yang dapat digambarkan sebagai matriks yang besar \citep{Ronen1989}.

Sel-sel pada \textit{spreadsheet} dapat diisi data berupa data mentah maupun formula. Data mentah dapat berupa angka, teks, tanggal, dan nilai mata uang. Formula merupakan perintah yang dapat dimengerti komputer untuk menghitung dan memanipulasi data pada sel. Data hasil pengolahan dan masukan pada \textit{spreadsheet} ditampilkan dalam bentuk sel yang namanya terdiri dari nama kolom dan nilai baris (Contoh: A1 untuk kolom pertama dan baris pertama). Di samping itu, sel tersebut juga dapat memiliki \textit{properties} berupa \textit{value} yang diisikan, format sel, serta format data yang digunakan.

\subsection{Teknologi \textit{Spreadsheet}} \label{TeknologiSpreadsheet}
Perkembangan teknologi \textit{spreadsheet} digital modern dimulai pada tahun 1978, saat Bricklin mengembangkan \textit{working prototype} dari konsep dasar \textit{spreadsheet} menggunakan Integer BASIC. Pada tahun yang sama, Frankston dan Fylstra bergabung dan membentuk sebuah perangkat lunak bernama VisiCalc (Visible Calculator) yang merupakan sebuah perangkat lunak \textit{spreadsheet} pertama yang bekerja dengan baik dan sukses dipasaran. Setelah keberhasilan VisiCalc, mulai muncul aplikasi serupa yang semakin baik salah satunya adalah Lotus. Dengan berkembangnya daya komputasi dan munculnya konsep \textit{graphical user interface}, Microsoft mengembangkan Microsoft Excel yang merupakan \textit{spreadsheet} pertama yang menggunakan antarmuka grafis dan menggunakan \textit{mouse} sebagai alat kontrol \citep{power2004brief}.

Saat ini, perangkat lunak berupa \textit{spreadsheet} sangat banyak variasi dan tipenya. Perangkat lunak \textit{spreadsheet} ini dapat dibagi berdasarkan konektivitasnya yakni \textit{offline spreadsheet} dan \textit{online spreadsheet}. Selain itu, perangkat lunak \textit{spreadsheet} dapat juga dibagi berdasarkan keterbukaan dari \textit{source code} yakni \textit{open source} dan \textit{closed source}. Bagian ini akan membahas masing-masing perangkat lunak tersebut secara umum.

    \subsubsection{Microsoft Excel}
    Microsoft Excel adalah perangkat lunak yang dikembangkan oleh Microsoft yang menyediakan fitur dasar dari \textit{spreadsheet} serta dengan fitur-fitur lainnya yang selalu ditambahkan pada setiap iterasi pengembangan Excel. Microsoft Excel dapat dimiliki oleh pengguna melalui pembelian paket Microsoft Office yang berisikan produk esensial Microsoft lainnya \citep{MSExcelProduct}. 

    Sejak Microsoft Excel 2007, Microsoft menggunakan format Office Open XML (OOXML) sebagai format penyimpanan \citep{MSExcelSupport}. Office Open XML dikembangkan oleh Microsoft mulai dari tahun 2000 dengan diimplementasinya dukungan XML pada Microsoft Office 2000. Pada awal penggunaan aplikasi \textit{office}, terdapat permasalahan \textit{data interoperability} antar mesin dan sulitnya manipulasi data. Office Open XML diharapkan dapat menyelesaikan permasalahan ini dengan membentuk standar yang dapat diimplementasi berbagai aplikasi \textit{office} \citep{OOXMLFormat}.

    \subsubsection{LibreOffice Calc}
    LibreOffice adalah perangkat lunak yang dikembangkan oleh komunitas dan proyek dari organisasi non-profit bernama The Document Foundation. LibreOffice adalah perangkat lunak yang gratis dan \textit{open source} yang awalnya didasarkan pada perangkat lunak serupa yakni OpenOffice.org dan merupakan pengembangan lanjutan dari OpenOffice yang paling aktif. Hampir serupa dengan Microsoft Office, LibreOffice memberikan 6 perangkat lunak yang ada di dalamnya yakni: Writer (pemrosesan teks), Calc (\textit{spreadsheet}), Impress (presentasi), Draw (grafik dan vektor), Base (basisdata), dan Math (editor formula) \citep{LibreOffice}. 

    LibreOffice Calc memiliki berbagai kemampuan yang dimiliki oleh kebanyakan \textit{spreadsheet}. Di dalam melakukan penyimpanan file, Calc menggunakan format OpenDocument. Format OpenDocument dikembangkan oleh Organization for the Advancement of Structured Information Standards (OASIS) yang bertujuan untuk membentuk \textit{open standard} bagi \textit{office document} \citep{OpenDocument}. 

    \subsubsection{EtherCalc} \label{AboutEtherCalc}
    EtherCalc merupakan perangkat lunak \textit{spreadsheet online} yang \textit{open-source} yang dikembangkan oleh Audrey Tang. EtherCalc merupakan pengembangan yang didasarkan dari perangkat lunak serupa yakni WikiCalc dan SocialCalc. WikiCalc merupakan aplikasi \textit{spreadsheet} yang mengandalkan komputasi server untuk dapat berkolaborasi, sedangkan SocialCalc merupakan aplikasi \textit{spreadsheet} yang menggunakan kemampuan javascript untuk melakukan komputasi pada \textit{client-side}. EtherCalc dikembangkan di atas Node.js dan menggunakan javascript sebagai alat komputasi. 

    Arsitektur aplikasi EtherCalc berjalan di atas aplikasi SocialCalc. Fungsi-fungsi dasar yang ada pada \textit{spreadsheet} hampir seluruhnya menggunakan implementasi SocialCalc. EtherCalc menjadikan aplikasi dalam bentuk Node.js, memiliki fitur kolaborasi yang lebih baik, dan beberapa fitur lain seperti pembuatan grafik data. Pada Gambar \ref{ModulEtherCalc} dapat dilihat interaksi antar modul yang terjadi pada EtherCalc.

    \begin{figure}[htb]
        \centering
        \includegraphics[width=0.6\textwidth]{resources/chapter-2-module-ethercalc.png}
        \caption{Arsitektur Module pada EtherCalc}
        \label{ModulEtherCalc}
    \end{figure}  

    Modul-modul tersebut memiliki tugas antara lain sebagai berikut:
    \begin{enumerate}
        \item Modul App \\
        Merupakan modul utama yang berisikan variabel konstanta dan pengaturan yang dibutuhkan oleh aplikasi.
        \item Modul Main \\
        Modul ini bertugas sebagai modul pertama yang menerima perintah dari pengguna dan mengatur API yang ada pada aplikasi. Modul ini mengatur hubungan antara modul player, modul database, dan modul socialcalc.
        \item Modul Socialcalc \\
        Merupakan modul yang bertugas merepresentasikan \textit{spreadsheet} pada aplikasi. Di dalam modul ini terdapat fungsi-fungsi yang dapat digunakan untuk memanipulasi sel-sel yang ada pada \textit{spreadsheet}.
        \item Modul Database \\
        Modul ini bertugas untuk memanipulasi basis data redis yang menyimpan perubahan-perubahan yang terjadi pada \textit{spreadsheet}.
        \item Modul Player \\
        Modul ini memiliki tugas untuk menjembatani \textit{front-end} dan \textit{back-end} dari aplikasi.
    \end{enumerate}

    Perangkat lunak hanya akan melakukan panggilan ke \textit{server} saat melakukan suatu aksi dan \textit{server} yang bertugas untuk menyimpan kumpulan aksi tersebut agar pengguna lain yang ikut berkolaborasi dapat melihat \textit{spreadsheet} yang sama satu dengan yang lain \citep{EtherCalc}. Gambar \ref{IlustrasiEtherCalc} merupakan gambaran umum cara kerja EtherCalc dalam berkolaborasi.

    \begin{figure}[htb]
        \centering
        \includegraphics[width=0.6\textwidth]{resources/chapter-2-ethercalc.png}
        \caption{Ilustrasi Cara Kerja Kolaborasi pada EtherCalc}
        \label{IlustrasiEtherCalc}
    \end{figure}

    \subsubsection{Google Sheet}
    Google Sheet merupakan salah satu perangkat lunak pada \textit{office suite} miliki Google. Google Sheet dapat berjalan di atas tiga \textit{platform} yang berbeda yakni: sebagai \textit{web application}, Chrome apps, dan \textit{mobile apps}. Sheet memiliki kemampuan berkolaborasi secara \textit{real-time} dan menyediakan fitur-fitur \textit{spreadsheet} yang ada pada umumnya. Sheet memiliki fitur \textit{revision history} sehingga setiap orang dapat melihat perubahan yang terjadi, \textit{add-ons} yang dapat menambahkan fitur kepada Sheet melalui program yang dibuat oleh komunitas serta fitur \textit{chat} dengan kolaborator. Google Sheet dikembangkan menggunakan bahasa \textit{javascript} yang memberikan kemampuan penyimpanan dan kolaborasi secara \textit{real-time} \citep{GoogleSheet}.

    \subsubsection{OnlyOffice}
    Merupakan perangkat lunak sebagai \textit{service} yang dikembangkan oleh Ascensio System SIA. OnlyOffice merupakan \textit{office suite} yang berjalan di atas \textit{web application} yang dipadukan dengan sistem \textit{customer relationship management}. OnlyOffice tidak hanya terdiri dari \textit{online office editor} namun juga terdapat fitur managemen dokumen, manajemen proyek, \textit{mail service}, serta manajemen pelanggan seperti kontak, \textit{invoice}, \textit{opportonities}, dan \textit{task}. OnlyOffice ditujukan kepada bisnis yang membutuhkan perangkat lunak yang dapat mengorganisir kebutuhan bisnis di dalam satu aplikasi yang saling terintegrasi \citep{OnlyOffice}. 


\section{Penggunaan \textit{Spreadsheet}}
\textit{Spreadsheet} dapat digunakan untuk melakukan kalkulasi terhadap suatu rumus atau formula yang sulit jika dikalkulasikan dengan cara manual. Di samping itu, \textit{spreadsheet} dapat juga digunakan untuk melakukan ramalan terhadap suatu perubahan variabel masukan. Pada perkembangannya, \textit{spreadsheet} memiliki fitur-fitur tambahan seperti visualisasi data dan ekstraksi data penting dari kumpulan data yang ada.

Penelitian tentang penggunaan \textit{spreadsheet} pada bisnis pernah dilakukan sebelumnya pada tahun 2014. Subjek yang diteliti adalah akuntan manajemen \citep{Bradbard2014}. Pada penelitian tersebut, didapatkan gambaran umum mengenai penggunaan \textit{spreadsheet} secara umum. Menurut hasil penelitian tersebut beberapa fitur yang sering digunakan oleh pengguna \textit{spreadsheet} secara terurut dari yang paling sering digunakan adalah sebagai berikut,

\begin{enumerate}
    \item Menghitung fungsi matematika dasar (tambah, kurang, kali, bagi, dan lainnya)
    \item Mengelola \textit{worksheet} dan \textit{workbook} (menambahkan, menghapus, merubah nama, dan lainnya)
    \item Melakukan perubahan format dasar (menebalkan, memberi garis bawah, format angkat, dan lainnya)
    \item Melakukan pengurutan data, penghitungan subtotal, serta meringkas data
    \item Menggunakan fitur \textit{cell addressing} baik absolut maupun relatif
    \item Penggunaan fungsi kondisi (IF, COUNTIF), fungsi logika (AND, OR), fungsi pencarian (VLOOKUP, HLOOKUP), menautkan \textit{workbook} lain, serta fungsi pembulatan (ROUND, CEILING, FLOOR)
\end{enumerate}

Penggunaan \textit{spreadsheet} sangat bergantung kepada domain bisnis atau organisasi yang menggunakan. Pada bisnis yang berorientasi komersial, \textit{spreadsheet} dapat digunakan sebagai alat bantu perhitungan laba, pengeluaran, investasi, dan pajak. Pada organisasi-organisasi non komersial, \textit{spreadsheet} dapat digunakan sebagai salah satu bentuk basis data yang menangani penyimpanan, pengelolaan, dan pengumpulan data yang mudah dan cepat.

\section{Kesalahan dalam Penggunaan \textit{Spreadsheet}}
\subsection{Kualitas Data}
Kualitas Data (\textit{Data Quality}) adalah tingkat kemampuan data untuk memenuhi kebutuhan penggunaannya (\textit{usage requirement}) sehingga data dapat digunakan dengan baik \citep{Khatri2010}. Dimensi yang ada pada kualitas data dapat dibagi menjadi:

    \begin{enumerate}
        \item Akurasi, merujuk kepada tingkat kebenaran dari data.
        \item Aktualitas, menunjukan bahwa data yang dicatat merupakan data terbaru.
        \item Kelengkapan, menunjukkan bahwa nilai-nilai yang diperlukan tercatat (tidak hilang).
        \item Kredibilitas, menunjukkan kepercayaan terhadap sumber serta isinya.
    \end{enumerate}

Tingkatan nilai untuk dimensi tersebut dapat berbeda pada setiap kasus yang ada. Contohnya, akurasi 85\% untuk data nama dan alamat dokter merupakan nilai yang cukup baik bagi perusahaan asuransi yang menargetkan doktor sebagai konsumen potensial, namun tidak cukup baik untuk perusahaan obat yang ingin melakukan \textit{recall} terhadap obat yang terdistribusi. Kualitas data yang buruk dapat menyebabkan akibat yang fatal dalam bisnis baik secara operasional maupun strategis. 

\subsection{Tingkat Kesalahan dalam Penggunaan \textit{Spreadsheet}}
Penelitian telah dilakukan oleh Panko \citep{Panko1998} untuk mengetahui banyaknya kesalahan yang terjadi pada pengembangan \textit{spreadsheet} terutama pada sektor bisnis. Dari penelitian ini, didapatkan bahwa 20\% hingga 40\% \textit{spreadsheet} mengandung kesalahan. Pada kasus tertentu, bahkan ditemukan 90\% \textit{spreadsheet} yang diteliti memiliki kesalahan \citep{Journal1996}. 

Penelitian yang dilakukan oleh Panko juga menemukan 88\% dari 113 \textit{spreadsheet} yang diaudit melalui 7 lebih studi yang diteliti. Beberapa hasil yang telah di rangkum oleh penelitian tersebut mengunakan \textit{spreadsheet} yang digunakan di dunia nyata dapat dilihat pada Tabel \ref{StudiKesalahan}.
  \begin{small}
  \begin{longtable}{ | L{3cm} | R{2cm} | R{2cm} | R{2cm} | L{3cm} | }
    \caption{Studi terhadap Kesalahan pada \textit{Spreadsheet}}
    \label{StudiKesalahan}\\ \hline
    \centering\bfseries{Pembuat} & \centering\bfseries{Jumlah yang Diaudit} & \centering\bfseries{Rata-rata Sel} & \centering\bfseries{Persentase Error} & \centering\bfseries{Keterangan Kesalahan} \tabularnewline \hline
    \endfirsthead
    \hline
    \centering\bfseries{Pembuat} & \centering\bfseries{Jumlah yang Diaudit} & \centering\bfseries{Rata-rata Sel} & \centering\bfseries{Persentase Error} & \centering\bfseries{Keterangan Kesalahan} \tabularnewline \hline
    \endhead
    Butler (1992) & 273 & - & 11\% & Kesalahan perhitungan pada pajak\\ \hline
    Dent (1994) & Tidak diketahui & - & 30\% & Menggunakan angka yang ditulis manuals yang mengakibatkan perhitungan berikutnya salah\\ \hline
    Hicks (1995) & 1 & 3856 & 100\% & Kesalahan interpretasi pada data \\ \hline
    Coopers \& Lybrand (1997) & 23 & 150+ & 91\% & Kesalahan perhitungan yang meleset hingga 5\% \\ \hline
    Lukasic (1998) & 2 & 2270 - 7027 & 100\% & Kesalahan akibat melebih-lebihkan perhitungan hingga 16\%\\ \hline
    Clermont, Hanin, \& Mittermeier (2002) & 3 & - & 100\% & Kesalahan akibat perhitungan sel kosong\\ \hline
    Lawrence and Lee (2004) & 30 & 2182 & 100\%  & Kesalahan perhitungan dan formula\\ \hline
    Powell, Lawson, and Baker (2007) & 25 & - & 64\%  & Kesalahan perhitungan dan formula \\ \hline
    Powell, Baker \& Lawson (2007) & 50 & - & 86\%  & Kesalahan perhitungan dan formula \\ \hline
  \end{longtable}
  \end{small}
Dari kumpulan data di atas, dapat dilihat bahwa di dalam pembentukan \textit{spreadsheet} pada bidang bisnis, tidak mungkin terlepas dari kesalahan. Dengan tingginya tingkat kesalahan ini, bisnis dapat mengalami kerugian secara material maupun moral yang cukup besar \citep{EUSPRIGHorrorStories}. Hal ini mengindikasikan bahwa tingginya tingkat kesalahan harus dapat diselesaikan agar tidak terjadi kerugian di dalam penggunaan \textit{spreadsheet} terutama dalam bisnis.

\subsection{Tipe Kesalahan dalam Penggunaan \textit{Spreadsheet}} \label{KesalahanPenggunaan}
Tingkat fleksibilitas \textit{spreadsheet} yang tinggi memberikan keleluasaan kepada penggunanya untuk melakukan banyak manipulasi dan pengelolaan data. Tingginya fleksibilitas ini dapat berakibat mudahnya \textit{human error} terjadi pada saat penggunaan \textit{spreadsheet} yang menyebabkan terjadinya kesalahan-kesalahan pada data. Tipe-tipe kesalahan pada \textit{spreadsheet} dapat dibagi menjadi dua jenis tipe kesalahan yakni kesalahan kuantitatif, dan kesalahan kualitatif \citep{Panko1998}. 

    \subsubsection{Kesalahan Kualitatif}
    Kesalahan kualitatif merupakan kesalahan yang berhubungan dengan kualitas \textit{spreadsheet} tersebut lebih menitikberatkan pada kebiasaan dan prosedur yang salah di dalam pembuatan \textit{spreadsheet}. Beberapa kesalahan yang dapat diklasifikasikan sebagai kesalahan kualitatif adalah \citep{Powell2009}:

    \begin{enumerate}
        \item Melakukan \textit{hard-code} pada suatu angka di dalam formula
        \item Menggunakan formula yang panjang dalam perhitungan
        \item Susunan data yang tidak direncanakan dengan baik
        \item Tidak adanya dokumentasi mengenai \textit{spreadsheet} yang dibuat
    \end{enumerate}

    Kesalahan ini tidak langsung mengakibatkan nilai hasil keluaran yang salah namun menurunkan kualitas dari \textit{spreadsheet} tersebut \citep{Rajalingham2001}. Di samping itu, kesalahan kualitatif dapat menyebabkan kesalahan kuantitatif terutama pada saat penggunaan fungsi analisis \textit{what-if} pada \textit{spreadsheet} \citep{Panko1998}.

    \subsubsection{Kesalahan Kuantitatif}
    Kesalahan ini mengakibatkan \textit{spreadsheet} mengeluarkan hasil dan nilai yang salah di dalam operasi perhitungannya. Kesalahan jenis ini dapat dibagi menjadi empat jenis kesalahan \citep{Howe2006}, yakni:

    % \begin{enumerate}
    %     \item Kesalahan mekanikal (\textit{mechanical error}) yang biasanya terjadi akibat kesalahan pengetikan angka atau rujukan sel yang salah pada suatu formula
    %     \item Kesalahan logika (\textit{logical error}) yang terjadi pada pembuatan formula yang salah atau penggunaan fungsi yang tidak tepat
    %     \item Kesalahan akibat kelalaian pada interpretasi situasi atau spesifikasi yang diberikan sehingga \textit{spreadsheet} yang dihasilkan tidak sesuai dengan domain permasalahan yang ada atau \textit{ommision error} \citep{Powell2009}
    % \end{enumerate}

    \begin{enumerate}
        \item Kesalahan \textit{clerical} dan non-material merupakan kesalahan yang tidak berpengaruh pada hasil dan perhitungan, namun tetap merupakan kesalahan. Contohnya adalah kesalahan ejaan.
        \item \textit{Rule violations} merupakan kesalahan yang melanggar aturan yang telah ditetapkan. Contohnya nilai seseorang tidak boleh melebihi angka 100.
        \item Kesalahan \textit{data-entry} merupakan kesalahan yang terjadi saat memasukan data. Contohnya data yang seharusnya angka, namun dimasukkan berupa teks.
        \item Kesalahan \textit{formula}  merupakan kesalahan yang terjadi akibat penggunaan \textit{formula}. Kesalahan yang terjadi bisa berupa referensi tidak sesuai, penggunaan konstanta yang salah, atau kesalahan logika.
    \end{enumerate}

    Pada penelitian yang dilakukan oleh Howe, diketahui bahwa faktor yang berpengaruh terhadap ditemukannya kesalahan \textit{formula} adalah umur, indeks prestasi, serta pengalaman pemrograman. Sedangkan kesalahan lain, tidak dipengaruhi oleh faktor-faktor eksternal yang diuji. \citep{Howe2006} Hal ini juga terlihat pada penelitian yang dilakuakn oleh Bishop, dimana profesional lebih baik dalam mendeteksi kesalahan \textit{formula} hingga lebih baik 16\% dibandingkan orang kebanyakan. \citep{bishop2008empirical} Sehingga, dapat terlihat bahwa dengan pelatihan yang baik, kesalahan formula dapat berkurang, namun kesalahan lain tidak berkurang cukup signifikan.

% \subsection{Penanganan Kesalahan pada \textit{Spreadsheet}} \label{PenangananKesalahan}
% Untuk mengatasi kesalahan yang dijelaskan pada subbab sebelumnya, menurut penelitian yang dilakukan oleh Panko \citep{Panko1998}, dijabarkan beberapa metode untuk menangani dan mengurangi kesalahan yang sering terjadi. Beberapa metode yang dapat digunakan yakni:

%     \begin{enumerate}
%         \item Membangun \textit{preliminary design} sebelum pembuatan \textit{spreadsheet} agar terdapat perencanaan yang baik di dalam pembangunan data di dalam \textit{spreadsheet}
%         \item Melakukan proteksi terhadap sel yang tidak boleh diubah.
%         \item Melakukan pengecekan terhadap semua rumus dan formula yang dimasukan bahkan hingga rumus yang cukup sederhana dengan cara melakukan pengecekan manual.
%         \item Membuat dokumentasi untuk \textit{spreadsheet} yang dibuat.
%         \item Tidak menekan pembuat \textit{spreadsheet} terhadap kesalahan yang dibuat dengan memberikan hukuman. Kesalahan yang terjadi pada \textit{spreadsheet} umumnya masih berada pada batas normal \textit{human error} sehingga memberikan hukuman akan membuat rasa takut dalam melaporkan kesalahan.
%         \item Melakukan inspeksi terhadap formula, rumus, dan kode yang dibuat baik oleh individual maupun secara berkelompok.
%     \end{enumerate}

\section{Data pada \textit{Spreadsheet}} \label{JenisSpreadsheet}

\subsection{Tipe Struktur Data pada \textit{Spreadsheet}}
Pada penelitian yang dilakukan oleh Chen dan Cafarella, \textit{spreadsheet} dapat dibagi menjadi 2 jenis yakni; \textit{data frame} dan \textit{non-data frame}. \textit{Data frame} merupakan tipe \textit{spreadsheet} yang terdiri dari 2 komponen utama: area nilai dan area atribut atau metadata (biasanya berada di atas dan atau di kiri area nilai). \textit{Non-data frame} adalah tipe \textit{spreadsheet} selain tipe \textit{data frame} yang telah didefinisikan sebelumnya. Tipe \textit{non-data frame} dapat dibagi menjadi beberapa jenis yakni:

    \begin{enumerate}
        \item Relasi merupakan tipe \textit{spreadsheet} yang dapat langsung diubah ke model relasional.
        \item Formulir merupakan \textit{spreadsheet} yang tidak ditujukan sebagai penyimpanan dan didesain untuk diisi oleh manusia.
        \item Diagram yang digunakan sebagai visualisasi data, biasanya berisi banyak data tanpa skema informasi yang detil.
        \item Daftar atau \textit{List} merupakan catatan sejumlah nama atau hal (tentang kata-kata, nama orang, barang, dan sebagainya) yang disusun berderet dari atas ke bawah \citep{pusat1991kamus}.
        \item Jadwal merupakan \textit{spreadsheet} digunakan sebagai pembuatan dan pengelolaan jadwal.
        \item Silabus merupakan kerangka unsur kursus pendidikan, disajikan dalam aturan yang logis, atau dalam tingkat kesulitan yang makin meningkat \citep{pusat1991kamus}.
        \item \textit{Scorecard} yakni suatu alat manajemen yang biasanya berguna untuk membantu manajer melacak aktivitas yang dilakukan oleh stafnya.
    \end{enumerate}

Penelitian ini menggunakan sampel 200 \textit{spreadsheet} yang dilabeli oleh ahli dan didapatkan bahwa 50.5\% \textit{spreadsheet} merupakan tipe \textit{data frame}, dimana 32.5\% memiliki label atribut dibagian atas atau bawah. Sedangkan 49.5\% \textit{spreadsheet} bertipe \textit{non-data frame} terdiri dari 22.0\% relasi, 10.5\% formulir, 3.5\% diagram, 3\% berupa \textit{list}, dan 10.5\% lainnya \citep{Chen2013}.

\subsection{Pengolahan Data pada \textit{Spreadsheet}}
    \textit{Extract-Transform-Load} (ETL) adalah proses yang digunakan sebagai metode integrasi data dari beberapa sumber dan aplikasi. ETL biasanya digunakan pada saat melakukan proses \textit{data warehouse} dimana data dari sumber eksternal diambil, lalu ditransformasikan ke bentuk yang sesuai dengan kebutuhan (di dalam prosesnya bisa terkadung pengecekan kualitas), dan memasukannya ke dalam basisdata yang telah ditentukan \citep{Bansal2014}. Terdapat tiga fase pada proses ETL yakni:

    \begin{enumerate}
        \item \textit{Extract}, fase pertama ini adalah proses yang melakukan ekstraksi data dari sumber yang dipilih. Data biasanya tersedia dalam format \textit{flat file} seperti csv, xls, dan txt atau melalui klien RESTful.
        \item \textit{Transform}, pada fase ini data dibersihkan agar sesuai dengan skema tujuan. Beberapa cara untuk mentransformasikan data adalah dengan menormalisasi data, menghapus duplikasi, melakukan pengecekan terhadap batasan-batasan, melakukan \textit{filtering}, melakukan pengurutan dan pengelompokan, atau fungsi-fungsi lain yang didefinisikan.
        \item \textit{Load}, pada fase ini data yang telah ditransformasikan dimasukan ke dalam \textit{data mart} atau \textit{data warehouse} yang ditentukan.
    \end{enumerate}

    \subsubsection{Metode ETL pada \textit{Spreadsheet}} \label{metodepencarian}
    Pada penelitian yang dilakukan oleh Chen, pengolahan data pada sebuah \textit{spreadsheet} bertipe \textit{data frame} dapat dibagi menjadi tiga proses yakni: \textit{frame finder}, \textit{hierarchy extractor}, dan \textit{tuple builder}. Proses \textit{frame finder} dilakukan dengan cara mengidentifikasi \textit{data frame} serta mencari lokasi dari atribut dan nilai. Pada proses \textit{hierarchy extractor}, atribut yang ada pada \textit{data frame} yang ditemukan dicari hirarkinya setelah itu proses \textit{tuple builder} membentuk \textit{tuple} relasional untuk setiap nilai yang ada. Proses ini tidak membedakan \textit{spreadsheet} tipe \textit{data frame} atau bukan, sehingga diasumsikan jika \textit{tuple} yang dihasilkan memiliki kualitas yang baik, dapat dikatakan bahwa \textit{spreadsheet} masukan bertipe \textit{data frame} dan sebaliknya \citep{Chen2013}.

        \paragraph{\textit{Frame Finder}}
        Tujuan dari proses \textit{frame finder} \citep{Chen2013} adalah mengidentifikasi wilayah nilai dan wilayah atribut yang dapat berupa \textit{left attribute} maupun \textit{top attribute}. Untuk mensimplifikasi permasalahan, Chen menganggap bahwa \textit{data frame} tidak akan berada sejajar secara horisontal, namun hanya secara vertikal. Sehingga proses ini dapat disimplifikasi menjadi labeling terhadap baris per baris. Label yang akan diberikan adalah \textit{title}, \textit{header}, \textit{data}, dan \textit{footnote}.

        Pelabelan dapat dilakukan dengan algoritma \textit{conditional random field} (CRF) karena terdapat keterkaitan antara satu baris terhadap baris yang lain dalam penggunaan baris. Contohnya, jika baris telah teridentifikasi sebagai \textit{header}, maka besar kemungkinan bahwa baris selanjutnya adalah \textit{data} atau \textit{header}. CRF memiliki kemampuan untuk melakukan \textit{machine learning} yang memperhitungkan label pada elemen sebelumnya.

        \paragraph{\textit{Hierarcy Extractor}}
        Proses \textit{hierarcy extractor} \citep{Chen2013} bertujuan untuk mendapatkan hirarki dari atribut-atribut yang ada. Masukan dari proses ini adalah \textit{data frame} dengan \textit{top attribute} dan \textit{left attribute} dan keluarannya berupa hirarki untuk masing-masing atribut atas dan kiri tersebut. Proses ini dapat dilakukan melalui dua algoritma: \textit{classification} dan \textit{enforced-tree classification}.

        \textit{Classification} dilakukan dengan cara berbeda untuk \textit{left attribute} dan \textit{top attribute}. Pada \textit{left attribute}, klasifikasi dapat dilakukan dengan dua cara: pengecekan terhadap \textit{formatting} pada sebuah sel dan kedekatan sel secara geometris. Semakin mirip \textit{formatting} sebuah sel, maka semakin mungkin bahwa kedua sel bukan merupakan pasangan \textit{parent} dan \textit{child} dan semakin dekat sel secara geometris, kemungkinan kedua sel meruapakan  pasangan \textit{parent} dan \textit{child} semakin besar. Sedangkan pada \textit{top attribute} dapat dilakukan pengecekan posisi antar baris atribut bagian atas.

        Kelemahan pada metode klasifikasi sebelumnya adalah terdapat kemungkinan tidak dapat dibentuk pohon dari hasil klasifikasi. \textit{Enforced-tree classification} mencoba untuk menyelesaikan permasalahan ini dengan dua langkah tambahan yakni: memastikan bahwa suatu atribut hanya dapat memiliki satu \textit{parent} dimana yang terpilih menjadi \textit{parent} adalah atribut dengan probabilitas tertinggi, dan memastikan bahwa tidak ada \textit{cycle} yang terbentuk dengan cara menghapus keterhubungan dengan nilai probabilitas terkecil. Klasifikasi ini tetap menggunakan metode klasifikasi fitur yang dilakukan pada algoritma \textit{classification}.

        \paragraph{\textit{Tuple Builder}}
        Proses ini dilakukan dengan cara mengiterasi setiap \textit{value (v)} dan mencari atribut akar pada atribut bagian kiri dan atas dari \textit{v}. Setelah dibentuk \textit{relational tuple} untuk nilai \textit{v} dengan atribut bagian kiri dan atas tersebut. Tingkat akurasi dari proses ini sangat bergantung dari dua proses sebelumnya.

% \section{\textit{Data Governance}}
% Penggunaan \textit{spreadsheet} pada bisnis tidak terlepas dari adanya penyimpanan dan pengelolaan informasi. 

\section{Studi dan Penelitian Terkait}
    \subsection{\textit{Senbazuru: A Prototype Spreadsheet Database Management System}}
    Senbazuru \citep{Chen2013-2} merupakan prototipe yang dikembangkan dengan tujuan untuk mempermudah pencarian, pengaksesan, pengubahan, dan melakukan \textit{query} terhadap \textit{spreadsheet}. Pengembangan ini ingin menyelesaikan permasalahan dimana data \textit{spreadsheet} sangat tersebar diberbagai tempat sehingga untuk mendapatkan informasi yang diinginkan atau membandingkan antar informasi di dalam beberapa \textit{spreadsheet} sangatlah sulit. 

    Di dalam pengembangan prototipe ini, kesulitan teknikal yang harus dihadapi adalah proses ekstraksi dan perbaikan data. Di dalam melakukan ekstraksi data, harus dilakukan beberapa proses berikut: mendeteksi mana atribut dan nilai, mengidentifikasi hirarki atribut, membentuk \textit{relational tuple}, dan membentuk \textit{tuple} tersebut menjadi tabel relasional. Dari hasil dari ekstrasi ini, masih sangat mungkin memiliki kesalahan sehingga proses perbaikan diperlukan di dalam pembentukan tabel relasional ini. Proses perbaikan pada protitipe ini dilakukan manual dengan bantuan pengguna.

    \subsubsection{Arsitektur Sistem}

    \begin{figure}[!htb]
        \centering
        \includegraphics[width=0.7\textwidth]{resources/chapter-2-senbazuru-architecture.png}
        \caption{Arsitektur Sistem Senbazuru \citep{Chen2013-2}}
        \label{ArsiSisSenbazuru}
    \end{figure}

    \textit{Spreadsheet Database Management System} (SSDBMS) yang dikembangkan pada prototipe ini memiliki tiga proses utama yakni: \textit{search}, \textit{extract}, dan \textit{query}. Proses pencarian dilakukan terhadap repositori \textit{spreadsheet} yang ada di internet, lalu data \textit{spreadsheet} tersebut diekstraksi dan dijadikan tabel relasional, dan setelah itu pengguna dapat melakukan \textit{query} terhadap tabel yang telah dibentuk. Gambaran arsitektur Senbazuru dapat dilihat pada Gambar \ref{ArsiSisSenbazuru}. Arsitektur ini terbagi menjadi tiga modul utama yakni Search, Extract, dan Query.

        \paragraph{\textit{Search}}
        Komponen pencarian ini memudahkan pengguna untuk menemukan dataset yang tepat menggunakan bantuan internet. Saat prototipe ini dikembangkan, Senbazuru telah mengindeks 1800 \textit{spreadsheet} yang didapatkan dari U.S. Census Bureau. Pengindeksan menggunakan bantuan \textit{library} Python yakni xlrd untuk mengekstraksi teks dari sel lalu menggunakan Apache Lucene untuk melakukan indeks pada teks. Pencarian menggunakan metode \textit{term frequency–inverse document frequency} (TF-IDF) untuk mendapatkan relevansi dokumen.
        
        \paragraph{\textit{Extract}}
        Proses ekstraksi data pada \textit{spreadsheet} dilakukan melalui empat tahapan yakni:

        \begin{enumerate}
            \item \textit{Frame Finder}\\
            Tahap ini dilakukan untuk mencari \textit{frame} pada \textit{spreadsheet} bertipe \textit{data frame}. Dengan menggunakan algoritma \textit{conditional random field} (CRF) untuk memberikan label pada setiap baris yang tidak kosong pada \textit{spreadsheet}. Tahap ini akan menghasilkan \textit{data frame} yang selanjutnya akan digunakan pada tahap selanjutnya, baris lain yang dianggap bukan \textit{data frame} akan diabaikan.

            \item \textit{Hierarchy Extractor}\\            
            Tahap selanjutnya adalah ekstraksi hirarki pada wilayah atribut dari \textit{data frame} yang ditemukan. Pada setiap atribut, akan dicari atribut mana yang dideskripsikan oleh atribut lainnya dan seterusnya hingga terbentuk hirarki dari atribut-atribut yang ada. Kesalahan pada pembentukan hirarki sangat mungkin terjadi sehingga pengguna akan diberikan kemampuan untuk memperbaiki hirarki yang salah pada bagian \textit{repair interface}. Setelah perbaikan dilakukan oleh pengguna, Senbazuru akan menjalankan kembali CRF untuk melakukan pembelajaran terhadap label baru yang diberikan.

            \item \textit{Tuple Builder}\\            
            Bagian ini melakukan pembentukan \textit{tuple} antara wilayah nilai dan wilayah atribut yang sesuai.

            \item \textit{Relation Constructor}\\
            Tahap ini melakukan transalasi dari \textit{tuple} yang terbentuk menjadi tabel relasional dengan cara membentuk kluster terhadap atribut yang satu jenis. Contohnya, terdapat atribut \textit{Male}, \textit{total}, dan \textit{Female}, ketiga atribut tersebut memiliki jenis yang sama sehingga harus digabungkan menjadi satu kolom yakni \textit{gender}. Pada Senbazuru, teknik pengklusteran ini menggunakan bantuan koleksi skema dari Freebase dan YAGO.
        \end{enumerate}

        \paragraph{\textit{Query}}
        Setelah proses sebelumnya selesai, maka pengguna dapat memasukan perintah relasional terhadap data \textit{spreadsheet} yang telah diubah menjadi tabel relasional. Pada prototipe yang dikembangkan, perintah yang diimplementasikan adalah \textit{join} dan \textit{select}.

    \subsubsection{Hasil Penelitian}
    Senbazuru merupakan prototipe untuk manajemen basis data berbasis \textit{spreadsheet} yang dapat melakukan pencarian data pada internet melalui kata kunci yang diberikan. Prototipe ini berhasil melakukan ekstraksi data secara otomatis walaupun tidak terlepas dari kesalahan. Kesalahan yang terjadi masih harus seringkali dilakukan perbaikan secara manual. Namun dengan penggunaan algoritma CRF, protitipe dapat mengurangi kesalahan yang terjadi. Prototipe ini juga ditujukan sebagai demo kepada peserta konferensi VLDB dan diharapkan dapat menarik perhatian komunitas basis data.

\subsection{\textit{Spreadsheet As a Relational Database Engine}}
Penelitian \citep{Tyszkiewicz2010} pernah dilakukan terhadap pembuatan \textit{spreadsheet} menjadi mesin basis data relasional. Penelitian ini dilatarbelakangi dengan tingginya penggunaan \textit{spreadsheet} pada banyak bidang dan kurangnya kualitas data yang ada di dalamnya yang dapat menyebabkan kesalahan-kesalahan terjadi pada perhitungan dan prediksi. Solusi yang dipaparkan pada penelitian ini adalah dengan menggabungkan \textit{spreadsheet} dan \textit{database engine} dengan menggunakan formula sebagai ganti dari \textit{SQL query}.

    \subsubsection{Cara Kerja Sistem}

    Pada implementasinya, sebuah \textit{workbook} akan memiliki satu \textit{worksheet} untuk setiap tabel data dan satu \textit{worksheet} untuk setiap \textit{view} pada basis data. Dengan menggunakan \textit{external compiler} yang menerima masukan berupa SQL yang akan mengubahnya ke dalam bentuk \textit{spreadsheet}. Program tersebut akan mengubah SQL menjadi beberapa formula yang diterima oleh \textit{spreadsheet} tersebut.

    Terdapat dua bagian utama pada \textit{spreadsheet} hasil implementasi yakni: tabel data dan bagian \textit{view}. Bagian tabel data adalah tempat pengguna untuk memasukkan, mengubah, serta menghapus data. Secara teori, bagian tabel data tidak memiliki formula, namun di dalam implementasinya mengikuti implementasi SQL dimana perlu dilakukan validasi data dan verifikasi terhadap \textit{primary key}, \textit{foreign key}, dan batasan lain yang ada diperintah \textit{create table}.

    Bagian \textit{view worksheet} tidak dapat diubah oleh pengguna dan berisikan formula-formula yang independen terhadap data yang dimasukkan oleh pengguna. Di samping itu, bagian \textit{view} berisikan kolom-kolom yang berguna sebagai \textit{intermediate result} yang selanjutnya akan digunakan oleh formula lain. Pada awalnya sel-sel akan berisikan "" (\textit{string} kosong) yang merepresentasikan sel yang belum digunakan.

    \subsubsection{Hasil Penelitian}

    Formula pada excel dilakukan menggunakan \textit{linear scan} dan hal ini dapat mengurangi kinerja. Beberapa cara untuk meningkatkan kinerja adalah mengeksploitasi \textit{lazy evaluation} dari \textit{if statement}, mengkomputasi hanya beberapa sel tetangga yang berkaitan, serta menggunakan file atau \textit{workbook} lain untuk membagi \textit{query} dan membangkitkannya ketika dibutuhkan.

    Pada tes kinerja yang dilakukan pada penelitian ini dapat disimpulkan bahwa untuk operasi dasar dan \textit{query} sederhana penggunaan \textit{spreadsheet} dapat digunakan dengan baik dengan waktu yang cukup cepat. Tingkat efektifitas penggunaan pada arsitektur ini cukup rendah, namun hasil tes tersebut menunjukan bahwa arsitektur ini memiliki potensial yang dapat dikembangkan.
\chapter{Analisis Masalah dan Perancangan Solusi Visualisasi \textit{distributed tracing}}

%Pada bab ini diuraikan analisis persoalan pengumpulan data pada \textit{spreadsheet} yang telah diuraikan pada Bab I. Hasil dari bab ini digunakan untuk merancang kakas yang akan diimplementasikan seperti yang dijelaskan pada Bab IV.
%Berdasarkan 


\section{Analisis Masalah}

 Pada masa sebelum adopsi arsitektur Microservice dan kebanyakan dari aplikasi masih menggunakan arsitektur Monolith, proses seperti \textit{debugging} adalah hal yang sederhana sebab jika terdapat suatu \textit{error} akan mudah untuk ditelusuri dari mana asal \textit{error} tersebut sebab hanya ada satu aplikasi yang digunakan. Hal tersebut tidak berlaku jika aplikasi menggunakan model Sistem Terdistribusi, salah satu contohnya adalah Microservice. Sifat dari Microservice yang melakukan \textit{decoupling} aplikasi menjadi bagian yang lebih kecil membuat proses \textit{debugging} menjadi tidak mudah sebab untuk mencari penyebab \textit{error} aplikasi yang terdistribusi, kita harus mengetahui terlebih dahulu sumber dari \textit{error} tersebut. Kompleksitas akan bertambah dalam proses debugging jika ternyata ditemukan bahwa sautu \textit{error} pada sebuah \textit{service} bukanlah akar atau penyebab utama dari \textit{error} tersebut melainkan suatu \textit{service} lainnya. Kompleksitas akan bertambah jika metode \textit{debugging} yang digunakan mengharuskan \textit{developer} yang menangani \textit{error} tersebut harus menelusuri satu per satu \textit{service} yang terdampak sampai menemukan akar dari masalahnya.

Seiring dengan meningkatnya penggunaan arsitektur, mengingkat pula kebutuhan bagi para \textit{developer} untuk dapat dengan segera mengetahui sumber dari permasalahan jika terjadi \textit{error} pada sistem.

Dari pemaparan mengenai masih kurangnya kakas visualisasi \textit{tracing} yang bersifat \textit{open source}
. Berdasarkan studi literatur mengenai \textit{observability} untuk Sistem Terdistribusi pada \ref{bab2-observability}, terdapat 

%Isu visualisasi 
%Overhead


\section{Analisis Alternatif Solusi}

\section{Rancangan Solusi}

%Overhead disini


\chapter{Implementasi dan Pengujian}

Bab ini akan membahas seluruh proses implementasi yang dilakukan untuk menerapkan rancangan yang sudah didefinisikan sebelumnya. Selain itu, bab ini juga akan membahas pengujian yang dilakukan terhadap hasil implementasi yang mencakup hal-hal yang diuji, metode pengujian, dan hasil pengujian yang diperoleh. 


\section{Implementasi}
Implementasi sistem \textit{Performance Regression Analysis} (PRA) akan dibuat berdasarkan rancangan arsitektur seperti yang terdapat pada gambar \ref{arch-pra}. Komponen seperti \textit{library} instrumentasi, melainkan akan melakukan \textit{fork} dan modifikasi solusi \textit{Open Source} \textit{distributed tracing} dari Zipkin. Komponen yang akan dibuat dari awal sepenuhnya adalah \textit{User Interface} (UI) dan \textit{engine} dari sistem PRA yang akan melakukan komputasi utama sistem pendeteksian dan analisis regresi. Komponen \textit{engine} juga akan berfungsi sebagai API yang akan diakses oleh komponen \textit{User Interface}.



\subsection{Implementasi \textit{Performance Regression Analysis} \textit{engine}}

Sebelum melakukan implementasi solusi, penulis harus terlebih dahulu melakukan pemilihan kakas yang akan digunakan melakukan implementasinya. Secara umum, terdapat banyak bahasa pemrograman yang dapat melakukan implementasi solusi komponeasdn \textit{engine} sesuai dengan alur yang terdapat pada gambar \ref{alur-pra}. Namun, ada suatu kebutuhan penting untuk melakukan tes statistik Kolmogorov-Smirnov (K-S) yang tidak semua bahasa pemrograman memiliki dukungan \textit{library} untuk melakukannya. Terdapat dua bahasa pemrograman yang memiliki \textit{library} untuk melakukan tes K-S yaitu bahasa pemrograman \textbf{Go} dan \textbf{Python}. Bahasa \textbf{Go} memiliki library Gonum \footnote{\url{gonum.org}} yang memiliki implementasi tes K-S dalam fungsi \texttt{KolmogorovSmirnov}, sementara bahasa \textbf{Python} memiliki library SciPy \footnote{\url{scipy.org}} yang memiliki implementasi tes K-S dalam fungsi \texttt{scipy.stat.ks\textunderscore 2samp}.

Dari kedua bahasa tersebut, penulis memilih bahasa \textbf{Python} untuk melakukan implementasi solusi setelah berhasil melakukan \textit{parsing} data \textit{trace} Zipkin dengan pendekatan pemrograman berorientasi objek yang didasarkan pada \textit{source code} yang dimiliki oleh aplikasi \textit{User Interface} milik Zipkin yang diimplementasikan dengan bahasa \textbf{Javascript}. 

Implementasi \textit{engine} akan terbagi menjadi beberapa modul seperti yang terlihat pada tabel \ref{engine-module}.

\begin{small}
	\begin{longtable}{ | p{1cm} | p{3cm} | p{10cm} | }
		\caption{Tabel pembagian modul komponen \textit{engine}}
		\label{engine-module}                                                           
		\\ \hline
		\centering\bfseries{ID} & \centering\bfseries{Nama Modul} & \centering\bfseries{Deskripsi} \tabularnewline \hline
		\endfirsthead
		EM-1 & zipkin (Pengambilan Data) & Modul ini bertanggung jawab untuk mengambil data dari API Zipkin yang memiliki data hasil \textit{trace} dari aplikasi. \\ \hline
		EM-2 & transform (Transformasi Data) & Modul ini bertanggung jawab untuk melakukan transformasi dari data \textit{trace} mentah yang diambil dari API Zipkin menjadi bentuk-bentuk model yang akan digunakan untuk komputasi di tahap selanjutnya seperti sampel data \textit{latency}, dan model data \textit{Critical Path}. Semua model akan disimpan dalam data berbentuk JSON. \\ \hline
		EM-3 & storage (Penyimpanan Data) & Modul ini bertanggung jawab untuk menyimpan model hasil transformasi \textit{baseline} dari modul EM-2 ke \textit{storage} untuk digunakan kembali pada fase \textit{Real-time Analysis}. Komponen \textit{storage} yang akan digunakan adalah Redis. Alasan utama pemilihan Redis sebagai komponen \textit{storage} adalah Redis telah memiliki modul penyimpanan data dalam bentuk JSON dan memiliki kinerja yang tinggi sebab data disimpan secara \textit{in-memory}. \\ \hline
		EM-4 & statistic (Perhitungan Statistik) & Modul ini bertanggung jawab untuk melakukan komputasi perhitungan statistik yang mencakup pendeteksian regresi dengan menghitung koefisien Kolmogorov-Smirnov seperti yang telah dijelaskan pada subbab \ref{approach-cumulative} menggunakan fungsi yang disediakan oleh library SciPy. \\ \hline
		EM-5 & critical\textunderscore path (Analisis \textit{Critical Path}) & Modul ini bertanggung jawab untuk melakukan analisis \textit{Critical Path} yang bertujuan untuk mencari penyebab regresi dengan mencari \textit{Critical Path} dari tiap \textit{service} dan melihat perbandingan \textit{latency} dari operasi tersebut dengan \textit{latency} yang telah direkam sebelumnya pada fase \textit{Baseline Loading}. Operasi yang selisih \textit{latency}-nya melebihi \textit{threshold} diduga kuat merupakan penyebab utama dari regresi yang terjadi. \\ \hline
		EM-6 & scheduling (\textit{Scheduling}) & Pada modul ini akan diimplentasikan \textit{job} atau pekerjaan utama yang akan digunakan untuk melakukan analisis regresi dengan menggunakan fungsi-fungsi yang telah diimplementasikan pada modul-modul sebelumnya dan juga bertanggung jawab melakukan penjadwalan \textit{job} tertentu selama interval yang ditentukan.  \\ \hline
	\end{longtable}
\end{small}

Aplikasi \textit{engine} akan dibuat sebagai REST API dan akan diimplementasikan menggunakan \textit{framework} FastAPI. Fungsionalitas \textit{engine} akan diekspos melalui REST API sehingga dapat digunakan baik oleh komponen UI maupun langsung melalui pemanggilan HTTP untuk keperluan pengujian. Beberapa \textit{endpoint} yang akan diimplementasikan terlihat pada tabel \ref{endpoints}.

\begin{small}
	\begin{longtable}{ | p{1cm} | p{2cm} | p{3.5cm} | p{7.5cm} | }
		\caption{Tabel \textit{endpoint} API \textit{engine}}
		\label{endpoints}                                                           
		\\ \hline
		\centering\bfseries{ID} & \centering\bfseries{Operasi HTTP} & \centering\bfseries{\textit{endpoint}} & \centering\bfseries{Deskripsi} \tabularnewline \hline
		\endfirsthead
		EP-1 & \centering{GET} & \centering\texttt{/state} & Data \textit{state} dari \textit{engine} yang akan berisikan informasi mengenai status pendeteksian regresi, pengecekan terakhir regresi, dan hasil analysis \textit{critical path} yang diduga menjadi penyebab regresi. \\ \hline
		EP-2 & \centering{POST} & \centering\texttt{/baseline} & Dengan \textit{endpoint} ini, \textit{user} dapat membuat \textit{engine} mengambil data \textit{baseline} baru dengan menyuplai informasi mengenai waktu mulai dan waktu selesai \textit{trace} di Zipkin beserta dengan batas banyaknya \textit{trace} yang akan diambil. \\ \hline
		EP-3 & \centering{DELETE} & \centering\texttt{/baseline} & \textit{Endpoint} ini akan menghapus informasi mengenai \textit{baseline} yang ada di data \textit{state}. \\ \hline	
		EP-4 & \centering{GET} & \centering\texttt{/analysis/range} & \textit{Endpoint} ini berfungsi untuk memicu \textit{engine} untuk melakukan analisis regresi dengan informasi waktu mulai dan waktu selesai trace yang disuplai oleh pengguna melalui \textit{query parameter}. API kemudian akan mengembalikan hasil analisis berupa apakah regresi terdeteksi beserta analisis \textit{critical path}. \\ \hline
		EP-5 & \centering{GET} & \centering\texttt{/analysis/realtime} & \textit{Endpoint} ini berfungsi untuk memicu \textit{engine} untuk melakukan analisis regresi dengan informasi waktu mulai dan waktu selesai trace yang telah ditentukan sebelumnya dan juga melakukan \textit{update} informasi state dengan hasil analisis regresi yang telah dilakukan. API kemudian akan mengembalikan hasil analisis berupa apakah regresi terdeteksi beserta analisis \textit{critical path}. \\ \hline				
	\end{longtable}
\end{small}

Secara umum, alur kerja endpoint EP-4 dan EP-5 serupa dengan perbedaan kecil waktu pengambilan \textit{trace} yang mana EP-4 dapat mengambil data \textit{trace} secara semaunya secara presisi sehingga endpoint ini akan digunakan untuk melakukan pengujian secara manual, sementara EP-5 digunakan untuk menyimulasikan tingkah laku dari \textit{engine} yang melakukan analisis dalam interval waktu tertentu tanpa harus menunggu interval waktunya.

Alur kerja analisis regresi pertama dimulai dengan mengambil data \textit{trace} yang akan dilakukan oleh module EM-1. Data \textit{trace} akan diambil dari API Zipkin dalam bentuk JSON dan akan dilakukan \textit{parsing} sehingga data akhir yang didapatkan mengandung informasi \textit{latency} dari semua \textit{span} yang terdaoat pada \textit{trace} tersebut. Setelah data \textit{trace} diambil dari API dan di-\textit{parse} oleh modul EM-1, selanjutnya data \textit{trace} tersebut akan ditransformasikan untuk diambil informasi \textit{latency}-nya oleh modul EM-2. Selanjutnya, dengan asumsi bahwa data \textit{baseline} telah didapatkan, \textit{engine} selanjutnya akan mengambil data \textit{latency} baseline yang akan dilakukan oleh modul EM-3. Setelah informasi \textit{latency} \textit{baseline} maupun \textit{realtime} sudah didapatkan, selanjutnya analisis pertama-tama dilakukan dengan menguji terjadinya regresi dengan fungsi statistik dari modul EM-4. Hasilnya adalah sebuah variabel \textit{boolean} yang melakukan tes K-S untuk menentukan apakah kedua sampel data tersebut berasal dari distribusi yang berbeda. Jika hasil tes menghasilkan kedua data berasal dari distribusi yang berbeda, dapat menjadi indikasi bahwa regresi telah terjadi. 

Jika terindikasi regresi telah terjadi, \textit{engine} selanjutnya akan melakukan analisis \textit{critical path}. Pertama-tama data \textit{trace} \textit{baseline} maupun \textit{realtime} akan ditransformasikan menjadi model \textit{critical path} oleh modul EM-2. Kemudian kedua model \textit{critical path} tersebut akan dibandingkan oleh modul EM-5 dan hasil akhirnya adalah data perbandingan operasi-operasi yang ada di data \textit{baseline} dan \textit{realtime}. Hasil akhir data pengecekan regresi dan analisis \textit{critical path} akan dikirim sebagai \textit{response} dalam bentuk JSON oleh API. Semua operasi dari alur kerja yang telah disebutkan di atas diimplementasikan dalam fungsi yang terdapat pada modul EM-6.

\subsubsection{Model dan Struktur Data}
Sistem PRA yang dibuat bergantung pada data \textit{trace} yang diperoleh dari Zipkin. Data tersebut tidak bisa begitu saja digunakan untuk menjalankan fungsi-fungsi dari \textit{engine} PRA sesuai yang terdapat pada gambar \ref{alur-pra} dan perlu ditransformasikan menjadi model dan struktur data yang lebih bermakna seperti yang telah disebutkan pada EM-2 di tabel \ref{engine-module}.

Dalam dokumentasinya, Zipkin menyediakan keterangan mengenai model data \textit{trace} yang digunakannya seperti yang terdapat pada \citep{zipkin-data}. Model data yang digunakan Zipkin terbagi menjadi dua yaitu \textit{span} dan \textit{trace}. \textit{Trace} sendiri merupakan serangkaian \textit{span} dengan id \textit{trace} sama yang merepresentasikan alur jalannya sebuah \textit{request} dalam sebuah \textit{service} yang telah terinstrumentasi, sementara \textit{span} merepresentasikan salah satu operasi yang terjadi sepanjang sebuah \textit{request}. Beberapa informasi yang terdapat pada data \textit{span} antara lain adalah informasi mengenai \textit{trace} dimana \textit{span} itu berada, metadata mengenai operasi yang direpresentasikan \textit{span}, durasi atau \textit{latency} dari operasi.

Sementara itu, untuk memenuhi fungsi-fungsi pada sistem PRA, data yang bersumber dari Zipkin perlu ditransformasikan menjadi struktur data yang sesuai dengan kebutuhan masing-masing tahap analisis. Dari rancangan alur sistem PRA, terdapat tiga tahap analisis yang akan dilakukan, yaitu tahap pendeteksian regresi dengan analisis statistik Kolmogorov-Smirnov, tahap analisis korelasi, dan tahap analisis \textit{critical path}.

Pada tahap pendeteksian regresi dengan tes statistik Kolmogorov-Smirnov (K-S), data yang diperlukan adalah data \textit{latency} atau durasi operasi dari \textit{span} yang dimiliki oleh Zipkin. Tes K-S akan membandingkan dua buah sampel data \textit{latency} dan akan menghasilkan hipotesis apakah kedua fungsi tersebut berada pada distribusi yang sama.

Pada tahap analisis \textit{critical path}, data yang dibutuhkan adalah pasangan \textit{key-value} dari nama operasi yang terrekam oleh Zipkin dan juga nilai \textit{latency} dari operasi tersebut. Data \textit{critical path} akan disimpan sebagai struktur data \textit{map} yang sesuai untuk menyimpan data berbentuk pasangan \textit{key-value}. Data ini akan didapatkan dari data \textit{trace} bawaan Zipkin yang sudah memiliki informasi mengenai \textit{critical path} dan juga nilai \textit{latency} nya masing-masing. 

\subsubsection{Algoritme penting}
Setelah pada subbab sebelumnya didefinisikan struktur data yang akan digunakan untuk merepresentasikan beberapa model seperti yang ada pada gambar \ref{alur-pra}, pada subbab ini akan didefinisikan beberapa algoritme penting dalam bentuk \textit{pseudocode} yang akan digunakan untuk melakukan berbagai operasi pada sistem PRA.

\begin{algorithm}[hbt!]
	\SetAlgorithmName{Algoritme}{Algoritme}
	
	\SetKwData{Left}{left}\SetKwData{This}{this}\SetKwData{Up}{up}
	\SetKwFunction{Union}{Union}\SetKwFunction{IsRegression}{IsRegression}
	\SetKwInOut{Input}{input}\SetKwInOut{Output}{output}
	\Input{Two array of float numbers representing latency: data1 and data2}
	\Output{Boolean indicating if regression has occured}
	\BlankLine
	
	
	\caption{Algoritme pendeteksian regresi}\label{alg:regression-detection}
\end{algorithm}
\begin{algorithm}[hbt!]
	\SetAlgorithmName{Algoritme}{Algoritme}
	
	
	
	
	\caption{Algoritme analisis regresi}\label{alg:regression-analysis}
\end{algorithm}

%\subsubsection{\textit{Deployment}}
%Agar aplikasi \textit{engine} dapat dijalankan di lingkungan Kubernetes, aplikasi harus dikemas dalam bentuk \textit{container} dan di-\textit{deploy} di Kubernetes menggunakan file \textit{manifest} berbentuk yaml. 

\subsubsection{Hasil implementasi}
Modul-modul pada tabel \ref{engine-module} masing-masing akan diimplementasikan sebagai \textit{package} pada bahasa \textit{Python} seperti yang terlihat pada gambar \ref{package}. Terdapat tujuh buah \textit{package}, enam buah \textit{package} dari modul dan satu buah \textit{package} \texttt{utils} yang berisi fungsi-fungsi untuk mendukung kerja fungsi di modul lainnya. Semua \textit{source code} hasil implementasi disimpan di Github \footnote{\url{https://github.com/masterraf21/pra\textunderscore engine}}.
\begin{figure}[!htb]
	\centering
	\includegraphics[width=0.35\textwidth]{resources/ch4/packages.png}
	\caption{Implementasi modul sebagai \textit{package}}
	\label{package}
\end{figure} 

Gambar \ref{api_docs} adalah hasil tangkapan layar dokumentasi API yang disediakan oleh FastAPI. 
\begin{figure}[!htb]
	\centering
	\includegraphics[width=0.5\textwidth]{resources/ch4/api_docs.png}
	\caption{Dokumentasi API}
	\label{api_docs}
\end{figure} 

Berikut adalah tangkapan layar dari hasil pemanggilan beberapa \textit{endpoint} yang dilakukan dengan aplikasi Postman \footnote{\url{https://www.postman.com/}}.
\begin{figure}[h!]
	\centering
	\includegraphics[width=0.75\textwidth]{resources/ch4/result_state.png}
	\caption{Hasil pemanggilan \textit{endpoint} \texttt{/state}}
	\label{api_state}
\end{figure} 
\begin{figure}[h!]
	\centering
	\includegraphics[width=0.75\textwidth]{resources/ch4/result_analysis_range.png}
	\caption{Hasil pemanggilan \textit{endpoint} \texttt{/analysis/range}}
	\label{api_analysis_range}
\end{figure} 
\begin{figure}[h!]
	\centering
	\includegraphics[width=0.75\textwidth]{resources/ch4/result_analysis_realtime.png}
	\caption{Hasil pemanggilan \textit{endpoint} \texttt{/analysis/realtime}}
	\label{api_analysis_realtime}
\end{figure} 
\pagebreak


\subsection{\textit{User Interface}}
Komponen lainnya yang akan diimplementasikan pada Tugas Akhir ini adalah komponen \textit{User Interface} (UI). Komponen ini akan menjadi antarmuka yang dapat digunakan untuk mengetahui \textit{state} dari sistem PRA. Fungsionalitas yang akan dibuat pada komponen UI akan dijabarkan pada tabel kebutuhan  fungsional \ref{ui-functional}.
\begin{small}
	\begin{longtable}{ | p{3cm} | p{9cm} | }
		\caption{Tabel kebutuhan fungsional komponen UI}
		\label{ui-functional}                                                           
		\\ \hline
		\centering\bfseries{ID} & \centering\bfseries{Deskripsi} \tabularnewline \hline
		\endfirsthead
		\centering{UI-1} & UI dapat menampilkan status pendeteksian regresi  \\ \hline
		\centering{UI-2} & UI dapat menampilkan hasil analisis \textit{critical path} berupa nama operasi dan nilai \textit{latency} \\ \hline

	\end{longtable}
\end{small}

\subsubsection{Hasil implementasi}
Berikut adalah tangkapan layar hasil implementasi komponen \textit{User Interface}.

\pagebreak

\section{Pengujian}
Untuk mengetahui keberhasilan dari sistem yang telah diimplementasikan, akan dilakukan pengujian sebagai berikut.
\subsection{Tujuan Pengujian}
Pengujian sistem PRA akan dilakukan dengan tujuan untuk:
\begin{enumerate}
	\item Mengukur keberhasilan sistem PRA dalam mendeteksi regresi
	\item Mengukur keberhasilan sistem PRA dalam menentukan kandidat sumber regresi
	\item Mengukur \textit{overhead} yang diakibatkan oleh sistem PRA dengan
	indikator berupa penggunaan memori dan pemanfaatan CPU.
\end{enumerate}

Tujuan utama pengujian sistem \textit{Performance Regression Analysis} adalah untuk menguji apakah sistem dapat mendeteksi terjadinya regresi atau penurunan kinerja dengan indikasi peningkatan nilai \textit{latency} yang disebabkan oleh perubahan yang terjadi pada aplikasi. Perubahan tersebut dapat berupa penambahan atau \textit{update} fitur, hasil perbaikan \textit{bug}, dsb. Sehingga kasus-kasus yang akan diujikan utamanya merupakan simulasi terjadinya perubahan pada level aplikasi \textit{microservice}. Namun sebagai perbandingan akan terdapat juga kasus-kasus yang menyimulasikan perubahan yang terjadi di luar level aplikasi seperti peningkatan jumlah pengguna dan peningkatan \textit{load} pada Load Generator pada \textit{service}-\textit{service} tertentu. Kasus pengujian akan dijelaskan lebih lanjut pada subbab \ref{metode-pengujian}.
\pagebreak

\subsection{Lingkungan Pengujian}
Pengujian akan dilakukan pada Google Kubernetes Engine dengan spesifikasi yang dipaparkan pada tabel \ref{testing-env}.
\begin{small}
	\begin{longtable}{ | p{5cm} | p{8cm} | }
		\caption{Spesifikasi Lingkungan Pengujian}
		\label{testing-env}                                                           
		\\ \hline
		\centering\bfseries{Layanan Kubernetes} & \centering\bfseries{Google Kubernetes Engine} \tabularnewline \hline
		\endfirsthead
		\textit{Operating System} & Container Optimized OS (COS) \\ \hline
		\textit{Instance Type} & n1-standard-2 (2 vCPU,  7,5 GiB RAM) \\ \hline
		Jumlah Node & 3 \\ \hline
		
	\end{longtable}
\end{small}

\subsection{Aplikasi Pengujian}
Aplikasi yang akan digunakan untuk melakukan pengujian sistem PRA adalah aplikasi Hipster Shop yang merupakan aplikasi \textit{e-commerce} berbasis web yang dibuat untuk mendemostrasikan berbagai macam teknologi yang dimiliki oleh Google. Seperti pada gambar \ref{butiq-arch}, Hipster Shop terdiri atas 10 microservice yang saling berkomunikasi melalui gRPC. Selain itu Hipster Shop juga memiliki load generator yang secara terus menerus mengirimkan request untuk menyimulasikan alur belanja pengguna. Hipster Shop sudah memiliki load generator yang dibuat menggunakan Locust
untuk menyimulasikan pengunaan aplikasi oleh sejumlah user. Aplikasi Hipster Shop akan dimodifikasi dengan diinstumentasikan menggunakan Zipkin untuk menyimulasikan lingkungan \textit{distributed tracing} yang menggunakan Zipkin.

\begin{figure}[!htb]
	\centering
	\includegraphics[width=1\textwidth]{resources/ch4/hipster-arch.png}
	\caption{Arsitektur aplikasi Hipster Shop}
	\label{butiq-arch}
\end{figure}

\subsection{Metode Pengujian}
\label{metode-pengujian}
Untuk melakukan pengujian pada sistem PRA yang telah dibuat, \textit{service} yang ada pada Hipster Shop akan dimodifikasi untuk meniru perilaku dari \textit{service} yang mengalami regresi kinerjan. Ada dua metode utama yang dapat dilakukan untuk membuat perilaku regresi pada \textit{service} yaitu dengan menambahkan perintah \textit{sleep} dan menambahkan \textit{loop} dengan operasi \textit{dereference} sebuah variabel yang membutuhkan banyak usaha dari CPU untuk menyelesaikannya sehingga diharapkan fungsi-fungsi tersebut akan memiliki \textit{latency} yang lebih besar. 

Ada dua tahap yang akan dilakukan untuk melakukan pengujian ini, yaitu pertama tahap pengambilan data \textit{baseline}. Data \textit{baseline} akan diambil dengan menyimulasikan pengunaan aplikasi dengan menggunakan Load Generator yang akan dijalankan dengan variabel \textbf{25 User} dan \textbf{5 Spawn Rate} selama \textbf{1 jam}. Data \textit{baseline} ini kemudian akan disimpan oleh \textit{engine} untuk dipergunakan di kasus-kasus pengujian selanjutnya. Tahap selanjutnya adalah melakukan pengujian dengan kasus-kasus yang akan dijelaskan lebih lanjut.

Terdapat dua jenis \textit{testcase} yang akan diujikan pada sistem. Jenis pertama adalah kasus-kasus yang menyimulasikan regresi dengan melakukan perubahan yang terjadi di level aplikasi dengan cara melakukan modifikasi secara sengaja di level kode. Kasus pertama akan memiliki ID dengan awalan \textbf{I}. Jenis kedua adalah kasus-kasus yang menyimulasikan regresi dengan meningkatkan jumlah pengguna dan \textit{load} pada Load Generator. Kasus kedua akan memliki ID dengan awalan \textbf{E}. Tabel \ref{testcase-i} akan menjelaskan kasus-kasus pengujian sistem.

% jenis satu, dan tabel \ref{testcase-e} akan menjelaskan kasus-kasus pengujian sistem jenis dua.


%		\begin{longtable}[l]{  @{}| p{0.7cm} | p{5cm} | p{2cm} | p{2cm} | p{4cm} | @{} }
%			\caption{Kasus-kasus pengujian}
%			\label{testcase-i}                                                           
%			\\ \hline
%			\centering\bfseries{ID} & \centering\bfseries{Service (Fungsi)} & \centering\bfseries{Extra Latency} & \centering\bfseries{User} & \centering\bfseries{Keterangan} \tabularnewline \hline
%			\endfirsthead
%			I1 & CheckoutService (placeorder) &  \centering{100ms} & \centering{25} & Kasus extra latency \#1 \\ \hline
%				
%		\end{longtable}


	% Please add the following required packages to your document preamble:
	% \usepackage{multirow}
	% \usepackage[table,xcdraw]{xcolor}
	% If you use beamer only pass "xcolor=table" option, i.e. \documentclass[xcolor=table]{beamer}
	\begin{table}[!h]
		\begin{tabular}{llccl}
			\hline
			\rowcolor[HTML]{009901} 
			\multicolumn{1}{c}{\cellcolor[HTML]{009901}{\color[HTML]{FFFFFF} \textbf{ID}}} &
			\multicolumn{1}{c}{\cellcolor[HTML]{009901}{\color[HTML]{FFFFFF} \textbf{Service (Fungsi)}}} &
			\multicolumn{1}{l}{\cellcolor[HTML]{009901}{\color[HTML]{FFFFFF} \textbf{Extra Latency}}} &
			\multicolumn{1}{l}{\cellcolor[HTML]{009901}{\color[HTML]{FFFFFF} \textbf{User}}} &
			{\color[HTML]{FFFFFF} \textbf{Keterangan}} \\ \hline
			\multicolumn{1}{|l|}{I1} &
			\multicolumn{1}{l|}{CheckoutService (placeorder)} &
			\multicolumn{1}{c|}{100ms} &
			\multicolumn{1}{c|}{25} &
			\multicolumn{1}{l|}{Kasus extra latency \#1} \\ \hline
			\multicolumn{1}{|l|}{I2} &
			\multicolumn{1}{l|}{CheckoutService (placeorder)} &
			\multicolumn{1}{c|}{250ms} &
			\multicolumn{1}{c|}{25} &
			\multicolumn{1}{l|}{Kasus extra latency \#2} \\ \hline
			\multicolumn{1}{|l|}{I3} &
			\multicolumn{1}{l|}{CheckoutService (placeorder)} &
			\multicolumn{1}{c|}{350ms} &
			\multicolumn{1}{c|}{25} &
			\multicolumn{1}{l|}{Kasus extra latency \#3} \\ \hline
			\multicolumn{1}{|l|}{I4} &
			\multicolumn{1}{l|}{CheckoutService (placeorder)} &
			\multicolumn{1}{c|}{250ms} &
			\multicolumn{1}{c|}{75} &
			\multicolumn{1}{l|}{\begin{tabular}[c]{@{}l@{}}Kasus extra latency \\ dengan peningkatan user \#1\end{tabular}} \\ \hline
			\multicolumn{1}{|l|}{I5} &
			\multicolumn{1}{l|}{CheckoutService (placeorder)} &
			\multicolumn{1}{c|}{250ms} &
			\multicolumn{1}{c|}{150} &
			\multicolumn{1}{l|}{\begin{tabular}[c]{@{}l@{}}Kasus extra latency \\ dengan peningkatan user \#1\end{tabular}} \\ \hline
			\multicolumn{1}{|l|}{I6} &
			\multicolumn{1}{l|}{CheckoutService (placeorder)} &
			\multicolumn{1}{c|}{-} &
			\multicolumn{1}{c|}{25} &
			\multicolumn{1}{l|}{Kasus peningkatan kerja CPU} \\ \hline
			\multicolumn{1}{|l|}{} &
			\multicolumn{1}{l|}{CheckoutService (placeorder)} &
			\multicolumn{1}{c|}{} &
			\multicolumn{1}{c|}{} &
			\multicolumn{1}{l|}{} \\ \cline{2-2}
			\multicolumn{1}{|l|}{} &
			\multicolumn{1}{l|}{\begin{tabular}[c]{@{}l@{}}ProductCatalog (getproduct\\ listproducts)\end{tabular}} &
			\multicolumn{1}{c|}{} &
			\multicolumn{1}{c|}{} &
			\multicolumn{1}{l|}{} \\ \cline{2-2}
			\multicolumn{1}{|l|}{} &
			\multicolumn{1}{l|}{ShippingService (getquote)} &
			\multicolumn{1}{c|}{} &
			\multicolumn{1}{c|}{} &
			\multicolumn{1}{l|}{} \\ \cline{2-2}
			\multicolumn{1}{|l|}{\multirow{-4}{*}{I7}} &
			\multicolumn{1}{l|}{\begin{tabular}[c]{@{}l@{}}RecommendationService \\ (listrecommendations)\end{tabular}} &
			\multicolumn{1}{c|}{\multirow{-4}{*}{250ms}} &
			\multicolumn{1}{c|}{\multirow{-4}{*}{25}} &
			\multicolumn{1}{l|}{\multirow{-4}{*}{\begin{tabular}[c]{@{}l@{}}Kasus extra latency\\  dengan beberapa service\end{tabular}}} \\ \hline
		\end{tabular}
	\end{table}



\subsection{Hasil Pengujian}
Berikut adalah hasil pengujian yang telah dilakukan pada sistem PRA.

\subsection{Analisis Hasil Pengujian}

\chapter{Kesimpulan dan Saran}
Bab ini berisi hal-hal yang dapat disimpulkan dari pelaksanaan Tugas Akhir ini. Bab ini juga mencakup saran untuk pengembangan Tugas Akhir ini di masa mendatang.

\section{Kesimpulan}
Berdasarkan hasil pengembangan dan pengujian sistem \textit{Performance Regression Analysis} (PRA) yang telah dilakukan. Berikut ini adalah kesimpulan yang diperoleh.
\begin{enumerate}
	\item Telah berhasil dilakukan pendeteksian regresi pada kinerja aplikasi berbasis Microservice.
	\item Telah berhasil dilakukan analisis untuk menentukan akar penyebab regresi pada aplikasi berbasis Microservice.
	\item Hasil pengujian menunjukkan sistem PRA telah berhasil menguji 10 dari 11 kasus regresi atau sekitar 90\% kasus. Satu-satunya kasus regresi gagal terdeteksi adalah kasus dengan penambahan \textit{latency} sebesar 100ms. Pada kasus ini regresi tidak terdeteksi karena algoritma tidak menganggap penambahan \textit{latency} sebesar 100ms menjadikan sampel data menjadi berbeda distribusinya dengan sampel data \textit{baseline} sehingga sistem tidak dapat mendeteksi terjadinya regresi.
	\item Pendeteksian regresi dapat dilakukan menggunakan tes statistik Kolmogorov-Smirnov dengan membandingkan data \textit{latency} kasus regresi dengan data \textit{latency} \textit{baseline} yang merepresentasikan kinerja aplikasi Microservice pada keadaan normal.
	\item Analisis penyebab regresi belum dapat ditentukan dari hasil pendeteksian regresi oleh algoritma tes Kolmogorov-Smirnov namun dapat ditentukan dari hasil analisis Critical Path yang membandingkan selisih data \textit{latency} operasi pada kasus regresi dan \textit{baseline}.
	\item Menurut pengujian, komponen PRA Engine mengakibatkan \textit{overhead} CPU sebanyak 0,78 \% dan Memory sebanyak 0,67 \% kepada \textit{cluster} Kubernetes. \textit{Overhead} yang diakibatkan oleh kakas termasuk rendah sehingga penggunaan kakas PRA Engine tidak akan berdampak pada kinerja \textit{cluster} Kubernetes secara keseluruhan.\textbf{}
\end{enumerate}

\section{Saran}
Saran yang dapat diberikan untuk pengembangan di masa mendatang adalah sebagai berikut:
\begin{enumerate}
	\item Pada pengembangan kakas selanjutnya dapat ditambahkan penanganan untuk pengujian pada lingkungan aplikasi selain Kubernetes
%	\item Menggunakan solusi \textit{distributed tracing} selain Zipkin untuk melakukan analisis serupa sebab operasi \textit{query} pada API Zipkin menjadi \textit{bottleneck} yang menjadikan pemanggilan API \textit{engine} lambat
	\item Pengembangan kakas menggunakan bahasa pemrograman yang lebih cepat dibandingkan Python seperti Go atau Rust
\end{enumerate}
%----------------------------------------------------------------%

% Daftar pustaka
% Bibliography to Daftar Pustaka
\renewcommand{\bibname}{Daftar Pustaka}
\cleardoublepage
\phantomsection
\addcontentsline{toc}{chapter}{DAFTAR PUSTAKA}
%\printbibliography
%\bibliography{references}
%ZCZC RMS 202107023
%\bibstyle{apa}
\bibliography{references}
\bibliographystyle{apalike}

\backmatter
% Index
\appendix
\addtocontents{toc}{\protect\setcounter{tocdepth}{2}}

\cleardoublepage
\phantomsection
%\part*{Lampiran}
%\addcontentsline{toc}{part}{LAMPIRAN}

% Setting judul appendix
\chapterfont{\Large}
\titleformat{\chapter}[hang]
{\Large\bfseries}
{\chaptertitlename\ \thechapter.\ }{0pt}
{\Large\bfseries}
\titlespacing*{\chapter}{0pt}{-25pt}{10pt}

\chapter{Lampiran A. Tangkapan Layar Hasil Pengujian}

\begin{figure}[!htb]
	\centering
	\includegraphics[width=0.75\textwidth]{resources/ch4/json/1.png}
	\caption{\textit{Response} JSON hasil pengujian kasus \textbf{SI1}}
	\label{result_json_1}
\end{figure}

%\begin{figure}[!htb]
%	\centering
%	\includegraphics[width=1\textwidth]{resources/ch4/log/1-log.png}
%	\caption{\textit{Log} hasil pengujian kasus SI1}
%	\label{result_log_1}
%\end{figure}

\begin{figure}[!htb]
	\centering
	\includegraphics[width=0.75\textwidth]{resources/ch4/json/2.png}
	\caption{\textit{Response} JSON hasil pengujian kasus \textbf{SI2}}
	\label{result_json_2}
\end{figure}

%\begin{figure}[!htb]
%	\centering
%	\includegraphics[width=1\textwidth]{resources/ch4/log/2-log.png}
%	\caption{\textit{Log} hasil pengujian kasus SI2}
%	\label{result_log_2}
%\end{figure}

\begin{figure}[!htb]
	\centering
	\includegraphics[width=0.75\textwidth]{resources/ch4/json/3.png}
	\caption{\textit{Response} JSON hasil pengujian kasus \textbf{SI3}}
	\label{result_json_3}
\end{figure}

%\begin{figure}[!htb]
%	\centering
%	\includegraphics[width=1\textwidth]{resources/ch4/log/3-log.png}
%	\caption{\textit{Log} hasil pengujian kasus SI3}
%	\label{result_log_3}
%\end{figure}
%\pagebreak

\begin{figure}[!htb]
	\centering
	\includegraphics[width=0.75\textwidth]{resources/ch4/json/4.png}
	\caption{\textit{Response} JSON hasil pengujian kasus \textbf{SI4}}
	\label{result_json_4}
\end{figure}

%\begin{figure}[!htb]
%	\centering
%	\includegraphics[width=1\textwidth]{resources/ch4/log/4-log.png}
%	\caption{\textit{Log} hasil pengujian kasus SI4}
%	\label{result_log_4}
%\end{figure}

\begin{figure}[!htb]
	\centering
	\includegraphics[width=0.75\textwidth]{resources/ch4/json/5.png}
	\caption{\textit{Response} JSON hasil pengujian kasus \textbf{SI5}}
	\label{result_json_5}
\end{figure}

%\begin{figure}[!htb]
%	\centering
%	\includegraphics[width=1\textwidth]{resources/ch4/log/5-log.png}
%	\caption{\textit{Log} hasil pengujian kasus \textbf{SI5}}
%	\label{result_log_5}
%\end{figure}

\begin{figure}[!htb]
	\centering
	\includegraphics[width=0.75\textwidth]{resources/ch4/json/6.png}
	\caption{\textit{Response} JSON hasil pengujian kasus \textbf{SI6}}
	\label{result_json_6}
\end{figure}

%\begin{figure}[!htb]
%	\centering
%	\includegraphics[width=1\textwidth]{resources/ch4/log/6-log.png}
%	\caption{\textit{Log} hasil pengujian kasus SI6}
%	\label{result_log_6}
%\end{figure}
%\pagebreak

\begin{figure}[!htb]
	\centering
	\includegraphics[width=0.75\textwidth]{resources/ch4/json/7.png}
	\caption{\textit{Response} JSON hasil pengujian kasus \textbf{SI7}}
	\label{result_json_7}
\end{figure}
%
%\begin{figure}[!htb]
%	\centering
%	\includegraphics[width=1\textwidth]{resources/ch4/log/7-log.png}
%	\caption{\textit{Log} hasil pengujian kasus SI7}
%	\label{result_log_7}
%\end{figure}

\begin{figure}[!htb]
	\centering
	\includegraphics[width=0.75\textwidth]{resources/ch4/json/8.png}
	\caption{\textit{Response} JSON hasil pengujian kasus \textbf{SE1}}
	\label{result_json_8}
\end{figure}

%\begin{figure}[!htb]
%	\centering
%	\includegraphics[width=1\textwidth]{resources/ch4/log/8-log.png}
%	\caption{\textit{Log} hasil pengujian kasus SE1}
%	\label{result_log_8}
%\end{figure}

\begin{figure}[!htb]
	\centering
	\includegraphics[width=0.75\textwidth]{resources/ch4/json/9.png}
	\caption{\textit{Response} JSON hasil pengujian kasus \textbf{SE2}}
	\label{result_json_9}
\end{figure}

%\begin{figure}[!htb]
%	\centering
%	\includegraphics[width=1\textwidth]{resources/ch4/log/9-log.png}
%	\caption{\textit{Log} hasil pengujian kasus SE2}
%	\label{result_log_9}
%\end{figure}

\begin{figure}[!htb]
	\centering
	\includegraphics[width=0.75\textwidth]{resources/ch4/json/10.png}
	\caption{\textit{Response} JSON hasil pengujian kasus \textbf{SE3}}
	\label{result_json_10}
\end{figure}
%\begin{figure}[!htb]
%	\centering
%	\includegraphics[width=1\textwidth]{resources/ch4/log/10-log.png}
%	\caption{\textit{Log} hasil pengujian kasus SE3}
%	\label{result_log_10}
%\end{figure}

\begin{figure}[!htb]
	\centering
	\includegraphics[width=0.75\textwidth]{resources/ch4/json/11.png}
	\caption{\textit{Response} JSON hasil pengujian kasus \textbf{SE4}}
	\label{result_json_9}
\end{figure}

%\begin{figure}[!htb]
%	\centering
%	\includegraphics[width=1\textwidth]{resources/ch4/log/11-log.png}
%	\caption{\textit{Log} hasil pengujian kasus SE4}
%	\label{result_log_11}
%\end{figure}
\chapter{Lampiran B. Tangkapan Layar Hasil Pengujian Kinerja}



\end{document}

