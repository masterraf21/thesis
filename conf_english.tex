\documentclass[conference]{configs/IEEEtran}
\IEEEoverridecommandlockouts
% The preceding line is only needed to identify funding in the first footnote. If that is unneeded, please comment it out.
\usepackage{cite}
\usepackage{amsmath,amssymb,amsfonts}
\usepackage{algorithmic}
\usepackage{graphicx}
\usepackage{textcomp}
\usepackage{xcolor}
\def\BibTeX{{\rm B\kern-.05em{\sc i\kern-.025em b}\kern-.08em
    T\kern-.1667em\lower.7ex\hbox{E}\kern-.125emX}}
\begin{document}

\title{Development of Performance Regression Analysis Tool using Distributed Tracing on Microservice-based Application\\
%{\footnotesize \textsuperscript{*}Note: Sub-titles are not captured in Xplore and should not be used}
%\thanks{Identify applicable funding agency here. If none, delete this.}
}

\author{\IEEEauthorblockN{Rafi Abbel Mohammad}
\IEEEauthorblockA{\textit{School of Electrical Engineering and Informatics} \\
\textit{Institut Teknologi Bandung}\\
Bandung, Indonesia \\
masterraf21@gmail.com}
%\and
%\IEEEauthorblockN{Achmad Imam Kistijantoro}
%\IEEEauthorblockA{\textit{School of Electrical Engineering and Informatics} \\
%\textit{ITB, Indonesia}\\
%City, Country \\
%imam@stei.itb.ac.id}
%\and
%\IEEEauthorblockN{3\textsuperscript{rd} Given Name Surname}
%\IEEEauthorblockA{\textit{dept. name of organization (of Aff.)} \\
%\textit{name of organization (of Aff.)}\\
%City, Country \\
%email address or ORCID}
%\and
%\IEEEauthorblockN{4\textsuperscript{th} Given Name Surname}
%\IEEEauthorblockA{\textit{dept. name of organization (of Aff.)} \\
%\textit{name of organization (of Aff.)}\\
%City, Country \\
%email address or ORCID}
%\and
%\IEEEauthorblockN{5\textsuperscript{th} Given Name Surname}
%\IEEEauthorblockA{\textit{dept. name of organization (of Aff.)} \\
%\textit{name of organization (of Aff.)}\\
%City, Country \\
%email address or ORCID}
%\and
%\IEEEauthorblockN{6\textsuperscript{th} Given Name Surname}
%\IEEEauthorblockA{\textit{dept. name of organization (of Aff.)} \\
%\textit{name of organization (of Aff.)}\\
%City, Country \\
%email address or ORCID}
}

\maketitle

\begin{abstract}
In applications with Microservice architecture, when a significant change occurs and resulting in decreased performance or regression, it is difficult to perform analysis to check which part of the whole Microservice is the main cause of the regression due to the distributed nature of Microservice. With the help of
distributed tracing can be analyzed to determine whether there is a regression and the cause of the regression by utilizing latency data from existing operations
on Microservices applications.

This system implements a Performance Regression Analysis system or
Performance Regression Analysis using distributed tracing tools
zipkin. Regression detection is carried out using statistical analysis
Kolmogorov-Smirnov who will compare the latency data samples that occur
periodically and sample latency baseline data that represents performance
application under normal circumstances. The analysis will compare Cummulative
Distribution Function (CDF) of the two samples and see if both
CDF comes from a different distribution. If it is found that both CDFs originate
from different distributions, it can be suspected that there has been a performance regression because
the distribution of the periodic data has deviated from the distribution of the baseline data.
If a regression is detected, then the system will then perform a critical analysis
path is to see which operations of each service are most likely
contributed to the regression. The analysis will be carried out by finding the difference
latency of periodic and baseline operation data and it will be seen which operation is
the difference in latency exceeds a predetermined limit.
The Performance Regression Analysis system has been tested on a Microservice application that
running on Kubernetes. The result is that out of 10 out of 11 test cases, the regression is successful
detected and in 6 out of 10 cases, surgery to determine the cause of the regression was successful.
Implementing the system adds an average CPU usage overhead of
0.78% and 0.67% Memory usage
\end{abstract}

\begin{IEEEkeywords}
performance regression, Distributed Tracing, Microservice, Kubernetes
\end{IEEEkeywords}

\section{Introduction}
This document is a model and instructions for \LaTeX.
Please observe the conference page limits. Hey parker \cite{Bansal2014}. \cite{parker2020distributed}

\section{Related Works}

\section{Proposed Solution}

\section{Experiment}

\section{Analysis}

\section{Conclusion \& Future Works}

%\section*{Acknowledgment}
%
%The preferred spelling of the word ``acknowledgment'' in America is without 
%an ``e'' after the ``g''. Avoid the stilted expression ``one of us (R. B. 
%G.) thanks $\ldots$''. Instead, try ``R. B. G. thanks$\ldots$''. Put sponsor 
%acknowledgments in the unnumbered footnote on the first page.

\bibliographystyle{configs/IEEEtran}
\bibliography{references}

\vspace{12pt}


\end{document}
