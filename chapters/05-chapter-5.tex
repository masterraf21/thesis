sd\chapter{Kesimpulan dan Saran}
Bab ini berisi hal-hal yang dapat disimpulkan dari pelaksanaan Tugas Akhir ini. Bab ini juga mencakup saran untuk pengembangan Tugas Akhir ini di masa mendatang.

\section{Kesimpulan}
Berdasarkan hasil pengembangan kakas pengumpulan data menggunakan \textit{spreadsheet} yang telah dilakukan. Berikut adalah kesimpulan yang diperoleh.
\begin{enumerate}
	\item Telah berhasil dilakukan penambahan fitur pada aplikasi EtherCalc yang dapat melakukan pengumpulan data ke dalam bentuk basis data.
	% \item Konflik pada kolaborasi dapat ditangani dengan baik oleh sistem pada EtherCalc sehingga penanganan konflik tidak perlu dibuat kembali.
	\item Data yang akan dimasukkan ke basis data penyimpanan berhasil divalidasi menggunakan fitur yang dibuat dengan tiga tipe validasi yakni tipe data, domain data, dan relasi data.
	\item Identifikasi tabel pada suatu \textit{sheet} dapat dilakukan dengan menggunakan algoritma kNN dengan mencari kedekatan antar sel. Identifikasi label suatu baris pada tabel dapat dilakukan dengan teknik \textit{frame finder} dengan membagi label menjadi empat jenis yakni \textit{title}, \textit{data}, \textit{header}, dan \textit{footer}. Jika teknik \textit{frame finder} tidak berhasil menemukan label dan data, maka pengguna dapat memasukkan \textit{metadata table} secara manual dan mengubahnya sesuai dengan keinginan pengguna.
	\item Penggabungan data antar \textit{spreadsheet} dapat dilakukan dengan fitur yang dibuat dan dapat digabungkan secara horizontal, vertikal, maupun gabungan keduanya. Data pada \textit{spreadsheet} berhasil dimasukkan ke dalam basis data yang ditentukan sesuai dengan \textit{metadata table} yang telah dibuat pengguna maupun hasil pencarian otomatis dari algoritma \textit{frame finder}.
	\item Alur kerja pengumpulan data berubah sehingga dapat diusulkan alur kerja baru dimana \textit{versioning} dilakukan oleh aplikasi EtherCalc karena seluruh data berada pada satu tempat menggunakan mekanisme penyimpanan oleh EtherCalc. Pada saat pengumpulan data dari berbagai \textit{spreadsheet} dapat dilakukan menggunakan aplikasi yang sama yakni EtherCalc, sehingga pada alur kerja yang diusulkan, pengumpulan data tidak memerlukan bantuan aplikasi lain ataupun manual. Hasil akhir dari pengumpulan data merupakan data pada basis data sehingga data mudah diolah, ditampilkan, maupun diubah menggunakan banyak aplikasi yang tersedia.
\end{enumerate}

\section{Saran}
Saran yang dapat diberikan untuk pengembangan di masa mendatang adalah sebagai berikut:
\begin{enumerate}
	\item Pada pembangunan selanjutnya dapat ditambahkan penanganan kasus penggunaan \textit{spreadsheet} selain \textit{data frame} dan relasi.
	\item Penambahan data pembelajaran untuk identifikasi label baris dapat dilakukan sehingga akan memperbaiki hasil identifikasi otomatis. Pada pengembangan selanjutnya dapat ditambahkan \textit{feedback} dari pengguna sebagai data pembelajaran.
	\item Menambahkan fungsionalitas yakni memperbolehkan kolom \textit{key} yang tidak hanya satu pada \textit{metadata table}.
	\item Pengembangan fitur validasi, contohnya adalah menambahkan jenis validasi contohnya validasi masukan berbentuk formula. Di samping itu, dapat ditambahkan jenis validasi pada validasi tipe seperti tipe tanggal. Dapat juga penambahan fitur pada validasi domain seperti atribut yang dapat menerima tidak hanya satu aturan validasi domain.
	\item Penanganan jenis tabel dengan \textit{header} yang berada di kiri dan kanan data mungkin dapat dikembangkan menggunakan teknik \textit{transpose}.

\end{enumerate}