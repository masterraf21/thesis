\chapter{Kesimpulan dan Saran}
Bab ini berisi hal-hal yang dapat disimpulkan dari pelaksanaan Tugas Akhir ini. Bab ini juga mencakup saran untuk pengembangan Tugas Akhir ini di masa mendatang.

\section{Kesimpulan}
Berdasarkan hasil pengembangan dan pengujian sistem \textit{Performance Regression Analysis} (PRA) yang telah dilakukan. Berikut ini adalah kesimpulan yang diperoleh.
\begin{enumerate}
	\item Telah berhasil dilakukan pendeteksian regresi pada kinerja aplikasi berbasis microservice menggunakan \textit{engine} PRA.
	\item Telah berhasil dilakukan analisis untuk menentukan akar penyebab regresi pada aplikasi berbasis microservice.
	\item Hasil pengujian menunjukkan sistem PRA telah berhasil menguji 90\% kasus regresi.
	\item Pendeteksian regresi dapat dilakukan menggunakan tes statistik Kolmogorov-Smirnov dengan membandingkan data \textit{latency} kasus regresi dengan data \textit{latency} \textit{baseline} yang merepresentasikan kinerja aplikasi Microservice di keadaan normal.
	\item Analisis penyebab regresi belum dapat ditentukan dari hasil pendeteksian regresi oleh algoritme tes Kolmogorov-Smirnov namun dapat ditentukan dari hasil analisis Critical Path yang membandingkan selisih data \textit{latency} operasi pada kasus regresi dan \textit{baseline}.

\end{enumerate}

\section{Saran}
Saran yang dapat diberikan untuk pengembangan di masa mendatang adalah sebagai berikut:
\begin{enumerate}
	\item Pada pengembangan kakas selanjutnya dapat ditambahkan penanganan untuk pengujian pada lingkungan aplikasi selain Kubernetes
	\item Pengembangan kakas menggunakan bahasa pemrograman yang lebih cepat dibandingkan Python seperti Go atau Rust
	\item Pengembangan fitur UI untuk melakukan \textit{fetching} data hasil analisis secara \textit{real-time}
	
%	\item Pada pembangunan selanjutnya dapat ditambahkan penanganan kasus penggunaan \textit{spreadsheet} selain \textit{data frame} dan relasi.
%	\item Penambahan data pembelajaran untuk identifikasi label baris dapat dilakukan sehingga akan memperbaiki hasil identifikasi otomatis. Pada pengembangan selanjutnya dapat ditambahkan \textit{feedback} dari pengguna sebagai data pembelajaran.
%	\item Menambahkan fungsionalitas yakni memperbolehkan kolom \textit{key} yang tidak hanya satu pada \textit{metadata table}.
%	\item Pengembangan fitur validasi, contohnya adalah menambahkan jenis validasi contohnya validasi masukan berbentuk formula. Di samping itu, dapat ditambahkan jenis validasi pada validasi tipe seperti tipe tanggal. Dapat juga penambahan fitur pada validasi domain seperti atribut yang dapat menerima tidak hanya satu aturan validasi domain.
%	\item Penanganan jenis tabel dengan \textit{header} yang berada di kiri dan kanan data mungkin dapat dikembangkan menggunakan teknik \textit{transpose}.

\end{enumerate}