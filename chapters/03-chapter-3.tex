\chapter{Analisis Masalah dan Perancangan Solusi Visualisasi \textit{distributed tracing}}

%Pada bab ini diuraikan analisis persoalan pengumpulan data pada \textit{spreadsheet} yang telah diuraikan pada Bab I. Hasil dari bab ini digunakan untuk merancang kakas yang akan diimplementasikan seperti yang dijelaskan pada Bab IV.
%Berdasarkan 


\section{Analisis Masalah}

 Pada masa sebelum adopsi arsitektur Microservice dan kebanyakan dari aplikasi masih menggunakan arsitektur Monolith, proses seperti \textit{debugging} adalah hal yang sederhana sebab jika terdapat suatu \textit{error} akan mudah untuk ditelusuri dari mana asal \textit{error} tersebut sebab hanya ada satu aplikasi yang digunakan. Hal tersebut tidak berlaku jika aplikasi menggunakan model Sistem Terdistribusi, salah satu contohnya adalah Microservice. Sifat dari Microservice yang melakukan \textit{decoupling} aplikasi menjadi bagian yang lebih kecil membuat proses \textit{debugging} menjadi tidak mudah sebab untuk mencari penyebab \textit{error} aplikasi yang terdistribusi, kita harus mengetahui terlebih dahulu sumber dari \textit{error} tersebut. Kompleksitas akan bertambah dalam proses debugging jika ternyata ditemukan bahwa sautu \textit{error} pada sebuah \textit{service} bukanlah akar atau penyebab utama dari \textit{error} tersebut melainkan suatu \textit{service} lainnya. Kompleksitas akan bertambah jika metode \textit{debugging} yang digunakan mengharuskan \textit{developer} yang menangani \textit{error} tersebut harus menelusuri satu per satu \textit{service} yang terdampak sampai menemukan akar dari masalahnya. Dari masalah tersebutlah timbul suatu kebutuhan untuk mendapatkan gambaran mengenai \textit{state} sebuah Sistem Terdistribusi ataupun yang disebut juga dengan \textit{observability}.
 
 Menurut Sridharan, ada tiga pilar untuk mencapai \textit{observability} pada sebuah Sistem Terdistribusi, yaitu melalui \textit{log}, \textit{metric}, dan \textit{trace} \citep{sridharan2018distributed}. \textit{Log} atau \textit{event log} adalah suatu catatan yang bersifat \textit{immutable} dari \textit{event} yang terjadi sepanjang waktu. Sebuah \textit{event log} pada umumnya terdiri dari informasi mengenai \textit{timestamp} dan \textit{payload} yang berisikan konteks. \textit{Metric} dan juga \textit{Trace} merupakan abstraksi yang dibuat di atas \textit{log} yang melakukan pra-pemrosesan dan melakukan dekode informasi berdasarkan dua sumbu, yang satu bersifat \textit{request} sentris yaitu \textit{trace}, dan yang lainnya bersifat sistem sentris yaitu \textit{metric}.
 
 Dengan bantuan dari \textit{trace} atau teknik yang disebut \textit{distributed tracing}, \textit{developer} bisa mendapatkan suatu gambaran dari masing-masing \textit{request} yang terjadi seperti yang sudah dijelaskan pada \ref{bab2-dtracing}
 
%  \textit{Log} tersebut dihasilkan oleh serangkaian \textit{event} yang terjadi pada sebuah sistem 
% 
% 
%
%Seiring dengan meningkatnya penggunaan arsitektur, mengingkat pula kebutuhan bagi para \textit{developer} untuk dapat dengan segera mengetahui sumber dari permasalahan jika terjadi \textit{error} pada sistem.
%
%Dari pemaparan mengenai masih kurangnya kakas visualisasi \textit{tracing} yang bersifat \textit{open source}
%. Berdasarkan studi literatur mengenai \textit{observability} untuk Sistem Terdistribusi pada \ref{bab2-observability}, terdapat 

%Isu visualisasi 
%Overhead


\section{Analisis Alternatif Solusi}

\section{Rancangan Solusi}

%Overhead disini

