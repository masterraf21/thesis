\chapter{Analisis Masalah dan Perancangan Solusi Visualisasi \textit{distributed tracing}}

%Pada bab ini diuraikan analisis persoalan pengumpulan data pada \textit{spreadsheet} yang telah diuraikan pada Bab I. Hasil dari bab ini digunakan untuk merancang kakas yang akan diimplementasikan seperti yang dijelaskan pada Bab IV.
%Berdasarkan 


\section{Analisis Masalah}

Seiring dengan meningkatnya penggunaan arsitektur, mengingkat pula kebutuhan bagi para \textit{developer} untuk dapat dengan segera mengetahui sumber dari permasalahan jika terjadi \textit{error} pada sistem. Pada masa sebelum adopsi arsitektur Microservice dan kebanyakan dari aplikasi masih menggunakan arsitektur Monolith, proses seperti \textit{debugging} adalah hal yang sederhana sebab jika terdapat suatu \textit{error} akan mudah untuk ditelusuri dari mana asal \textit{error} tersebut sebab hanya ada satu aplikasi yang digunakan. 

Dari pemaparan mengenai masih kurangnya kakas visualisasi \textit{tracing} yang bersifat \textit{open source}
. Berdasarkan studi literatur mengenai \textit{observability} untuk Sistem Terdistribusi pada \ref{bab2-observability}, terdapat 

%Isu visualisasi 
%Overhead


\section{Analisis Alternatif Solusi}

\section{Rancangan Solusi}

%Overhead disini

