\chapter{Implementasi, dan Pengujian}

Bab ini akan membahas seluruh proses implementasi yang dilakukan untuk menerapkan rancangan yang sudah didefinisikan sebelumnya. Selain itu, bab ini juga akan membahas pengujian yang dilakukan terhadap hasil implementasi yang mencakup hal-hal yang diuji, metode pengujian, dan hasil pengujian yang diperoleh. 


\section{Implementasi}
Implementasi sistem \textit{Performance Regression Analysis} (PRA) akan dibuat berdasarkan rancangan arsitektur seperti yang terdapat pada gambar \ref{arch-pra}. Beberapa komponen seperti \textit{library} instrumentasi dan juga \textit{User Interface} tidak akan dibuat dari awal, melainkan akan melakukan \textit{fork} dan modifikasi solusi \textit{Open Source} \textit{distributed tracing} dari Zipkin. Komponen yang akan dibuat dari awal sepenuhnya adalah \textit{engine} dari sistem PRA yang akan melakukan komputasi utama sistem pendeteksian dan analisis regresi. Komponen \textit{engine} juga akan berfungsi sebagai API yang akan diakses oleh komponen \textit{User Interface}.



\subsection{\textit{Performance Regression Analysis} \textit{engine}}
Implementasi komponen \textit{engine} ini akan dibagi menjadi beberapa modul, seperti yang terlihat pada tabel \ref{engine-module}.

\begin{small}
	\begin{longtable}{ | p{1cm} | p{3cm} | p{10cm} | }
		\caption{Tabel pembagian modul komponen \textit{engine}}
		\label{engine-module}                                                           
		\\ \hline
		\centering\bfseries{ID} & \centering\bfseries{Nama Modul} & \centering\bfseries{Deskripsi} \tabularnewline \hline
		\endfirsthead
		EM-1 & Pengambilan Data & Modul ini bertanggung jawab untuk mengambil data dari API Zipkin yang memiliki data hasil \textit{trace} dari aplikasi. Selain mengambil data \textit{trace} dari Zipkin, modul ini juga bertanggung jawab untuk mengambil model-model \textit{baseline} yang terdapat pada \textit{storage}. \\ \hline
		EM-2 & Transformasi Data & Modul ini bertanggung jawab untuk melakukan transformasi dari data \textit{trace} mentah yang diambil dari API Zipkin menjadi bentuk-bentuk model yang akan digunakan untuk komputasi di tahap selanjutnya seperti model \textit{Cumulative Distribution Function} (CDF), model data \textit{Critical Path}, dan sampel data \textit{trace}. \\ \hline
		EM-3 & Penyimpanan Data & Modul ini bertanggung jawab untuk menyimpan model hasil transformasi \textit{baseline} dari modul EM-2 ke \textit{storage} untuk digunakan kembali pada fase \textit{Real-time Analysis}. \\ \hline
		EM-4 & Perhitungan Statistik & Modul ini bertanggung jawab untuk melakukan komputasi perhitungan statistik yang mencakup pendeteksian regresi dengan menghitung koefisien Kolmogorov-Smirnov seperti yang telah dijelaskan pada subbab \ref{approach-cumulative}. \\ \hline
		EM-5 & Analisis Korelasi & Komponen \\ \hline
		EM-6 & Analisis \textit{Critical Path} & Komponen \\ \hline
		EM-7 & API & Komponen \\ \hline
		EM-8 & \textit{Scheduling} & Komponen \\ \hline
	\end{longtable}
\end{small}

%\begin{small}
%	\begin{longtable}{| p{2cm} | p{3cm} | p{10cm} |}
%		\caption{}
%		
%		\label{}                                                           
%		\\ \hline
%		\centering\bfseries{ID} & \centering\bfseries{Nama Modul} & \centering\bfseries{Deskripsi} \tabularnewline \hline
%		\endfirsthead
%		EM-1 &  & Komponen instrumentasi dapat menyediakan informasi kausalitas antar \textit{service} yang 
%	\end{longtable}
%\end{small}

\subsection{\textit{User Interface}}


\section{Pengujian}
