\chapter{Implementasi, dan Pengujian}

Bab ini akan membahas seluruh proses implementasi yang dilakukan untuk menerapkan rancangan yang sudah didefinisikan sebelumnya. Selain itu, bab ini juga akan membahas pengujian yang dilakukan terhadap hasil implementasi yang mencakup hal-hal yang diuji, metode pengujian, dan hasil pengujian yang diperoleh. 


\section{Implementasi}
Implementasi sistem \textit{Performance Regression Analysis} (PRA) akan dibuat berdasarkan rancangan arsitektur seperti yang terdapat pada gambar \ref{arch-pra}. Beberapa komponen seperti \textit{library} instrumentasi dan juga \textit{User Interface} tidak akan dibuat dari awal, melainkan akan melakukan \textit{fork} dan modifikasi solusi \textit{Open Source} \textit{distributed tracing} dari Zipkin. Komponen yang akan dibuat dari awal sepenuhnya adalah \textit{engine} dari sistem PRA yang akan melakukan komputasi utama sistem pendeteksian dan analisis regresi. Komponen \textit{engine} juga akan berfungsi sebagai API yang akan diakses oleh komponen \textit{User Interface}.

%Sementara komponen inti dari Tugas Akhir ini adalah \textit{engine} 
\begin{small}
	\begin{longtable}{ | p{3cm} | p{10cm} |}
		\caption{Kriteria pemilihan solusi \textit{tracing}}
		\label{ch3-trace-crit}                                                           
		\\ \hline
		\centering\bfseries{ID Kriteria} & \centering\bfseries{Deskripsi} \tabularnewline \hline
		\endfirsthead
		TC-1 & Komponen instrumentasi dapat menyediakan informasi kausalitas antar \textit{service} yang terjadi selama \textit{request} pada \textit{span} \\ \hline
		TC-2 & Komponen \textit{deployment} menyediakan dukungan untuk Kubernetes \\ \hline
		TC-3 & Komponen \textit{deplyoment} menyediakan dukungan untuk \textit{storage} eksternal \\ \hline
	\end{longtable}
\end{small}

\subsection{\textit{Performance Regression Analysis} \textit{engine}}
Implementasi komponen \textit{engine} ini akan dibagi menjadi beberapa modul, seperti yang terlihat pada tabel \ref{}.

\begin{small}
	\begin{longtable}{ | p{2cm} | p{3cm} | p{10cm} | }
		\caption{Tabel pembagian modul komponen \textit{engine}}
		\label{engine-module}                                                           
		\\ \hline
		\centering\bfseries{ID} & \centering\bfseries{Nama Modul} & \centering\bfseries{Deskripsi} \tabularnewline \hline
		\endfirsthead
		EM-1 & Pengambilan Data & Komponen \\ \hline
%		\textbf{client.browser}: mozilla68 & 111 ms & 548 ms & +437 ms \\ \hline
%		\textbf{db.instance}: cassandra.4 & 117 ms & 464 ms & +348 ms \\ \hline
%		\textbf{runinfo.host}: vm123 & 115 ms & 453 ms & +337 ms \\ \hline
	\end{longtable}
\end{small}

%\begin{small}
%	\begin{longtable}{| p{2cm} | p{3cm} | p{10cm} |}
%		\caption{}
%		
%		\label{}                                                           
%		\\ \hline
%		\centering\bfseries{ID} & \centering\bfseries{Nama Modul} & \centering\bfseries{Deskripsi} \tabularnewline \hline
%		\endfirsthead
%		EM-1 &  & Komponen instrumentasi dapat menyediakan informasi kausalitas antar \textit{service} yang 
%	\end{longtable}
%\end{small}

\subsection{\textit{User Interface}}


\section{Pengujian}
